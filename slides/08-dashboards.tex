\documentclass[aspectratio=169]{beamer}
\usetheme{metropolis}
\usepackage{appendixnumberbeamer}
\usepackage{booktabs, hyperref}

\input{mgmt675-style}

\subtitle{MGMT 675: Generative AI for Finance}
\title{Replacing Dashboards with Natural Language}
\author{Kerry Back}

\date{}

\begin{document}

\maketitle

% ========================================
% SECTION 1: THE PROBLEM WITH DASHBOARDS
% ========================================

\begin{frame}{The Dashboard Trap}
Organizations spend millions building dashboards that answer \alert{yesterday's questions}. When a new question arises, the cycle restarts: requirements gathering, design, development, QA, deployment.
\vspace{0.3cm}
\begin{columns}[T]
\begin{column}{0.5\textwidth}
\begin{shadedbox}[title=\textbf{The Typical Dashboard Lifecycle}]
\begin{enumerate}\small
  \item Business user requests a report
  \item Analyst translates to requirements
  \item Engineer builds SQL queries
  \item Designer creates visualizations
  \item IT deploys to Tableau/Power BI
  \item User asks a follow-up question
  \item \alert{Back to step 1}
\end{enumerate}
\end{shadedbox}
\end{column}
\begin{column}{0.5\textwidth}
\begin{shadedbox}[title=\textbf{The Cost}]
\begin{itemize}\small
  \item \textbf{Time}: Weeks to months per dashboard
  \item \textbf{Money}: Tableau/Power BI licenses, engineering hours, maintenance
  \item \textbf{Rigidity}: Fixed views of fixed data
\end{itemize}
\end{shadedbox}
\end{column}
\end{columns}
\end{frame}

\begin{frame}{Dashboard Fatigue is Real}
\begin{itemize}
  \item Gartner (2019): ``Through 2022, only \textbf{20\%} of analytic insights will deliver business outcomes'' (\href{https://designingforanalytics.com/resources/failure-rates-for-analytics-bi-iot-and-big-data-projects-85-yikes/}{Designing for Analytics summary})
  \item Most dashboards are built, viewed a few times, and abandoned
  \item The bottleneck isn't data --- it's the translation layer between \textit{questions} and \textit{answers}
\end{itemize}
\vspace{0.3cm}
The fundamental problem: dashboards answer \textbf{pre-defined questions}. But the most valuable analysis comes from \textbf{ad-hoc questions} that arise in the moment.
\vspace{0.3cm}
\begin{itemize}
  \item ``What happened to margins in the Southeast last quarter?''
  \item ``Show me our top 10 customers by growth rate, excluding one-time orders''
  \item ``Compare Q3 headcount vs.\ budget by department, and flag anyone over 110\%''
\end{itemize}
\vspace{0.3cm}
These are simple questions.  Getting answers shouldn't require a development cycle.
\end{frame}

% ========================================
% SECTION 2: THE NEW MODEL
% ========================================

\section{Natural Language as the Query Interface}

\begin{frame}{The Shift: From Dashboards to Conversations}
\begin{center}
\begin{tabular}{@{} l l l @{}}
\toprule
& \textbf{Traditional Dashboard} & \textbf{Natural Language AI} \\
\midrule
Query method & Click filters, select dates & Ask in plain English \\
Time to answer & Minutes to weeks & Seconds \\
Follow-up questions & New dashboard request & Next sentence \\
Visualization & Pre-built charts & Generated on demand \\
Data sources & Pre-configured & Connected via MCP/API \\
Who can use it & Trained users & Anyone \\
Maintenance & Ongoing engineering & Prompt updates \\
Cost per question & High (amortized) & Near zero (marginal) \\
\bottomrule
\end{tabular}
\end{center}
\vspace{0.3cm}
\centering
The dashboard was a \textit{workaround} for the fact that databases don't speak English.\\Now they do.
\end{frame}

\begin{frame}{How It Works}
AI replaces the entire stack between a business question and a visual answer.
\vspace{0.3cm}
\begin{enumerate}
  \item \textbf{User asks a question} in natural language
  \item \textbf{AI writes and executes a query} (SQL, Python, API call), generates a visualization, and explains the results
  \item \textbf{User asks a follow-up} --- AI refines, drills down, or pivots
\end{enumerate}
\vspace{0.3cm}
No SQL knowledge.  No chart configuration.  No waiting for IT.  The user \textit{describes what they want to see}, and AI produces it.
\end{frame}

\begin{frame}{What This Looks Like in Practice}
\begin{columns}[T]
\begin{column}{0.5\textwidth}
\begin{shadedbox}[title=\textbf{The Conversation}]
\begin{itemize}\small
  \item \textit{``Show me monthly revenue by product line for 2025''}
  \item AI: writes SQL, runs it, produces a grouped bar chart with a summary
  \item \textit{``Now break out the Enterprise segment by region''}
  \item AI: refines the query, adds a region dimension, updates the chart
  \item \textit{``Which region had the biggest drop from Q2 to Q3?''}
  \item AI: calculates quarter-over-quarter changes, highlights the answer
\end{itemize}
\end{shadedbox}
\end{column}
\begin{column}{0.5\textwidth}
\begin{shadedbox}[title=\textbf{What AI Did Behind the Scenes}]
\begin{itemize}\small
  \item Wrote 4 different SQL queries
  \item Produced 3 visualizations (matplotlib/plotly)
  \item Calculated derived metrics and generated a narrative summary
\end{itemize}
\end{shadedbox}
\vspace{0.1cm}
\begin{shadedbox}[title=\textbf{What the User Needed to Know}]
\begin{itemize}\small
  \item What questions to ask
  \item Whether the answers make sense
  \item \alert{Nothing else}
\end{itemize}
\end{shadedbox}
\end{column}
\end{columns}
\end{frame}

% ========================================
% SECTION 3: THE TECHNOLOGY STACK
% ========================================

\section{Connecting AI to Your Data}

\begin{frame}{Three Ways to Connect AI to Data}
\begin{columns}[T]
\begin{column}{0.33\textwidth}
\begin{shadedbox}[title=\textbf{Upload Files}]
\begin{itemize}\small
  \item Drag CSV, Excel, PDF into chat
  \item No setup required
  \item Best for: ad-hoc analysis of exported data
\end{itemize}
\vspace{0.2cm}
\centering\scriptsize
\textit{``Here's our Q3 P\&L.\\Analyze margin trends.''}
\end{shadedbox}
\end{column}
\begin{column}{0.33\textwidth}
\begin{shadedbox}[title=\textbf{MCP Connectors}]
\begin{itemize}\small
  \item Connect to databases, APIs, services directly
  \item AI queries live data
  \item Best for: recurring analysis on production data
\end{itemize}
\vspace{0.2cm}
\centering\scriptsize
\textit{``Query the sales database for YTD revenue by region.''}
\end{shadedbox}
\end{column}
\begin{column}{0.33\textwidth}
\begin{shadedbox}[title=\textbf{API Calls}]
\begin{itemize}\small
  \item Financial data providers (FMP, Alpha Vantage, Bloomberg)
  \item Internal REST APIs
  \item Best for: market data, public filings, external sources
\end{itemize}
\vspace{0.2cm}
\centering\scriptsize
\textit{``Pull Apple's last 4 quarters and chart revenue growth.''}
\end{shadedbox}
\end{column}
\end{columns}
\end{frame}

\begin{frame}{The Database Agent Pattern}
The most powerful dashboard replacement is an \textbf{AI agent connected to your database}.  The agent knows the schema, writes SQL, and presents results visually.
\vspace{0.3cm}
\begin{columns}[T]
\begin{column}{0.5\textwidth}
\begin{shadedbox}[title=\textbf{How to Build It}]
\begin{enumerate}\small
  \item Connect database via MCP (or provide a connection string)
  \item Give AI the schema: table names, columns, relationships
  \item Describe the business context: ``revenue is in the \texttt{orders} table, dates are fiscal quarters''
  \item Start asking questions
\end{enumerate}
\end{shadedbox}
\end{column}
\begin{column}{0.5\textwidth}
\begin{shadedbox}[title=\textbf{What the Agent Can Do}]
\begin{itemize}\small
  \item Write and execute SQL queries
  \item Compute derived metrics (growth rates, ratios, moving averages)
  \item Generate charts and export to Excel or PowerPoint
\end{itemize}
\end{shadedbox}
\end{column}
\end{columns}
\end{frame}

\begin{frame}{Example: From SQL to Insight in Seconds}
\begin{columns}[T]
\begin{column}{0.5\textwidth}
\begin{shadedbox}[title=\textbf{User Prompt}]\small
``Compare revenue and gross margin by business unit for Q3 2025 vs.\ Q3 2024.  Highlight any unit where margin declined more than 200 basis points.  Show it as a table and a chart.''
\end{shadedbox}
\vspace{0.1cm}
\begin{shadedbox}[title=\textbf{AI Generates}]\scriptsize
\texttt{SELECT business\_unit,}\\
\texttt{\ \ SUM(CASE WHEN fiscal\_qtr='Q3-2025'}\\
\texttt{\ \ \ \ THEN revenue END) as rev\_25,}\\
\texttt{\ \ SUM(CASE WHEN fiscal\_qtr='Q3-2024'}\\
\texttt{\ \ \ \ THEN revenue END) as rev\_24, ...}\\
\texttt{FROM financials}\\
\texttt{GROUP BY business\_unit}\\
\texttt{ORDER BY margin\_change ASC}
\end{shadedbox}
\end{column}
\begin{column}{0.5\textwidth}
\begin{shadedbox}[title=\textbf{AI Delivers}]
\begin{itemize}\small
  \item Formatted comparison table and grouped bar chart
  \item Margin waterfall chart with red flags on $>$200bp declines
  \item Written summary: ``Two units saw margin compression \ldots''
\end{itemize}
\end{shadedbox}
\vspace{0.1cm}
\begin{shadedbox}[title=\textbf{Follow-up}]\small
\textit{``Drill into the Hardware unit --- is the margin decline driven by pricing or COGS?''}
\vspace{0.2cm}

AI writes a new query, decomposes the variance, and updates the analysis.
\end{shadedbox}
\end{column}
\end{columns}
\end{frame}

% ========================================
% SECTION 4: VISUALIZATION ON DEMAND
% ========================================

\section{AI-Generated Visualizations}

\begin{frame}{Describe the Chart, Not the Config}
Traditional BI tools require you to configure chart type, axes, filters, colors, and labels manually.  With AI, you \alert{describe what you want to see} and the visualization is generated.
\vspace{0.3cm}
\begin{columns}[T]
\begin{column}{0.5\textwidth}
\begin{shadedbox}[title=\textbf{Natural Language Prompts}]
\begin{itemize}\small
  \item ``Bar chart of revenue by quarter''
  \item ``Scatter plot of P/E vs.\ growth rate for our peer group''
  \item ``Waterfall chart from budget to actual operating income''
\end{itemize}
\end{shadedbox}
\end{column}
\begin{column}{0.5\textwidth}
\begin{shadedbox}[title=\textbf{AI Handles the Details}]
\begin{itemize}\small
  \item Chooses appropriate chart libraries (matplotlib, plotly, seaborn)
  \item Sets axis labels, titles, legends, and color schemes
  \item Iterates if you say ``make it cleaner'' or ``add a trend line''
\end{itemize}
\end{shadedbox}
\end{column}
\end{columns}
\end{frame}

\begin{frame}{Interactive Artifacts}
Claude's \textbf{artifacts} are interactive, publishable applications generated from a single prompt.  They go far beyond static charts.
\vspace{0.3cm}
\begin{columns}[T]
\begin{column}{0.5\textwidth}
\begin{shadedbox}[title=\textbf{What Artifacts Can Be}]
\begin{itemize}\small
  \item Interactive dashboards with filters and dropdowns
  \item Calculators (loan amortization, DCF, option pricing)
  \item Scenario analysis tools with sliders
\end{itemize}
\end{shadedbox}
\end{column}
\begin{column}{0.5\textwidth}
\begin{shadedbox}[title=\textbf{Key Advantage}]
\begin{itemize}\small
  \item Generated in seconds, publishable via a shareable link
  \item Fully interactive (React under the hood)
  \item \alert{Disposable}: Generate a new one when needs change instead of maintaining the old one
\end{itemize}
\end{shadedbox}
\end{column}
\end{columns}
\vspace{0.3cm}
An artifact is a \textit{single-serving dashboard} --- built for the question at hand, not for every possible question.
\end{frame}

\begin{frame}[shrink=3]{Artifacts Across Platforms}
Every major AI chatbot now generates interactive web applications from a prompt.  The feature goes by different names, but the concept is the same.
\vspace{0.2cm}
\begin{center}
\small
\begin{tabular}{@{} p{2.5cm} p{3.5cm} p{3.5cm} p{3.5cm} @{}}
\toprule
& \textbf{Claude Artifacts} & \textbf{ChatGPT Canvas} & \textbf{Gemini Canvas} \\
\midrule
Technology & React (HTML/CSS/JS) & React (HTML/CSS/JS) & HTML/CSS/JS \\
Side panel & Yes & Yes & Yes \\
Publish / share & One-click shareable link & Shareable link & Export or copy \\
Viewer needs account & No & No & Yes (Google) \\
Iterative editing & Chat to refine & Inline edits + chat & Chat to refine \\
Python execution & Separate (code tool) & Built into canvas & Via Colab link \\
Best for & Interactive apps, calculators, dashboards & Collaborative writing + code editing & Google ecosystem, Sheets/Slides integration \\
\bottomrule
\end{tabular}
\end{center}
\vspace{0.2cm}
All three can produce the same kinds of interactive finance tools --- DCF calculators, portfolio dashboards, scenario analyzers.  Choose based on your ecosystem and subscription.
\end{frame}

% ========================================
% SECTION 5: FINANCE USE CASES
% ========================================

\section{Finance Applications}

\begin{frame}[shrink=5]{Replacing Finance Dashboards}
\begin{columns}[T]
\begin{column}{0.5\textwidth}
\textbf{FP\&A}
\begin{itemize}\small
  \item \textit{``Show budget vs.\ actual for Q3, decompose the variance into volume, price, and cost drivers''}
  \item \textit{``Create a rolling 12-month revenue forecast chart with confidence bands''}
  \item \textit{``Which cost centers are trending above budget? Rank by dollar impact.''}
\end{itemize}
\vspace{0.2cm}
\textbf{Treasury \& Risk}
\begin{itemize}\small
  \item \textit{``Plot our cash position over the next 90 days using the latest AR/AP forecasts''}
  \item \textit{``Show VaR by desk for the last 30 days and flag any breaches''}
  \item \textit{``What's our FX exposure by currency? Show hedged vs.\ unhedged.''}
\end{itemize}
\end{column}
\begin{column}{0.5\textwidth}
\textbf{Portfolio Management}
\begin{itemize}\small
  \item \textit{``Chart our sector allocation vs.\ benchmark and show the active weights''}
  \item \textit{``Performance attribution for the quarter --- how much came from allocation vs.\ selection?''}
  \item \textit{``Show the return distribution of our book with a histogram''}
\end{itemize}
\vspace{0.2cm}
\textbf{Executive Reporting}
\begin{itemize}\small
  \item \textit{``Build a one-page executive summary with KPIs, trends, and commentary''}
  \item \textit{``Create a board deck from this quarter's financials --- 5 slides max''}
  \item \textit{``What are the top 3 things the CFO should know from this data?''}
\end{itemize}
\end{column}
\end{columns}
\end{frame}

\begin{frame}{The Weekly Operations Review --- Before and After}
\begin{columns}[T]
\begin{column}{0.5\textwidth}
\begin{shadedbox}[title=\textbf{Before: The Dashboard Era}]
\begin{enumerate}\small
  \item Data team pulls exports (2 hours)
  \item Analyst builds slides in Excel/PPT (4 hours)
  \item Manager reviews and requests changes (1 hour)
  \item Analyst revises (2 hours)
  \item VP asks a question not on the slide
  \item ``We'll get back to you next week''
\end{enumerate}
\vspace{0.2cm}
\centering\small
\textbf{Total: 9+ hours per week}
\end{shadedbox}
\end{column}
\begin{column}{0.5\textwidth}
\begin{shadedbox}[title=\textbf{After: Natural Language AI}]
\begin{enumerate}\small
  \item AI agent connected to database
  \item Analyst: \textit{``Generate the weekly ops review''}
  \item AI: pulls data, creates charts, writes narrative (3 minutes)
  \item Manager reviews, asks for changes in chat
  \item VP asks a question not on the slide
  \item \alert{AI answers it in 10 seconds}
\end{enumerate}
\vspace{0.2cm}
\centering\small
\textbf{Total: 15 minutes + live Q\&A}
\end{shadedbox}
\end{column}
\end{columns}
\vspace{0.2cm}
This is what HPE's ``Alfred'' did.  The 90\% reduction in prep time wasn't efficiency --- it was \alert{eliminating the translation layer} between data and decisions.
\end{frame}


% ========================================
% SECTION 6: PRACTICAL CONSIDERATIONS
% ========================================

\section{Making It Work}

\begin{frame}{When Dashboards Still Win}
Natural language AI doesn't replace \textit{all} dashboards.  Some use cases still favor traditional BI tools.
\vspace{0.3cm}
\begin{columns}[T]
\begin{column}{0.5\textwidth}
\begin{shadedbox}[title=\textbf{Keep the Dashboard}]
\begin{itemize}\small
  \item \textbf{Real-time monitoring}: Trading floors, server health, live KPIs on a wall screen
  \item \textbf{Regulatory reporting}: Fixed formats required by regulators (SEC, Basel)
  \item \textbf{Embedded analytics}: Dashboards inside products for customers
\end{itemize}
\end{shadedbox}
\end{column}
\begin{column}{0.5\textwidth}
\begin{shadedbox}[title=\textbf{Replace with AI}]
\begin{itemize}\small
  \item \textbf{Ad-hoc analysis}: One-off questions that arise in meetings
  \item \textbf{Executive reporting}: Weekly/monthly decks assembled from data
  \item \textbf{Variance analysis}: ``Why did we miss?'' questions
\end{itemize}
\end{shadedbox}
\end{column}
\end{columns}
\vspace{0.2cm}
The rule of thumb: if someone \textit{asks the same question every day}, build a dashboard.\\If they \textit{ask different questions every day}, give them an AI agent.
\end{frame}

\begin{frame}{Ensuring Accuracy}
The biggest objection: ``How do I know the AI queried the right data?''  This is solvable.
\vspace{0.3cm}
\begin{itemize}
  \item \textbf{Show the code}: AI can display the SQL or Python it wrote --- analysts can review it
  \item \textbf{Schema descriptions}: Tell AI exactly what each table and column means (via system prompts or instructions)
  \item \textbf{Read-only access}: Connect AI to a read replica --- it can query but never modify
\end{itemize}
\vspace{0.3cm}
The same analysts who currently build dashboards become \alert{AI reviewers} --- checking AI-generated queries instead of writing them from scratch.
\end{frame}

\begin{frame}{Getting Started: A Practical Roadmap}
\begin{enumerate}
  \item \textbf{Start with uploaded files}: Export a CSV or Excel from your existing system.  Ask AI to analyze it in chat.  No integration needed.
  \item \textbf{Connect a read-only database}: Use MCP to connect AI to a read replica or data warehouse.  Start with one department's data.
  \item \textbf{Save your best prompts}: Write down the instructions that produce good results (weekly review, variance analysis, portfolio report).  Reuse them each time.
\end{enumerate}
\vspace{0.3cm}
Step 1 takes \textbf{5 minutes}.  Steps 1--3 can be done in \textbf{a single afternoon}.\\You don't need to replace your BI platform --- you just stop building new dashboards.
\end{frame}

\begin{frame}{Summary}
\begin{columns}[T]
\begin{column}{0.33\textwidth}
\begin{shadedbox}[title=\textbf{The Problem}]
\begin{itemize}\small
  \item Dashboards answer fixed questions
  \item Building them is slow and expensive
  \item Follow-ups require a new build cycle
\end{itemize}
\end{shadedbox}
\end{column}
\begin{column}{0.33\textwidth}
\begin{shadedbox}[title=\textbf{The Solution}]
\begin{itemize}\small
  \item Ask questions in plain English
  \item AI writes queries and charts
  \item Follow-ups are instant
\end{itemize}
\end{shadedbox}
\end{column}
\begin{column}{0.33\textwidth}
\begin{shadedbox}[title=\textbf{How to Start}]
\begin{itemize}\small
  \item Upload a file and ask questions
  \item Connect a database via MCP
  \item Save and reuse your best prompts
\end{itemize}
\end{shadedbox}
\end{column}
\end{columns}
\vspace{0.5cm}
\begin{center}
\textbf{The best dashboard is no dashboard} --- it's a conversation with your data.
\end{center}
\end{frame}

\end{document}
