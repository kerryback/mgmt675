\documentclass[aspectratio=169]{beamer}
\usetheme{metropolis}
\usepackage{appendixnumberbeamer}
\usepackage{booktabs}
\usepackage{textcomp}

\input{mgmt675-style}

\subtitle{MGMT 675: Generative AI for Finance}
\title{Google Gemini in Colab}
\author{Kerry Back}

\date{}
\titlegraphic{\includegraphics[width=4cm]{images/colab_slide1_img4.png}}

\begin{document}

\maketitle

\begin{frame}{A Different Approach}
\begin{shadedbox}\centerline{Code Environment + Chatbot}\end{shadedbox}

\begin{baritemize}
  \item ChatGPT and Claude: Chatbots with code execution added
  \item Google Colab: Code execution environment with chatbot added
  \item Colab started as Jupyter notebooks in the cloud (2017)
  \item Gemini was integrated into Colab later (2024)
  \item Philosophy: Write and run code first, use AI to assist
\end{baritemize}
\end{frame}

\begin{frame}{What is Google Colab?}
\begin{baritemize}
  \item A free tool from Google for running code in your browser
  \item No software installation required
  \item Works on any computer with internet access
  \item All your work saves automatically to Google Drive
\end{baritemize}
\end{frame}

\begin{frame}{What You Need}
\begin{baritemize}
  \item Just two things:
  \item A Google account (Gmail works)
  \item A web browser (Chrome recommended)
  \item That's it!
\end{baritemize}
\end{frame}

\begin{frame}{Accessing Colab: From Google Drive}
Click New $\rightarrow$ More $\rightarrow$ Google Colaboratory
\begin{center}
\includegraphics[width=0.8\textwidth]{images/colab_slide4_img2.png}
\end{center}
\end{frame}

\begin{frame}{Accessing Colab: Direct}
\begin{center}
\begin{shadedbox}
\centerline{\Large\href{https://colab.research.google.com}{colab.research.google.com}}
\end{shadedbox}
\end{center}
\end{frame}

\begin{frame}{The Open Notebook Dialog}
\begin{center}
\includegraphics[width=0.9\textwidth]{images/colab_slide6_img2.png}
\end{center}
\end{frame}

\begin{frame}{Opening Notebooks}
\begin{baritemize}
  \item Examples: Google's tutorial notebooks
  \item Recent: Your recently opened notebooks
  \item Google Drive: Notebooks saved in your Drive
  \item GitHub: Open notebooks from GitHub repos
  \item Upload: Upload a .ipynb file
  \item Click + New notebook to start a fresh notebook
\end{baritemize}
\end{frame}

\begin{frame}{The Colab Interface: Notebook + Gemini}
\begin{center}
\includegraphics[width=0.9\textwidth]{images/colab_slide8_img2.png}
\end{center}
\end{frame}

\begin{frame}{How Notebooks Work}
\begin{baritemize}
\item Three elements: notebook, notebook interface (Colab or other), and Python runtime environment
\item A notebook (.ipynb file) is just a text file
\item The interface renders the file to create what you see and handles communication with the runtime environment
\begin{enumerate}
\item When you run a cell, the code is transmitted to a runtime environment (called a kernel).
\item The runtime processes and executes your code.
\item Results flow back to the notebook interface.
\item The interface renders outputs, visualizations, and any error messages.
\end{enumerate}
\end{baritemize}

\end{frame}

\begin{frame}{Navigating a Notebook}
\begin{baritemize}
  \item + Code: Add a new code cell
  \item + Text: Add a text/markdown cell
  \item Connect: Connect to Google's servers
  \item Files (folder icon): View and upload files
\end{baritemize}
\end{frame}

\begin{frame}{What is a Cell?}
\begin{baritemize}
  \item A cell is a box where you write code or text.
  \item Two types:
  \item Code cells: Run Python code
  \item Text cells: Write notes and explanations
  \item You can have as many cells as you need.
\end{baritemize}
\end{frame}

\begin{frame}{Your First Code: Simple Math}
Type 5*3 and press Shift + Enter $\rightarrow$ Result: 15
\begin{center}
\includegraphics[width=0.7\textwidth]{images/colab_slide12_img2.png}
\end{center}
\end{frame}

\begin{frame}{Your First Code: Hello World}
Type print('hello world') and press Shift + Enter
\begin{center}
\includegraphics[width=0.7\textwidth]{images/colab_slide13_img2.png}
\end{center}
\end{frame}

\begin{frame}{Running Code: Three Ways}
\begin{baritemize}
  \item Click the play button ($>$) on the left of the cell
  \item Press Shift + Enter (runs and moves to next cell)
  \item Press Ctrl + Enter (runs and stays in cell)
  \item Tip: Shift + Enter is the most common method
\end{baritemize}
\end{frame}

\begin{frame}{Understanding the Play Button}
\begin{baritemize}
  \item Before running:
  \item Circle with play icon ($>$) - Cell is ready
  \item While running:
  \item Spinning circle - Code is executing
  \item After running:
  \item Checkmark - Output appears below
\end{baritemize}
\end{frame}

\begin{frame}{Cell Numbers}
\begin{baritemize}
  \item Notice the [1] or [2] next to cells?
  \item Shows the order cells were run
  \item Empty [ ] means not yet run
  \item [*] means currently running
  \item Important: Can run cells in any order but top to bottom avoids confusion.
\end{baritemize}
\end{frame}

\begin{frame}{Adding New Cells}
\begin{baritemize}
  \item From the toolbar:
  \item Click + Code for a code cell
  \item Click + Text for a text cell
  \item Using keyboard:
  \item Ctrl + M, B $\rightarrow$ Add cell below
  \item Ctrl + M, A $\rightarrow$ Add cell above
\end{baritemize}
\end{frame}

\begin{frame}{Deleting and Moving Cells}
\begin{baritemize}
  \item To delete a cell:
  \item Click the trash icon in the cell toolbar
  \item Or: Ctrl + M, D
  \item To move a cell:
  \item Click the up/down arrows in the cell toolbar
  \item Or drag and drop the cell
\end{baritemize}
\end{frame}

\begin{frame}{Meet Gemini: Your AI Assistant}
Gemini is built into Colab to help you write code
\begin{center}
\includegraphics[width=0.8\textwidth]{images/colab_slide19_img2.png}
\end{center}
\end{frame}

\begin{frame}{What Gemini Can Do}
\begin{baritemize}
  \item Generate code from plain English descriptions
  \item Explain what existing code does
  \item Fix errors in your code
  \item Suggest improvements
  \item Answer Python questions
\end{baritemize}
\end{frame}

\begin{frame}{Runtime: What Powers Your Code}
\begin{baritemize}
  \item When you click Connect, Colab gives you a virtual computer:
  \item CPU (standard processing)
  \item RAM (memory)
  \item Disk space
  \item And optionally: GPU or TPU for machine learning
\end{baritemize}
\end{frame}

\begin{frame}{Restarting the Runtime}
If your code isn't working as expected:

\begin{shadedbox}
\centerline{Runtime $\rightarrow$ Restart runtime}
\end{shadedbox}

This clears all variables and starts fresh.

Note: You'll need to re-run your cells after restarting

\end{frame}

\begin{frame}{Session Limits}
\begin{baritemize}
   \item Sessions disconnect after $\sim$90 minutes idle
  \item Maximum $\sim$12 hours continuous use
  \item Limited GPU/TPU hours per week
\end{baritemize}
\end{frame}


\begin{frame}{Get Started}
  \begin{barenumerate}
  \item Ask Gemini to get stock price data from Yahoo Finance and compute daily returns.
  \item Ask Gemini to generate a boxplot of the daily returns.
  \item Ask Gemini how you can save the boxplot.
  \item Ask Gemini how you can save the return data.
  \item Ask Gemini how yu can save the notebook
  \end{barenumerate}
\end{frame}

\begin{frame}{Exercise: Computing Returns}
\begin{baritemize}
  \item Download \href{https://kerryback.com/mgmt675/exercises/returns.xlsx}{returns.xlsx} to your Google Drive
  \item Ask Gemini to mount your Google Drive
  \item Ask Gemini to compute daily returns including dividends
  \item Ask Gemini to calculate annualized mean return and volatility
\end{baritemize}
\end{frame}

\begin{frame}{Exercise: Estimating Betas}
\begin{baritemize}
  \item Download \href{https://kerryback.com/mgmt675/exercises/betas.xlsx}{betas.xlsx} to your Google Drive
  \item Ask Gemini to mount your Google Drive
  \item Ask Gemini to estimate betas for each stock using regression
  \item Ask Gemini to interpret the results
\end{baritemize}
\end{frame}

\begin{frame}{Exercise: Mean-Variance Analysis}
\begin{baritemize}
  \item Download \href{https://kerryback.com/mgmt675/exercises/meanvariance.xlsx}{meanvariance.xlsx} to your Google Drive
  \item Ask Gemini to mount your Google Drive
  \item Ask Gemini to find the tangency portfolio
  \item Ask Gemini to plot the efficient frontier
\end{baritemize}
\end{frame}

\end{document}
