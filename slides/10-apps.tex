\documentclass[aspectratio=169]{beamer}
\usetheme{metropolis}
\usepackage{appendixnumberbeamer}
\usepackage{booktabs}

\input{mgmt675-style}

\subtitle{MGMT 675: Generative AI for Finance}
\title{Creating Web Apps}
\author{Kerry Back}

\date{}

\begin{document}

\maketitle

\begin{frame}{What is Streamlit?}
\begin{columns}[T]
\begin{column}{0.5\textwidth}
\begin{shadedbox}[title=\textbf{Streamlit Overview}]
\begin{itemize}\small
  \item Python library for web apps
  \item No HTML, CSS, or JavaScript needed
  \item Designed for data apps and dashboards
  \item Interactive widgets built-in
  \item Hot reload during development
\end{itemize}
\end{shadedbox}
\end{column}
\begin{column}{0.5\textwidth}
\begin{shadedbox}[title=\textbf{Perfect For}]
\begin{itemize}\small
  \item Data visualization dashboards
  \item Machine learning demos
  \item Financial analysis tools
  \item Interactive reports
  \item Chatbot interfaces
\end{itemize}
\end{shadedbox}
\end{column}
\end{columns}
\end{frame}

\begin{frame}{From Idea to Public App}
\begin{shadedbox}
The complete workflow to create and deploy a Streamlit app---all handled by Claude Code with simple natural language requests.
\end{shadedbox}
\vspace{0.3cm}
\begin{barenumerate}
  \item Create the Streamlit app
  \item Initialize a Git repository
  \item Create a GitHub repository
  \item Commit and push the code
  \item Deploy to Koyeb with auto-deploy
  \item Get your public URL
\end{barenumerate}
\vspace{0.3cm}
\begin{center}
\alert{Ask Claude to do each step---no commands to memorize}
\end{center}
\end{frame}

\begin{frame}{Step 1: Create the Streamlit App}
\begin{columns}[T]
\begin{column}{0.5\textwidth}
\begin{shadedbox}[title=\textbf{What You Need}]
\begin{itemize}\small
  \item Main app file (\texttt{app.py})
  \item Dependencies file (\texttt{requirements.txt})
  \item Any data files or assets
\end{itemize}
\end{shadedbox}
\vspace{0.3cm}
\begin{shadedbox}[title=\textbf{Ask Claude}]
``Create a Streamlit app that [describe your app]''
\end{shadedbox}
\end{column}
\begin{column}{0.5\textwidth}
\begin{baritemize}\small
  \item Claude writes the Python code
  \item Creates requirements.txt automatically
  \item Can test locally first
  \item Iterates based on your feedback
\end{baritemize}
\end{column}
\end{columns}
\end{frame}

\begin{frame}{Step 2: Initialize Git Repository}
\begin{columns}[T]
\begin{column}{0.5\textwidth}
\begin{shadedbox}[title=\textbf{What Happens}]
\begin{itemize}\small
  \item Creates \texttt{.git} folder in your project
  \item Enables version control
  \item Tracks all file changes
  \item Creates \texttt{.gitignore} for Python
\end{itemize}
\end{shadedbox}
\end{column}
\begin{column}{0.5\textwidth}
\begin{shadedbox}[title=\textbf{Ask Claude}]
``Initialize this folder as a git repository''
\end{shadedbox}
\vspace{0.5cm}
\begin{baritemize}\small
  \item Claude runs \texttt{git init}
  \item Creates appropriate .gitignore
  \item Ready for commits
\end{baritemize}
\end{column}
\end{columns}
\end{frame}

\begin{frame}{Step 3: Create GitHub Repository}
\begin{columns}[T]
\begin{column}{0.5\textwidth}
\begin{shadedbox}[title=\textbf{What Happens}]
\begin{itemize}\small
  \item Creates new repo on GitHub.com
  \item Uses GitHub CLI (\texttt{gh})
  \item Links local repo to remote
  \item Can be public or private
\end{itemize}
\end{shadedbox}
\end{column}
\begin{column}{0.5\textwidth}
\begin{shadedbox}[title=\textbf{Ask Claude}]
``Create a GitHub repository for this project called [name]''
\end{shadedbox}
\vspace{0.5cm}
\begin{baritemize}\small
  \item Claude uses \texttt{gh repo create}
  \item Sets up remote origin
  \item Ready to push code
\end{baritemize}
\end{column}
\end{columns}
\end{frame}

\begin{frame}{Step 4: Commit and Push}
\begin{columns}[T]
\begin{column}{0.5\textwidth}
\begin{shadedbox}[title=\textbf{What Happens}]
\begin{itemize}\small
  \item Stages all project files
  \item Creates initial commit
  \item Pushes to GitHub
  \item Code now on GitHub.com
\end{itemize}
\end{shadedbox}
\end{column}
\begin{column}{0.5\textwidth}
\begin{shadedbox}[title=\textbf{Ask Claude}]
``Commit all files and push to GitHub''
\end{shadedbox}
\vspace{0.5cm}
\begin{baritemize}\small
  \item Claude runs git add, commit, push
  \item Writes meaningful commit message
  \item Verifies push succeeded
\end{baritemize}
\end{column}
\end{columns}
\end{frame}

\begin{frame}{Step 5: Deploy to Koyeb}
\begin{columns}[T]
\begin{column}{0.5\textwidth}
\begin{shadedbox}[title=\textbf{What is Koyeb?}]
\begin{itemize}\small
  \item Cloud platform for deploying apps
  \item Free tier available
  \item Auto-deploy from GitHub
  \item Handles SSL, scaling, etc.
\end{itemize}
\end{shadedbox}
\end{column}
\begin{column}{0.5\textwidth}
\begin{shadedbox}[title=\textbf{Ask Claude}]
``Create a Koyeb service for this app linked to the GitHub repo with auto-deploy''
\end{shadedbox}
\vspace{0.3cm}
\begin{baritemize}\small
  \item Claude uses Koyeb CLI
  \item Links to GitHub repo
  \item Enables auto-deploy on push
\end{baritemize}
\end{column}
\end{columns}
\end{frame}

\begin{frame}{Step 6: Get Your Public URL}
\begin{columns}[T]
\begin{column}{0.5\textwidth}
\begin{shadedbox}[title=\textbf{What Happens}]
\begin{itemize}\small
  \item Koyeb provides public URL
  \item Format: \texttt{app-name.koyeb.app}
  \item SSL included automatically
  \item App accessible worldwide
\end{itemize}
\end{shadedbox}
\end{column}
\begin{column}{0.5\textwidth}
\begin{shadedbox}[title=\textbf{Ask Claude}]
``What is the public URL for my Koyeb service?''
\end{shadedbox}
\vspace{0.5cm}
\begin{baritemize}\small
  \item Claude queries Koyeb CLI
  \item Returns your public URL
  \item You can share with anyone
\end{baritemize}
\end{column}
\end{columns}
\end{frame}

\begin{frame}{Auto-Deploy: The Magic}
\begin{shadedbox}
Once set up, every \texttt{git push} automatically updates your live app.
\end{shadedbox}
\vspace{0.5cm}
\begin{columns}[T]
\begin{column}{0.5\textwidth}
\begin{shadedbox}[title=\textbf{Update Workflow}]
\begin{enumerate}\small
  \item Make changes locally
  \item Ask Claude to commit and push
  \item Koyeb detects the push
  \item Rebuilds and redeploys
  \item Live app updated automatically
\end{enumerate}
\end{shadedbox}
\end{column}
\begin{column}{0.5\textwidth}
\begin{baritemize}\small
  \item No manual deployment steps
  \item Changes live in minutes
  \item Rollback if needed
  \item View deployment logs
\end{baritemize}
\end{column}
\end{columns}
\end{frame}

\begin{frame}{Exercise: Mean-Variance App}
\begin{shadedbox}
Build and deploy a Streamlit app for mean-variance portfolio optimization.
\end{shadedbox}
\vspace{0.3cm}
\begin{columns}[T]
\begin{column}{0.5\textwidth}
\begin{shadedbox}[title=\textbf{User Inputs}]
\begin{itemize}\small
  \item Risk-free rate
  \item Number of risky assets
  \item Expected returns (means)
  \item Standard deviations
  \item Correlation matrix
\end{itemize}
\end{shadedbox}
\end{column}
\begin{column}{0.5\textwidth}
\begin{shadedbox}[title=\textbf{App Outputs}]
\begin{itemize}\small
  \item Tangency portfolio weights
  \item Downloadable image of:
  \begin{itemize}\footnotesize
    \item Frontier of risky assets
    \item Capital allocation line
  \end{itemize}
\end{itemize}
\end{shadedbox}
\end{column}
\end{columns}
\end{frame}

\begin{frame}{Summary: One Conversation, Live App}
\begin{center}
\begin{tabular}{ll}
\toprule
\textbf{Step} & \textbf{Ask Claude} \\
\midrule
Create app & ``Create a Streamlit app that...'' \\
Init git & ``Initialize as a git repository'' \\
Create GitHub repo & ``Create a GitHub repo called...'' \\
Commit \& push & ``Commit and push to GitHub'' \\
Deploy & ``Deploy to Koyeb with auto-deploy'' \\
Get URL & ``What's my public URL?'' \\
\bottomrule
\end{tabular}
\end{center}
\vspace{0.5cm}
\begin{shadedbox}
\centering
\alert{No commands to memorize.} Just describe what you want.
\end{shadedbox}
\end{frame}

\end{document}
