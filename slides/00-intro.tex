\documentclass[aspectratio=169]{beamer}
\usetheme{metropolis}
\usepackage{appendixnumberbeamer}
\usepackage{booktabs, hyperref}

\input{mgmt675-style}

\subtitle{MGMT 675: Generative AI for Finance}
\title{Course Introduction}
\author{Kerry Back}

\date{}
\titlegraphic{\includegraphics[width=4cm]{images/slide1_img1.png}}

\begin{document}

\maketitle

% ========================================
% SECTION 1: WHY AI MATTERS FOR FINANCE
% ========================================

\begin{frame}{From Makers to Checkers}
\begin{shadedbox}\centerline{Derek Waldron, JP Morgan Chief Analytics Officer}\end{shadedbox}

What we're working towards is that every employee will have their own personalized AI assistant; every process is powered by AI agents, and every client experience has an AI concierge.

You'll still have people at the top who are managing and have relationships with clients, but many, many of the processes underneath are now being done by AI systems.

Workers would shift from being creators of reports or software updates, or 'makers' ... to 'checkers' or managers of AI agents doing that work.
\end{frame}

\begin{frame}{The New Skills Landscape}
\begin{shadedbox}\centerline{Brookings Institute, 2025}\end{shadedbox}

As AI models begin to handle underwriting, compliance, and asset allocation, the traditional architecture of financial work is undergoing a fundamental shift.

As job descriptions evolve, so does the definition of financial talent. Excel is no longer a differentiator. Python is fast becoming the new Excel.

But technical skills alone will not cut it. The most in demand profiles today are those that speak both AI and finance.
\end{frame}

\begin{frame}{Why AI is Useful for Finance}
\begin{baritemize}
  \item AI can look up and answer questions about company practices, past company reports, communications, etc.
  \item AI can search the web, pull data from databases, or call APIs
  \item AI can analyze documents, contracts, or financial statements
  \item AI can write code for financial analysis and building things (vibe coding)
  \item AI can generate Excel workbooks
  \item AI can draft reports (Word docs) and presentation decks (PowerPoint)
   \item AI can automate repetitive tasks and workflows
  \item AI can communicate directly with clients (with strong controls)
\end{baritemize}
\end{frame}


\begin{frame}{Why AI with Code Execution is Especially Useful}
\begin{baritemize}
   \item Attaching a code execution tool enables:
  \begin{itemize}
    \item Data analysis with real calculations
    \item Visualizations and charts
    \item File processing (Excel, CSV, PDF)
     \end{itemize}
  \item Transforms chatbots into computational tools
\end{baritemize}
\end{frame}


% ========================================
% SECTION 2: COURSE OUTLINE
% ========================================
\section{Course Outline}

\begin{frame}{Learning Objectives}
\begin{enumerate}
  \item How to use AI with code-execution tools to perform financial analysis and generate spreadsheets, reports, and presentations
  \item How to use AI with code-execution tools to search, organize, and filter data and text
  \item How to collaborate with AI in planning and evaluating work
  \item How custom chatbots and AI agents work and are used in finance
  \item How to use AI with code-execution tools as general agents
\end{enumerate}
\end{frame}

\begin{frame}{Course Topics}
\begin{columns}[T]
\begin{column}{0.45\textwidth}
\begin{enumerate}
  \item AI and coding
  \item Claude for Excel
  \item Searching, organizing, and filtering data and text
  \item Claude artifacts
  \item Mean-variance analysis
\end{enumerate}
\end{column}
\begin{column}{0.45\textwidth}
\begin{enumerate}
  \setcounter{enumi}{6}
  \item DCF analysis
  \item Claude skills
  \item Claude Code and local execution
  \item Apps, custom chatbots, and agents
  \item Databases
\end{enumerate}
\end{column}
\end{columns}
\end{frame}

\end{document}
