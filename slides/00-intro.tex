\documentclass[aspectratio=169]{beamer}
\usetheme{metropolis}
\usepackage{appendixnumberbeamer}
\usepackage{booktabs, hyperref}

\input{mgmt675-style}

\subtitle{MGMT 675: Generative AI for Finance}
\title{Course Introduction}
\author{Kerry Back}

\date{}
\titlegraphic{\includegraphics[width=4cm]{images/slide1_img1.png}}

\begin{document}

\maketitle

% ========================================
% SECTION 1: WHY AI MATTERS FOR FINANCE
% ========================================

\begin{frame}{From Makers to Checkers}
\begin{shadedbox}\centerline{Derek Waldron, JP Morgan Chief Analytics Officer}\end{shadedbox}

What we're working towards is that every employee will have their own personalized AI assistant; every process is powered by AI agents, and every client experience has an AI concierge.

You'll still have people at the top who are managing and have relationships with clients, but many, many of the processes underneath are now being done by AI systems.

Workers would shift from being creators of reports or software updates, or 'makers' ... to 'checkers' or managers of AI agents doing that work.
\end{frame}

\begin{frame}{The New Skills Landscape}
\begin{shadedbox}\centerline{Brookings Institute, 2025}\end{shadedbox}

As AI models begin to handle underwriting, compliance, and asset allocation, the traditional architecture of financial work is undergoing a fundamental shift.

As job descriptions evolve, so does the definition of financial talent. Excel is no longer a differentiator. Python is fast becoming the new Excel.

But technical skills alone will not cut it. The most in demand profiles today are those that speak both AI and finance.
\end{frame}

\begin{frame}{Something Big is Happening}
\begin{shadedbox}\centerline{Matt Shumer, CEO of HyperWrite AI, Feb.\ 2025}\end{shadedbox}

\begin{baritemize}
  \item AI capability is doubling every few months --- 2022: couldn't do arithmetic; 2023: passed bar exam; 2024: wrote functional software; 2025--26: autonomous expert-level work
  \item AI now contributes to building its own successors (self-improving loop)
  \item Dario Amodei (CEO, Anthropic): 50\% of entry-level white-collar jobs eliminated in 1--5 years
  \item Unlike prior automation, AI improves across \emph{all} knowledge domains simultaneously
\end{baritemize}
\vfill
{\small \href{https://shumer.dev/something-big-is-happening}{shumer.dev/something-big-is-happening} \\
\href{https://www.linkedin.com/pulse/something-big-happening-matt-shumer-so5he/}{LinkedIn version}}
\end{frame}

\begin{frame}{Something Big is Happening (cont.)}
\begin{shadedbox}\centerline{Matt Shumer}\end{shadedbox}

The person who walks into a meeting and says ``I used AI to do this analysis in an hour instead of three days'' is going to be the most valuable person in the room. Not eventually. Right now.

\bigskip

Here's a simple commitment that will put you ahead of almost everyone: spend one hour a day experimenting with AI. Not passively reading about it. Using it. Every day, try to get it to do something new \ldots\ something you haven't tried before, something you're not sure it can handle. Try a new tool. Give it a harder problem. One hour a day, every day. If you do this for the next six months, you will understand what's coming better than 99\% of the people around you. That's not an exaggeration. Almost nobody is doing this right now. The bar is on the floor.
\end{frame}

\begin{frame}{Why AI is Useful for Finance}
\begin{baritemize}
  \item AI can look up and answer questions about company practices, past company reports, communications, etc.
  \item AI can search the web, pull data from databases, or call APIs
  \item AI can analyze documents, contracts, or financial statements
  \item AI can write code for financial analysis and building things (vibe coding)
  \item AI can generate Excel workbooks
  \item AI can draft reports (Word docs) and presentation decks (PowerPoint)
   \item AI can automate repetitive tasks and workflows
  \item AI can communicate directly with clients (with strong controls)
\end{baritemize}
\end{frame}


\begin{frame}{Why AI with Code Execution is Especially Useful}
\begin{baritemize}
   \item Attaching a code execution tool enables:
  \begin{itemize}
    \item Data analysis with real calculations
    \item Visualizations and charts
    \item File processing (Excel, CSV, PDF)
     \end{itemize}
  \item Transforms chatbots into computational tools
\end{baritemize}
\end{frame}


% ========================================
% SECTION 2: COURSE OUTLINE
% ========================================
\section{Course Outline}

\begin{frame}{Learning Objectives}
\begin{barenumerate}
\item How to use AI with code-execution tools to perform financial analysis, handle data and text, and generate visualizations, spreadsheets, reports, and presentations
\item How to use AI with other tools -- terminal execution, browser control, database connections -- for finance applications 
\item How to collaborate with AI in planning, executing, and evaluating financial analysis
\item How the different AI code execution environments work - cloud sandboxed, virtual machines, local
\item How to create and deploy task-specific prompts that can be used repeatedly
\item How to reduce the hallucination rate of AI through retrieval-augmented generation, fine tuning, or building specialized models
\item How AI can be used to classify the sentiment of news or social media for trading
\end{barenumerate}
\end{frame}

\begin{frame}{Course Topics}
\begin{columns}[T]
\begin{column}{0.45\textwidth}
\begin{enumerate}
  \item AI that writes and executes code
  \item AI coding for mean-variance analysis
  \item AI-written code in Jupyter notebooks
  \item Connecting tools to AI
  \item Connecting a virtual machine to AI
  \item Connecting your computer to AI
  \item Using AI inside Excel
\end{enumerate}
\end{column}
\begin{column}{0.45\textwidth}
\begin{enumerate}
  \setcounter{enumi}{7}
  \item Specialized prompt automation
  \item Using AI inside an IDE
  \item AI for DCF valuation
  \item Retrieval augmented generation
  \item Fine-tuning and small language models
  \item Building an AI agent
  \item Trading on news with AI
\end{enumerate}
\end{column}
\end{columns}
\end{frame}

\begin{frame}{Grading}
  \begin{baritemize}
\item Six group assignments  (15\% each)
\item Due Tuesdays 11:59 pm March 24 through April 28 (exam week)
\item Each assignment consists of three exercises
\item Peer assessments (10\% each)
  \end{baritemize}
\end{frame}

\begin{frame}{Account Options and Model Access}
\begin{center}\scriptsize
\begin{tabular}{@{} p{2.2cm} p{2.5cm} p{2.5cm} p{2.5cm} p{2.5cm} @{}}
\toprule
& \textbf{Claude Free} & \textbf{Claude Pro (\$20/mo)} & \textbf{ChatGPT Free} & \textbf{ChatGPT Plus (\$20/mo)} \\
\midrule
Chat model & Sonnet 4.5 & Sonnet 4.5 + limited Opus 4.6 & GPT-5 (limited) & GPT-5.2 \\
Code execution & No & Cowork (local VM) & Data Analysis & Data Analysis \\
Coding agent & No & Claude Code (Sonnet 4.5) & Codex (temporary) & Codex (GPT-5.3) \\
VS Code & No & Claude Code extension & No & Codex extension \\
\bottomrule
\end{tabular}
\end{center}
\end{frame}

\end{document}
