\documentclass[aspectratio=169]{beamer}
\usetheme{metropolis}
\usepackage{appendixnumberbeamer}
\usepackage{booktabs, hyperref}

\input{mgmt675-style}

\subtitle{MGMT 675: Generative AI for Finance}
\title{Course Introduction}
\author{Kerry Back}

\date{}
\titlegraphic{\includegraphics[width=4cm]{images/slide1_img1.png}}

\begin{document}

\maketitle

% ========================================
% SECTION 1: WHY AI MATTERS FOR FINANCE
% ========================================

\begin{frame}{From Makers to Checkers}
\centerline{Derek Waldron, JP Morgan Chief Analytics Officer}

What we're working towards is that every employee will have their own personalized AI assistant; every process is powered by AI agents, and every client experience has an AI concierge.

You'll still have people at the top who are managing and have relationships with clients, but many, many of the processes underneath are now being done by AI systems.

Workers would shift from being creators of reports or software updates, or 'makers' ... to 'checkers' or managers of AI agents doing that work.
\end{frame}

\begin{frame}{CFOs Are Going All In on AI}
\centerline{Deloitte Q4 2025 CFO Signals Survey --- 200 CFOs at \$1B+ companies}
\vspace{0.3cm}
\begin{columns}[T]
\begin{column}{0.5\textwidth}
\begin{itemize}
  \item \textbf{87\%} say AI will be extremely or very important to finance operations in 2026
  \item \textbf{54\%} prioritize integrating AI agents in their finance departments
  \item \textbf{50\%} cite digital transformation of finance as their \#1 priority
\end{itemize}
\end{column}
\begin{column}{0.5\textwidth}
\begin{itemize}
  \item \textbf{49\%} prioritize automating processes to free employees for higher-value work
  \item Only \textbf{2\%} say AI won't be important
  \item CFO confidence at highest level since late 2021
\end{itemize}
\end{column}
\end{columns}
\end{frame}

\begin{frame}[shrink=5]{Case Study: HPE's ``Alfred''}
\centerline{Marie Myers, CFO of Hewlett Packard Enterprise (\#143 on Fortune 500)}
\vspace{0.2cm}
\begin{columns}[T]
\begin{column}{0.5\textwidth}
\begin{shadedbox}[title=\textbf{The Problem}]
\begin{itemize}\small
  \item Weekly 90-minute operational review required 100+ slides
  \item Hundreds of hours of preparation across business units
  \item No time left for forward-looking analysis
\end{itemize}
\end{shadedbox}
\end{column}
\begin{column}{0.5\textwidth}
\begin{shadedbox}[title=\textbf{The AI Solution}]
\begin{itemize}\small
  \item Built ``Alfred''---AI agents that pull, reconcile, and analyze data automatically
  \item \textbf{90\%} of manual prep eliminated; cycle time reduced \textbf{40\%}, costs down \textbf{25\%}
  \item 3,000+ finance employees being reskilled to build their own agents
\end{itemize}
\end{shadedbox}
\end{column}
\end{columns}
\vspace{0.2cm}
\centering
``The goal is for finance professionals to become \alert{masters of their own destiny} rather than casualties of automation.'' --- Marie Myers
\end{frame}

\begin{frame}{Something Big is Happening}
\centerline{Matt Shumer, CEO of HyperWrite AI, Feb.\ 2025}

\begin{itemize}
  \item AI capability is doubling every few months --- 2022: couldn't do arithmetic; 2023: passed bar exam; 2024: wrote functional software; 2025--26: autonomous expert-level work
  \item AI now contributes to building its own successors (self-improving loop)
  \item Unlike prior automation, AI improves across \emph{all} knowledge domains simultaneously
\end{itemize}
\vfill
{\small \href{https://shumer.dev/something-big-is-happening}{shumer.dev/something-big-is-happening} \\
\href{https://www.linkedin.com/pulse/something-big-happening-matt-shumer-so5he/}{LinkedIn version}}
\end{frame}

\begin{frame}{Something Big is Happening (cont.)}
\centerline{Matt Shumer}

The person who walks into a meeting and says ``I used AI to do this analysis in an hour instead of three days'' is going to be the most valuable person in the room. Not eventually. Right now.

\bigskip

Here's a simple commitment that will put you ahead of almost everyone: spend one hour a day experimenting with AI. Not passively reading about it. Using it. Every day, try to get it to do something new \ldots\ something you haven't tried before, something you're not sure it can handle. Try a new tool. Give it a harder problem. One hour a day, every day. If you do this for the next six months, you will understand what's coming better than 99\% of the people around you. That's not an exaggeration. Almost nobody is doing this right now. The bar is on the floor.
\end{frame}

\begin{frame}{Why AI is Useful for Finance}
\begin{itemize}
  \item AI can search the web, pull data, analyze documents, and answer questions about company reports and communications
  \item AI can write code for financial analysis, generate Excel workbooks, draft reports, and build presentation decks
  \item AI can automate repetitive tasks and workflows, and communicate directly with clients (with strong controls)
\end{itemize}
\end{frame}


\begin{frame}{Why AI with Code Execution is Especially Useful}
\begin{itemize}
   \item Attaching a code execution tool enables:
  \begin{itemize}
    \item Data analysis with real calculations
    \item Visualizations and charts
    \item File processing (Excel, CSV, PDF)
     \end{itemize}
  \item Transforms chatbots into computational tools
\end{itemize}
\end{frame}


% ========================================
% SECTION 2: COURSE OUTLINE
% ========================================
\section{Course Outline}

\begin{frame}{Learning Objectives}
\begin{enumerate}
\item How to use AI with code-execution tools and other tools (terminal, browser, databases) to perform financial analysis, generate deliverables, and automate workflows
\item How to collaborate with AI through planning, executing, and evaluating---and how to create task-specific prompts for repeated use
\item How to reduce hallucinations through RAG, fine-tuning, or specialized models, and how to apply AI to tasks like sentiment-based trading
\end{enumerate}
\end{frame}

\begin{frame}{Course Topics}
\begin{columns}[T]
\begin{column}{0.45\textwidth}
\begin{enumerate}
  \item AI that writes and executes code
  \item AI coding for mean-variance analysis
  \item AI-written code in Jupyter notebooks
  \item Connecting tools to AI
  \item Connecting a virtual machine to AI
  \item Connecting your computer to AI
  \item Using AI inside Excel
  \item Replacing dashboards with natural language
\end{enumerate}
\end{column}
\begin{column}{0.45\textwidth}
\begin{enumerate}
  \setcounter{enumi}{8}
  \item Specialized prompt automation
  \item Using AI inside an IDE
  \item AI for DCF valuation
  \item Retrieval augmented generation
  \item Fine-tuning and small language models
  \item Building an AI agent
  \item Trading on news with AI
\end{enumerate}
\end{column}
\end{columns}
\end{frame}

\begin{frame}{Grading}
  \begin{itemize}
\item Six group assignments (15\% each), each consisting of three exercises
\item Due Tuesdays 11:59 pm March 24 through April 28 (exam week)
\item Peer assessments (10\% each)
  \end{itemize}
\end{frame}

\begin{frame}{Account Options and Model Access}
\begin{center}\scriptsize
\begin{tabular}{@{} p{2.2cm} p{2.5cm} p{2.5cm} p{2.5cm} p{2.5cm} @{}}
\toprule
& \textbf{Claude Free} & \textbf{Claude Pro (\$20/mo)} & \textbf{ChatGPT Free} & \textbf{ChatGPT Plus (\$20/mo)} \\
\midrule
Chat model & Sonnet 4.5 & Sonnet 4.5 + limited Opus 4.6 & GPT-5 (limited) & GPT-5.2 \\
Code execution & No & Cowork (local VM) & Data Analysis & Data Analysis \\
Coding agent & No & Claude Code (Sonnet 4.5) & Codex (temporary) & Codex (GPT-5.3) \\
VS Code & No & Claude Code extension & No & Codex extension \\
\bottomrule
\end{tabular}
\end{center}
\end{frame}

\end{document}
