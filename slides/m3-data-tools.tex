\documentclass[aspectratio=169]{beamer}
\usetheme{metropolis}
\usepackage{appendixnumberbeamer}
\usepackage{booktabs, hyperref, graphicx}

\input{mgmt675-style}

\usepackage{tikz}
\usetikzlibrary{shapes.geometric, arrows.meta, positioning, calc}

% Title info
\subtitle{MGMT 675: Generative AI for Finance}
\title{Module 3: Connecting AI to Data and Tools}
\author{Kerry Back}
\date{}

\begin{document}

\maketitle

% ========================================
% SECTION 1: THE CONNECTIVITY PROBLEM
% ========================================
\section{The Connectivity Problem}

\begin{frame}{AI in a Browser Is Powerful but Isolated}
In Modules 1--2 we used Claude's built-in sandbox to analyze data and build models. But the sandbox has limits:
\vspace{0.3cm}
\begin{itemize}
  \item Cannot access your local files, databases, or APIs
  \item Must upload data manually; no internet access from the sandbox
  \item State is lost between conversations
\end{itemize}
\vspace{0.3cm}
\alert{To do real financial work, AI needs to connect to the outside world.}
\end{frame}

\begin{frame}{Three Ways to Extend AI's Reach}
\begin{columns}[T]
\begin{column}{0.33\textwidth}
\begin{shadedbox}[title=\textbf{MCP Servers}]
\begin{itemize}\small
  \item Plug-in protocol
  \item Connect to APIs, databases, browsers
  \item Works with Claude Desktop
\end{itemize}
\end{shadedbox}
\end{column}
\begin{column}{0.33\textwidth}
\begin{shadedbox}[title=\textbf{Code Execution}]
\begin{itemize}\small
  \item Claude Code runs on your machine
  \item Full file and network access
  \item VS Code or terminal
\end{itemize}
\end{shadedbox}
\end{column}
\begin{column}{0.33\textwidth}
\begin{shadedbox}[title=\textbf{AI in Apps}]
\begin{itemize}\small
  \item Excel add-in, Colab, Sheets
  \item AI works inside tools you already use
  \item Platform-specific features
\end{itemize}
\end{shadedbox}
\end{column}
\end{columns}
\vspace{0.3cm}
This module covers all three approaches and when to use each.
\end{frame}

% ========================================
% SECTION 2: MCP — CONNECTING AI TO TOOLS
% ========================================
\section{MCP: The Universal Connector}

\begin{frame}{What is MCP?}
The \textbf{Model Context Protocol (MCP)} is an open standard that lets AI applications connect to external tools and data sources through a uniform interface.
\vspace{0.3cm}
\begin{columns}[T]
\begin{column}{0.5\textwidth}
\begin{shadedbox}[title=\textbf{The Idea}]
\begin{itemize}\small
  \item A standard ``plug-in'' protocol
  \item AI app connects to MCP servers
  \item Each server provides specific tools
  \item Tools appear automatically in the AI
\end{itemize}
\end{shadedbox}
\end{column}
\begin{column}{0.5\textwidth}
\begin{shadedbox}[title=\textbf{Analogy}]
\begin{itemize}\small
  \item USB is a standard for peripherals
  \item Plug in a keyboard, it just works
  \item MCP is a standard for AI tools
  \item Connect a server, tools just appear
\end{itemize}
\end{shadedbox}
\end{column}
\end{columns}
\end{frame}

\begin{frame}{How MCP Works}
\begin{center}
\begin{tikzpicture}[
  node distance=1.2cm and 2cm,
  every node/.style={font=\small},
  sbox/.style={draw=titlegray, rounded corners, minimum width=3.2cm, minimum height=0.9cm, fill=excelinput, text=titlegray, font=\small\bfseries, align=center},
  sarrow/.style={-{Stealth[length=3mm]}, thick, draw=accentblue}
]
\node[sbox, minimum width=3.5cm] (app) {Claude Desktop\\(MCP Client)};
\node[sbox, right=2.5cm of app, yshift=1.5cm] (s1) {Browser\\Server};
\node[sbox, right=2.5cm of app] (s2) {Financial\\Data Server};
\node[sbox, right=2.5cm of app, yshift=-1.5cm] (s3) {Terminal\\Server};

\draw[sarrow] (app) -- (s1) node[midway, above, font=\scriptsize\itshape, text=titlegray, sloped] {MCP protocol};
\draw[sarrow] (app) -- (s2);
\draw[sarrow] (app) -- (s3);

\node[right=0.5cm of s1, font=\scriptsize, text=titlegray] {Chrome, Firefox, ...};
\node[right=0.5cm of s2, font=\scriptsize, text=titlegray] {Alpha Vantage, FMP, ...};
\node[right=0.5cm of s3, font=\scriptsize, text=titlegray] {PowerShell, Bash, ...};
\end{tikzpicture}
\end{center}
\vspace{0.3cm}
\begin{itemize}
  \item Claude Desktop is the \textbf{client}; each external service runs as a \textbf{server}
  \item Servers expose \textbf{tools} (functions the AI can call) via a standard protocol
  \item You can connect multiple servers at once
\end{itemize}
\end{frame}

\begin{frame}{Two Ways to Install MCP Servers}
\begin{columns}[T]
\begin{column}{0.5\textwidth}
\begin{shadedbox}[title=\textbf{Desktop Extensions (Easy)}]
\begin{enumerate}\small
  \item Open Claude Desktop
  \item Go to \textbf{Settings $\rightarrow$ Extensions}
  \item Click \textbf{Browse extensions}
  \item Find the server you want
  \item Click \textbf{Install}
\end{enumerate}
\vspace{0.2cm}
One-click install. Anthropic-reviewed.
\end{shadedbox}
\end{column}
\begin{column}{0.5\textwidth}
\begin{shadedbox}[title=\textbf{Manual Configuration}]
\begin{enumerate}\small
  \item Install prerequisites (Node.js or uv)
  \item Edit a JSON config file
  \item Specify the server command
  \item Restart Claude Desktop
  \item Tools appear automatically
\end{enumerate}
\vspace{0.2cm}
More control. Works with any server.
\end{shadedbox}
\end{column}
\end{columns}
\end{frame}

\begin{frame}[fragile]{Prerequisites: Node.js and uv}
MCP servers are distributed as Node.js or Python packages.
\vspace{0.2cm}
\begin{columns}[T]
\begin{column}{0.5\textwidth}
\begin{shadedbox}[title=\textbf{Node.js} (for \texttt{npx} servers)]
\begin{itemize}\small
  \item Download the \textbf{LTS} installer from \url{https://nodejs.org}
  \item Run the installer (includes \texttt{npm} and \texttt{npx})
  \item Verify: \texttt{node --version}
\end{itemize}
\vspace{0.1cm}
{\scriptsize Used by: DesktopCommander, FMP}
\end{shadedbox}
\end{column}
\begin{column}{0.5\textwidth}
\begin{shadedbox}[title=\textbf{uv} (for \texttt{uvx} servers)]
\begin{itemize}\small
  \item \textbf{Mac}: \texttt{brew install uv}
  \item \textbf{Windows}: \texttt{winget install astral-sh.uv}
  \item Verify: \texttt{uv --version}
\end{itemize}
\vspace{0.1cm}
{\scriptsize Used by: Browser-Use, Alpha Vantage}
\end{shadedbox}
\end{column}
\end{columns}
\vspace{0.2cm}
\begin{center}\small
\alert{Install these before configuring any MCP server.}
\end{center}
\end{frame}

% ========================================
% SECTION 3: FINANCIAL DATA SERVERS
% ========================================
\section{Connecting to Financial Data}

\begin{frame}[fragile]{MCP Servers for Financial Data}
\begin{columns}[T]
\begin{column}{0.5\textwidth}
\begin{shadedbox}[title=\textbf{Alpha Vantage}]
\begin{itemize}\small
  \item 115+ tools: stock prices, options, forex, crypto, economic indicators
  \item Free API key at \url{https://www.alphavantage.co}
  \item \alert{Free tier}: 25 requests/day
\end{itemize}
\end{shadedbox}
\end{column}
\begin{column}{0.5\textwidth}
\begin{shadedbox}[title=\textbf{Financial Modeling Prep}]
\begin{itemize}\small
  \item 253+ tools: financial statements, fundamentals, SEC filings
  \item Free API key at \url{https://financialmodelingprep.com}
  \item \alert{Free tier}: 250 requests/day
\end{itemize}
\end{shadedbox}
\end{column}
\end{columns}
\vspace{0.3cm}
\begin{center}\small
Install the server, ask Claude to pull data --- \alert{no code, no API documentation needed}.\\[0.3em]
Example: ``Get Apple's income statement for the last 5 years and compute revenue growth rates''
\end{center}
\end{frame}

\begin{frame}[fragile]{Installing Financial Data Servers}
\begin{shadedbox}[title=\textbf{Alpha Vantage}]
\begin{verbatim}
"alphavantage": {
  "command": "uvx",
  "args": ["av-mcp", "YOUR_API_KEY"]
}
\end{verbatim}
\end{shadedbox}
\vspace{0.2cm}
\begin{shadedbox}[title=\textbf{Financial Modeling Prep}]
\begin{verbatim}
"financial-modeling-prep": {
  "command": "npx",
  "args": ["-y", "@houtini/fmp-mcp"],
  "env": { "FMP_API_KEY": "YOUR_API_KEY" }
}
\end{verbatim}
\end{shadedbox}
\vspace{0.2cm}
\begin{center}\small
Replace \texttt{YOUR\_API\_KEY} with your actual key. Then restart Claude Desktop.
\end{center}
\end{frame}

\begin{frame}{Browser-Use: Web Automation via MCP}
\textbf{Browser-Use} is an MCP server that lets Claude autonomously control a web browser. Describe a task in plain English, and it handles the navigation.
\vspace{0.3cm}
\begin{columns}[T]
\begin{column}{0.5\textwidth}
\textbf{Finance Applications}
\begin{itemize}\small
  \item ``Search for AAPL on Yahoo Finance and get the P/E ratio''
  \item ``Download all linked CSV files from this SEC page''
  \item ``Fill out this application form with my information''
\end{itemize}
\end{column}
\begin{column}{0.5\textwidth}
\textbf{Key Features}
\begin{itemize}\small
  \item Plain-language task descriptions
  \item Navigates pages and fills forms
  \item Extracts data from web pages
  \item \alert{Recommended}: local (self-hosted) for full privacy
\end{itemize}
\end{column}
\end{columns}
\end{frame}

% ========================================
% SECTION 4: CONNECTING TO YOUR COMPUTER
% ========================================
\section{Connecting to Your Computer}

\begin{frame}{Three Ways Claude Runs Code}
\begin{columns}[T]
\begin{column}{0.32\textwidth}
\begin{shadedbox}[title=\textbf{Chat (Analysis)}]
\begin{itemize}\footnotesize
  \item Sandboxed Python in the browser
  \item No local file access
  \item Must upload data manually
  \item Good for quick calculations
\end{itemize}
\end{shadedbox}
\end{column}
\begin{column}{0.32\textwidth}
\begin{shadedbox}[title=\textbf{Cowork}]
\begin{itemize}\footnotesize
  \item Runs in a local VM
  \item Can install packages
  \item Files synced to/from VM
  \item \alert{No internet access}
\end{itemize}
\end{shadedbox}
\end{column}
\begin{column}{0.32\textwidth}
\begin{shadedbox}[title=\textbf{Code}]
\begin{itemize}\footnotesize
  \item Runs on \alert{your machine}
  \item Full local file access
  \item Full internet access
  \item Most token-efficient
\end{itemize}
\end{shadedbox}
\end{column}
\end{columns}
\vspace{0.3cm}
\begin{center}
\alert{Code mode} gives Claude direct access to your machine --- no sandbox, no VM, no upload step.
\end{center}
\end{frame}

\begin{frame}{DesktopCommander: Terminal Control via MCP}
\textbf{DesktopCommander} gives Claude Desktop the ability to execute terminal commands, manage files, and control processes --- without switching to Code mode.
\vspace{0.3cm}
\begin{columns}[T]
\begin{column}{0.5\textwidth}
\begin{shadedbox}[title=\textbf{What It Can Do}]
\begin{itemize}\small
  \item Run terminal commands (\texttt{pip install}, git, scripts)
  \item Read, write, and search files
  \item Manage running processes
\end{itemize}
\end{shadedbox}
\end{column}
\begin{column}{0.5\textwidth}
\begin{shadedbox}[title=\textbf{When to Use It}]
\begin{itemize}\small
  \item You want to stay in Claude Desktop Chat
  \item Quick terminal tasks
  \item \alert{Always review commands before approving}
\end{itemize}
\end{shadedbox}
\end{column}
\end{columns}
\end{frame}

\begin{frame}{Which Tool When?}
\begin{center}
\small
\begin{tabular}{@{} p{4.5cm} c c c c @{}}
\toprule
\textbf{Task} & \textbf{Chat} & \textbf{Artifacts} & \textbf{Cowork} & \textbf{Code} \\
\midrule
Explain WACC formula & \checkmark & & & \\
Interactive DCF calculator & & \checkmark & & \\
Analyze CSV, create Excel & & & \checkmark & \checkmark \\
Run Fama--French regressions & & & \checkmark & \checkmark \\
Organize 50 PDF 10-Ks & & & \checkmark & \\
Fetch live data from APIs & & & & \checkmark \\
\bottomrule
\end{tabular}
\end{center}
\vspace{0.3cm}
\textbf{Rule of thumb:} Start with Chat. If you need a visual, use Artifacts. If you need file I/O, use Code. Save Cowork for complex multi-file tasks.
\end{frame}

% ========================================
% SECTION 5: WORKING IN NOTEBOOKS
% ========================================
\section{Working in Notebooks}

\begin{frame}{Google Colab + Gemini}
\begin{columns}[T]
\begin{column}{0.5\textwidth}
\begin{shadedbox}[title=\textbf{What is Colab?}]
\begin{itemize}\small
  \item Free browser-based notebooks from Google
  \item No installation required
  \item Saves to Google Drive
  \item Gemini AI built in
\end{itemize}
\end{shadedbox}
\end{column}
\begin{column}{0.5\textwidth}
\begin{shadedbox}[title=\textbf{How Gemini Helps}]
\begin{itemize}\small
  \item Generate code from English descriptions
  \item Explain existing code and fix errors
  \item Suggest improvements
  \item Mount Google Drive for data files
\end{itemize}
\end{shadedbox}
\end{column}
\end{columns}
\vspace{0.3cm}
Philosophy: \textit{code environment with a chatbot} (vs.\ Claude: chatbot with code execution).
\end{frame}

\begin{frame}{VS Code + Claude Code}
\begin{columns}[T]
\begin{column}{0.5\textwidth}
\begin{shadedbox}[title=\textbf{VS Code}]
\begin{itemize}\small
  \item Free desktop editor from Microsoft
  \item Works on Windows, Mac, Linux
  \item Supports Jupyter notebooks
  \item Explorer, Terminal, Extensions
\end{itemize}
\end{shadedbox}
\end{column}
\begin{column}{0.5\textwidth}
\begin{shadedbox}[title=\textbf{Claude Code Extension}]
\begin{itemize}\small
  \item AI sidebar inside VS Code
  \item Reference files with \texttt{@filename}
  \item Writes and runs code with your approval
  \item Same Claude Code as the terminal/desktop
\end{itemize}
\end{shadedbox}
\end{column}
\end{columns}
\vspace{0.3cm}
\begin{center}
\textbf{When to use which:} Colab for quick browser-based exploration; VS Code for serious projects with local files.
\end{center}
\end{frame}

% ========================================
% SECTION 6: AI IN SPREADSHEETS
% ========================================
\section{AI in Spreadsheets}

\begin{frame}{Formulas vs.\ Hardcoded Values}
\begin{columns}[T]
  \begin{column}{0.45\textwidth}
\begin{shadedbox}[title=\textbf{Hardcoded (Bad)}]
\small
\texttt{sheet['B10'] = 1500}
\vspace{0.3cm}

Cell shows 1500, but if inputs change, the total doesn't update.
\end{shadedbox}
\end{column}
\begin{column}{0.45\textwidth}
\begin{shadedbox}[title=\textbf{Formula (Good)}]
\small
\texttt{sheet['B10'] = '=SUM(B2:B9)'}
\vspace{0.3cm}

Cell contains a formula that recalculates when inputs change.
\end{shadedbox}
\end{column}
\end{columns}
\vspace{0.5cm}
\centerline{\alert{Always instruct AI to use formulas, not hardcoded values.}}
\end{frame}

\begin{frame}{Claude for Excel Add-in}
\begin{columns}[T]
\begin{column}{0.5\textwidth}
\begin{shadedbox}[title=\textbf{What It Does}]
\begin{itemize}\small
  \item AI sidebar inside Excel
  \item Reads your workbook, modifies cells
  \item Preserves formula dependencies
  \item Works with local files --- \alert{no OneDrive required}
\end{itemize}
\end{shadedbox}
\end{column}
\begin{column}{0.5\textwidth}
\begin{shadedbox}[title=\textbf{Capabilities}]
\begin{itemize}\small
  \item Ask questions about workbook content
  \item Build financial models from scratch
  \item Trace and fix formula errors
  \item Create charts and pivot tables
\end{itemize}
\end{shadedbox}
\end{column}
\end{columns}
\vspace{0.3cm}
\begin{center}\small
Requires Claude Pro (\$20/mo). Install via \textbf{Insert $\rightarrow$ Get Add-ins $\rightarrow$ ``Claude by Anthropic''}.
\end{center}
\end{frame}

\begin{frame}[shrink=3]{AI Spreadsheet Tools Compared}
\begin{center}
\scriptsize
\begin{tabular}{p{2.5cm} p{3.2cm} p{3.2cm} p{3.2cm}}
\toprule
 & \textbf{Claude for Excel} & \textbf{Microsoft Copilot} & \textbf{Google Sheets + Gemini} \\
\midrule
Cost & Claude Pro (\$20/mo) & M365 Copilot (\$30/mo) & Google One AI Pro (\$20/mo) \\
Platform & Excel (Win/Mac/Web) & Excel (Win/Mac/Web) & Google Sheets (Web) \\
Key feature & Formula-preserving edits & Agent Mode, \texttt{=COPILOT()} & \texttt{=AI()} function \\
Python & Server-side sandbox & Python in Excel & Apps Script \\
OneDrive & Not required & Required (AutoSave on) & N/A (Google Drive) \\
Best for & Financial modeling & Enterprise ecosystem & Cloud collaboration \\
\bottomrule
\end{tabular}
\end{center}
\vspace{0.2cm}
All three can produce the same kinds of financial tools --- DCF calculators, amortization tables, scenario analyzers.  Choose based on your ecosystem.
\end{frame}

% ========================================
% SECTION 7: COMBINING CONNECTIONS
% ========================================
\section{Combining Multiple Connections}

\begin{frame}[fragile,shrink=5]{Multi-Server MCP Configurations}
You can connect multiple MCP servers at once. Claude sees all their tools and can use them together.
\vspace{0.2cm}
\begin{shadedbox}[title=\textbf{Example: Three Servers in One Config}]
\begin{verbatim}
{
  "mcpServers": {
    "browser-use": {
      "command": "uvx",
      "args": ["--from", "browser-use[cli]",
               "browser-use", "--mcp"],
      "env": {"ANTHROPIC_API_KEY": "sk-ant-..."}
    },
    "desktop-commander": {
      "command": "npx",
      "args": ["-y", "desktop-commander-mcp"]
    },
    "alphavantage": {
      "command": "uvx",
      "args": ["av-mcp", "YOUR_AV_KEY"]
    }
  }
}
\end{verbatim}
\end{shadedbox}
\end{frame}

\begin{frame}{Built-In Alternatives to MCP}
DesktopCommander and Browser-Use add capabilities to Claude's \textbf{Chat} mode. But Code and Cowork modes have some of these built in.
\vspace{0.2cm}
\begin{center}
\small
\begin{tabular}{@{} l c c c @{}}
\toprule
& \textbf{Chat} & \textbf{Code tab} & \textbf{Cowork tab} \\
\midrule
Terminal / files & \alert{MCP needed} & Built in & Built in \\
Browser control & \alert{MCP needed} & Chrome extension & Chrome extension \\
Internet access & No & Yes & No \\
\bottomrule
\end{tabular}
\end{center}
\vspace{0.2cm}
MCP servers for \textbf{data sources} (Alpha Vantage, FMP, databases) remain valuable on any mode.
\end{frame}

% ========================================
% SECTION 8: EXERCISES
% ========================================
\section{Exercises}

\begin{frame}{Exercise 1: Financial Data via MCP}
\begin{enumerate}
  \item Get a free API key from Alpha Vantage or Financial Modeling Prep
  \item Install the corresponding MCP server in Claude Desktop
  \item Ask Claude to retrieve a company's financial data, analyze trends, and produce formatted Excel output
\end{enumerate}
\vspace{0.3cm}
\alert{No code required} --- just install the server and ask questions in natural language.
\end{frame}

\begin{frame}{Exercise 2: Browser Automation}
Use Browser-Use MCP to navigate a financial website:
\begin{itemize}
  \item Extract market cap, P/E ratio, and revenue for a company
  \item Compile a summary comparison of 3 companies
  \item Save results to a formatted document
\end{itemize}
\end{frame}

\begin{frame}{Exercise 3: Stock Analysis in Colab}
\begin{enumerate}
  \item Open Google Colab (\url{https://colab.research.google.com})
  \item Ask Gemini to download daily returns for 3 stocks using \texttt{pandas-datareader} or \texttt{yfinance}
  \item Ask Gemini to compute annualized mean returns, volatilities, betas, and a correlation matrix
  \item Ask Gemini to create boxplots and cumulative return charts
\end{enumerate}
\end{frame}

\begin{frame}{Exercise 4: Loan Amortization in Excel}
\begin{enumerate}
  \item Open a new workbook in Excel
  \item Launch the Claude add-in (Insert $\rightarrow$ Claude icon)
  \item Ask Claude to build a 30-year mortgage amortization table for a \$200{,}000 loan at 6.5\% with monthly payments
  \item Ask Claude to chart principal vs.\ interest over time
  \item Verify: \alert{all values should use live formulas, not hardcoded numbers}
\end{enumerate}
\end{frame}

\begin{frame}{Summary}
\begin{columns}[T]
\begin{column}{0.33\textwidth}
\begin{shadedbox}[title=\textbf{MCP}]
\begin{itemize}\small
  \item Universal plug-in protocol
  \item Financial data, browser, terminal
  \item Works with Claude Desktop
\end{itemize}
\end{shadedbox}
\end{column}
\begin{column}{0.33\textwidth}
\begin{shadedbox}[title=\textbf{Code Execution}]
\begin{itemize}\small
  \item Chat, Cowork, Code modes
  \item Match the tool to the task
  \item Code for APIs and files
\end{itemize}
\end{shadedbox}
\end{column}
\begin{column}{0.33\textwidth}
\begin{shadedbox}[title=\textbf{AI in Apps}]
\begin{itemize}\small
  \item Excel, Colab, Sheets
  \item Formulas, not hardcodes
  \item Choose by ecosystem
\end{itemize}
\end{shadedbox}
\end{column}
\end{columns}
\vspace{0.5cm}
\begin{center}
\alert{AI becomes truly powerful when it can reach your data, your tools, and your files.}
\end{center}
\end{frame}

\end{document}
