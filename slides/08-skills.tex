\documentclass[aspectratio=169]{beamer}
\usetheme{metropolis}
\usepackage{appendixnumberbeamer}
\usepackage{booktabs, hyperref}

\input{mgmt675-style}

\subtitle{MGMT 675: Generative AI for Finance}
\title{Automating Specialized Prompts}
\author{Kerry Back}

\date{}

\begin{document}

\maketitle

\begin{frame}{What is a Skill?}
\begin{shadedbox}
A \textbf{skill} is a folder containing instructions and optional code that transforms Claude Code into a specialized agent for a specific domain or task.
\end{shadedbox}
\vspace{0.5cm}
\begin{columns}[T]
\begin{column}{0.5\textwidth}
\begin{shadedbox}[title=\textbf{A Skill Provides}]
\begin{itemize}\small
  \item System prompt (domain knowledge)
  \item Workflow instructions
  \item Best practices and constraints
  \item Python/JavaScript helper scripts
  \item Reference documentation
\end{itemize}
\end{shadedbox}
\end{column}
\begin{column}{0.5\textwidth}
\begin{shadedbox}[title=\textbf{Claude Code Provides}]
\begin{itemize}\small
  \item The LLM (Opus or Sonnet)
  \item Agent control loop
  \item File read/write tools
  \item Code execution sandbox
  \item Web search capability
\end{itemize}
\end{shadedbox}
\end{column}
\end{columns}
\end{frame}

\begin{frame}[fragile]{Anatomy of a Skill}
\begin{columns}[T]
\begin{column}{0.45\textwidth}
\begin{shadedbox}[title=\textbf{Skill Folder Structure}]\small
\begin{verbatim}
skills/
  xlsx/
    SKILL.md     <- Main file
    scripts/
      recalc.py
    references/
      schema.md
\end{verbatim}
\end{shadedbox}
\end{column}
\begin{column}{0.55\textwidth}
\begin{shadedbox}[title=\textbf{SKILL.md Structure}]\small
\begin{verbatim}
---
name: xlsx
description: "Excel file..."
---

# Requirements for Outputs
- Zero formula errors
- Use formulas, not hardcodes

# Workflows
1. Choose pandas or openpyxl
2. Create/modify file
3. Recalculate formulas
\end{verbatim}
\end{shadedbox}
\end{column}
\end{columns}
\end{frame}

\begin{frame}{SKILL.md: The System Prompt}
\begin{baritemize}
  \item \textbf{Frontmatter}: Name and description (YAML header)
  \item \textbf{Requirements}: Quality standards and constraints
  \item \textbf{Workflows}: Step-by-step procedures
  \item \textbf{Code Examples}: Patterns for the LLM to follow
  \item \textbf{Error Handling}: How to diagnose and fix problems
\end{baritemize}
\vspace{0.3cm}
\begin{shadedbox}
The SKILL.md file is automatically injected into Claude's context when working in a project that contains the skill.
\end{shadedbox}
\end{frame}

\begin{frame}{Scripts: The Custom Tools}
\begin{columns}
\begin{column}{0.5\textwidth}
\begin{baritemize}\small
  \item Python or JavaScript files in \texttt{scripts/} folder
  \item Claude Code can execute them
  \item Extend Claude's capabilities
  \item Handle tasks LLM can't do directly
\end{baritemize}
\end{column}
\begin{column}{0.5\textwidth}
\begin{shadedbox}[title=\textbf{Example Scripts}]
\begin{itemize}\small
  \item \texttt{recalc.py} -- Recalculate Excel formulas via LibreOffice
  \item \texttt{validate.py} -- Check file structure
  \item \texttt{unpack.py} -- Extract XML from Office files
  \item \texttt{html2pptx.js} -- Convert HTML to PowerPoint
\end{itemize}
\end{shadedbox}
\end{column}
\end{columns}
\end{frame}

\begin{frame}{Example: Rice Database Skill}
\begin{columns}
\begin{column}{0.5\textwidth}
\begin{shadedbox}[title=\textbf{Stand-Alone App}]
\begin{itemize}\small
  \item Custom web application
  \item Hard-coded SQL generation prompt
  \item Fixed agent logic in Python
  \item Single-purpose: query database
  \item Requires developer to update
\end{itemize}
\end{shadedbox}
\end{column}
\begin{column}{0.5\textwidth}
\begin{shadedbox}[title=\textbf{Claude Code + Skill}]
\begin{itemize}\small
  \item SKILL.md with database schema
  \item Connection code examples
  \item Table descriptions
  \item \alert{Plus}: Can also make charts, Excel files, Word reports
  \item User can extend easily
\end{itemize}
\end{shadedbox}
\end{column}
\end{columns}
\vspace{0.3cm}
\begin{center}
\alert{Same database access, but infinitely more flexible}
\end{center}
\end{frame}

\begin{frame}{Skill vs Stand-Alone Agent}
\begin{center}
\begin{tabular}{lcc}
\toprule
\textbf{Feature} & \textbf{Stand-Alone} & \textbf{Skill} \\
\midrule
Agent logic & Custom code & Claude Code \\
System prompt & Hard-coded & SKILL.md file \\
LLM & Your choice & Claude Opus/Sonnet \\
Tools & Custom built & Scripts + built-in \\
Maintenance & Developer & Edit markdown \\
Combine tasks & No & Yes \\
\bottomrule
\end{tabular}
\end{center}
\vspace{0.5cm}
\begin{shadedbox}
Skills let you create \alert{specialized agents} without writing agent logic. Claude Code handles the control flow; you just provide the domain knowledge.
\end{shadedbox}
\end{frame}

\begin{frame}{Where Skills Live}
\begin{baritemize}
  \item \textbf{Project skills}: \texttt{.claude/skills/} in your project folder
  \item \textbf{User skills}: \texttt{\textasciitilde/.claude/skills/} (shared across projects)
  \item Claude Code automatically loads skills from both locations
  \item Skills can reference each other (e.g., xlsx skill uses ooxml validation)
  \item Skills also work in \textbf{Claude.ai}, \textbf{Chat}, and \textbf{Cowork} (details later)
\end{baritemize}
\vspace{0.3cm}
\begin{shadedbox}
\textbf{Community Skills Repository:} \href{https://github.com/VoltAgent/awesome-agent-skills}{github.com/VoltAgent/awesome-agent-skills}\\[0.2em]
200+ skills from official teams and the community
\end{shadedbox}
\end{frame}

\begin{frame}{Creating Your Own Skill}
\begin{barenumerate}
  \item Create folder: \texttt{.claude/skills/my-skill/}
  \item Create \texttt{SKILL.md} with:
  \begin{itemize}
    \item YAML frontmatter (name, description)
    \item Domain knowledge and instructions
    \item Code examples and workflows
  \end{itemize}
  \item Optionally add \texttt{scripts/} folder with helper code
  \item Start using Claude Code -- skill is automatically loaded
\end{barenumerate}
\end{frame}

\begin{frame}[fragile]{Example: Investment Memo Skill}
\begin{columns}[T]
\begin{column}{0.45\textwidth}
\begin{shadedbox}[title=\textbf{SKILL.md}]\scriptsize
\begin{verbatim}
---
name: investment-memo
description: "Create investment
  memos with DCF, comps,
  and risk analysis"
---

# Output Format
- Executive summary (1 paragraph)
- Business overview
- Financial analysis (DCF + comps)
- Risk factors (bull/bear case)
- Recommendation with price target

# Workflows
1. Gather financials from SEC filings
   or uploaded files
2. Build DCF in Excel with formulas
3. Pull peer group multiples
4. Write memo in Word format
\end{verbatim}
\end{shadedbox}
\end{column}
\begin{column}{0.55\textwidth}
\begin{shadedbox}[title=\textbf{What This Gives You}]
\begin{itemize}\small
  \item Type: ``Write an investment memo for Apple''
  \item Claude follows the structure automatically
  \item Consistent format across every memo
  \item Produces Excel model + Word report
\end{itemize}
\end{shadedbox}
\vspace{0.2cm}
\begin{shadedbox}[title=\textbf{Other Finance Skill Ideas}]
\begin{itemize}\small
  \item \textbf{Earnings call}: Summarize transcripts, extract guidance, flag surprises
  \item \textbf{Portfolio report}: Monthly performance, attribution, charts
  \item \textbf{Loan analysis}: Parse credit agreements, extract covenants
  \item \textbf{ESG scoring}: Rate companies on ESG criteria from filings
\end{itemize}
\end{shadedbox}
\end{column}
\end{columns}
\end{frame}

\begin{frame}{Skills Across Claude Formats}
\begin{shadedbox}
Skills are not limited to Claude Code. The same skill can be deployed and used across Claude.ai, Claude Desktop, and Claude Code.
\end{shadedbox}
\vspace{0.2cm}
\begin{center}
\small
\begin{tabular}{@{} l l l @{}}
\toprule
\textbf{Format} & \textbf{How to Install} & \textbf{How to Invoke} \\
\midrule
Claude.ai (web) & Upload ZIP in Settings $\rightarrow$ Capabilities & Automatic or \texttt{/name} \\
Chat (Desktop) & Same as Claude.ai (shared account) & Automatic or \texttt{/name} \\
Cowork (Desktop) & Install as a plugin via sidebar & Automatic or \texttt{/name} \\
Code tab (Desktop) & Place in \texttt{.claude/skills/} & Automatic or \texttt{/name} \\
Claude Code CLI & Place in \texttt{.claude/skills/} & Automatic or \texttt{/name} \\
VS Code extension & Place in \texttt{.claude/skills/} & Automatic or \texttt{/name} \\
\bottomrule
\end{tabular}
\end{center}
\vspace{0.2cm}
\begin{baritemize}
  \item \textbf{Automatic}: Claude loads the skill when it detects a relevant task
  \item \textbf{\texttt{/name}}: You invoke it explicitly (e.g., \texttt{/investment-memo})
  \item Skills in \texttt{\textasciitilde/.claude/skills/} are shared across all your projects
\end{baritemize}
\end{frame}

\begin{frame}{Deploying Skills: Three Methods}
\begin{columns}[T]
\begin{column}{0.33\textwidth}
\begin{shadedbox}[title=\textbf{Claude.ai / Chat}]
\begin{enumerate}\scriptsize
  \item ZIP your skill folder
  \item Go to \textbf{Settings} $\rightarrow$ \textbf{Capabilities}
  \item Click \textbf{Upload skill}
  \item Toggle it \textbf{ON}
  \item Requires code execution enabled
\end{enumerate}
\vspace{0.1cm}
{\scriptsize Pro, Max, Team, Enterprise}
\end{shadedbox}
\end{column}
\begin{column}{0.33\textwidth}
\begin{shadedbox}[title=\textbf{Cowork}]
\begin{enumerate}\scriptsize
  \item Click \textbf{Plugins} in sidebar
  \item Browse official plugins or click \textbf{Upload plugin}
  \item Plugins bundle skills + connectors + slash commands
  \item 11 official plugins available
\end{enumerate}
\vspace{0.1cm}
{\scriptsize Pro, Max, Team, Enterprise}
\end{shadedbox}
\end{column}
\begin{column}{0.33\textwidth}
\begin{shadedbox}[title=\textbf{Code / CLI / VS Code}]
\begin{enumerate}\scriptsize
  \item Create folder:\newline\texttt{.claude/skills/name/}
  \item Add \texttt{SKILL.md}
  \item Optionally add \texttt{scripts/}
  \item Done---auto-detected
\end{enumerate}
\vspace{0.35cm}
{\scriptsize Pro, Max (via Claude Code)}
\end{shadedbox}
\end{column}
\end{columns}
\end{frame}

\begin{frame}{Using Skills: Examples Across Formats}
\begin{columns}[T]
\begin{column}{0.5\textwidth}
\begin{shadedbox}[title=\textbf{Claude.ai or Chat}]
\begin{itemize}\small
  \item ``Write an investment memo for Tesla''
  \item Claude sees the skill is relevant and loads it automatically
  \item Uses code execution to build Excel model
  \item Produces downloadable Word doc
\end{itemize}
\end{shadedbox}
\vspace{0.2cm}
\begin{shadedbox}[title=\textbf{Claude Code / VS Code}]
\begin{itemize}\small
  \item Type \texttt{/investment-memo} to invoke explicitly
  \item Or just describe the task---Claude loads the skill automatically
  \item Creates files directly in your project folder
  \item Can combine with other skills (e.g., xlsx skill for the model)
\end{itemize}
\end{shadedbox}
\end{column}
\begin{column}{0.5\textwidth}
\begin{shadedbox}[title=\textbf{Cowork}]
\begin{itemize}\small
  \item Drag a folder of 10-K filings into Cowork
  \item ``Analyze these filings and produce an investment memo for each company''
  \item Claude works through them autonomously
  \item Outputs organized files to your folder
\end{itemize}
\end{shadedbox}
\vspace{0.2cm}
\begin{shadedbox}[title=\textbf{Key Difference}]
\begin{itemize}\small
  \item \textbf{Claude.ai / Chat}: best for one-off tasks; download results manually
  \item \textbf{Cowork}: best for batch tasks on local files
  \item \textbf{Code / VS Code}: best when you want to see and edit the code
\end{itemize}
\end{shadedbox}
\end{column}
\end{columns}
\end{frame}

\begin{frame}{Skills vs.\ Subagents}
\begin{shadedbox}
Skills and subagents both customize Claude, but they serve different purposes. A \textbf{skill} is a reference manual Claude follows in your conversation. A \textbf{subagent} is a separate worker you dispatch to handle a task independently.
\end{shadedbox}
\vspace{0.2cm}
\begin{center}
\small
\begin{tabular}{@{} l l l @{}}
\toprule
& \textbf{Skill} & \textbf{Subagent} \\
\midrule
What it is & Instructions + scripts & A separate agent \\
Analogy & Giving Claude a playbook & Handing a task to a colleague \\
Runs where & Same conversation & Its own context window \\
Works in & Claude.ai, Chat, Cowork, Code & Claude Code only \\
Defined in & \texttt{SKILL.md} & \texttt{AGENT.md} \\
Created with & Manual or \texttt{/agents} & \texttt{/agents} \\
Best for & Standards, templates, workflows & Delegating well-defined tasks \\
\bottomrule
\end{tabular}
\end{center}
\vspace{0.2cm}
\begin{baritemize}
  \item \textbf{Use a skill} when you want Claude to follow specific standards \textit{while you work with it interactively} (e.g., ``always format memos this way'')
  \item \textbf{Use a subagent} when you want to \textit{hand off} a self-contained task (e.g., ``go analyze these 10 files and report back'')
  \item They combine naturally: a subagent can have skills loaded, giving it both independence and domain expertise
\end{baritemize}
\end{frame}

\begin{frame}{Summary}
\begin{columns}[T]
\begin{column}{0.5\textwidth}
\begin{shadedbox}[title=\textbf{Claude Code}]
\begin{itemize}\small
  \item General-purpose agent
  \item LLM + Agent Logic + Tools
  \item Works out of the box
\end{itemize}
\end{shadedbox}
\end{column}
\begin{column}{0.5\textwidth}
\begin{shadedbox}[title=\textbf{Skills}]
\begin{itemize}\small
  \item Specialize Claude Code
  \item System prompt + scripts
  \item Easy to create and modify
\end{itemize}
\end{shadedbox}
\end{column}
\end{columns}
\vspace{0.5cm}
\begin{center}
\begin{shadedbox}[width=0.8\textwidth]
\centering
\textbf{Skills replace stand-alone agents}\\[0.3em]
Same capabilities, less code, more flexibility
\end{shadedbox}
\end{center}
\end{frame}

\end{document}
