\documentclass[aspectratio=169]{beamer}
\usetheme{metropolis}
\usepackage{appendixnumberbeamer}
\usepackage{booktabs}

\input{mgmt675-style}

\subtitle{MGMT 675: Generative AI for Finance}
\title{Creating AI Agents}
\author{Kerry Back}

\date{}

\begin{document}

\maketitle

\begin{frame}{Chatbot vs Agent}
\begin{columns}[T]
\begin{column}{0.5\textwidth}
\begin{shadedbox}[title=\textbf{Chatbot}]
\begin{itemize}\small
  \item Receives prompts
  \item Generates text responses
  \item That's it---just conversation
  \item Cannot take actions
  \item Cannot access external data
\end{itemize}
\end{shadedbox}
\end{column}
\begin{column}{0.5\textwidth}
\begin{shadedbox}[title=\textbf{Agent}]
\begin{itemize}\small
  \item Receives prompts
  \item Can generate text \alert{or} request actions
  \item Has access to \textbf{tools}
  \item Can query databases, run code, search web
  \item Takes actions to accomplish goals
\end{itemize}
\end{shadedbox}
\end{column}
\end{columns}
\vspace{0.3cm}
\begin{center}
\alert{An agent is a chatbot with tools}
\end{center}
\end{frame}

\begin{frame}{What Are Tools?}
\begin{shadedbox}
\textbf{Tools} are functions the agent can call to interact with the outside world.
\end{shadedbox}
\vspace{0.3cm}
\begin{columns}[T]
\begin{column}{0.5\textwidth}
\begin{shadedbox}[title=\textbf{Example Tools}]
\begin{itemize}\small
  \item Execute SQL queries
  \item Run Python code
  \item Read/write files
  \item Search the web
  \item Send emails
  \item Call APIs
\end{itemize}
\end{shadedbox}
\end{column}
\begin{column}{0.5\textwidth}
\begin{shadedbox}[title=\textbf{How Tools Work}]
\begin{enumerate}\small
  \item LLM decides to use a tool
  \item Returns tool name + parameters
  \item Agent executes the tool
  \item Result sent back to LLM
  \item LLM continues reasoning
\end{enumerate}
\end{shadedbox}
\end{column}
\end{columns}
\end{frame}

\begin{frame}{Example: Database Analytics Agent}
\begin{shadedbox}
User request: ``Analyze quarterly revenue trends for our top 5 customers and create a summary report.''
\end{shadedbox}
\vspace{0.3cm}
\begin{shadedbox}[title=\textbf{What the Agent Needs to Do}]
\begin{enumerate}\small
  \item Write SQL to identify top 5 customers by revenue
  \item Execute the SQL query against the database
  \item Write SQL to get quarterly revenue for those customers
  \item Execute that query
  \item Write Python to analyze trends and create visualizations
  \item Execute the Python code
  \item Analyze the results
  \item Generate a written report for the user
\end{enumerate}
\end{shadedbox}
\end{frame}

\begin{frame}{The Agent Loop}
\begin{shadedbox}
The agent runs in a loop: LLM thinks $\rightarrow$ tool executes $\rightarrow$ result returns $\rightarrow$ LLM thinks again.
\end{shadedbox}
\vspace{0.3cm}
\begin{center}
\begin{tabular}{ccc}
& \textbf{Tool Call} & \\
\fbox{\textbf{LLM}} & $\longrightarrow$ & \fbox{\textbf{Tool}} \\
& $\longleftarrow$ & \\
& \textbf{Result} & \\
\end{tabular}
\end{center}
\vspace{0.3cm}
\begin{columns}[T]
\begin{column}{0.5\textwidth}
\begin{baritemize}\small
  \item LLM decides next action
  \item Returns tool name + parameters
  \item Agent executes the tool
\end{baritemize}
\end{column}
\begin{column}{0.5\textwidth}
\begin{baritemize}\small
  \item Result added to message history
  \item LLM sees result, reasons again
  \item Loop continues until task complete
\end{baritemize}
\end{column}
\end{columns}
\end{frame}

\begin{frame}{Step 1: User Prompt Arrives}
\begin{shadedbox}[title=\textbf{Messages Sent to LLM}]
\begin{tabular}{ll}
\textbf{Role} & \textbf{Content} \\
\midrule
system & You are a data analyst with SQL and Python tools... \\
user & Analyze quarterly revenue trends for top 5 customers... \\
\end{tabular}
\end{shadedbox}
\vspace{0.3cm}
\begin{columns}[T]
\begin{column}{0.5\textwidth}
\begin{baritemize}\small
  \item System prompt defines capabilities
  \item Lists available tools
  \item Specifies database schema
  \item Sets response format
\end{baritemize}
\end{column}
\begin{column}{0.5\textwidth}
\begin{shadedbox}[title=\textbf{LLM Decides}]
``I need to first find the top 5 customers. I'll write a SQL query.''
\end{shadedbox}
\end{column}
\end{columns}
\end{frame}

\begin{frame}[fragile]{Step 2: LLM Requests SQL Tool}
\begin{shadedbox}[title=\textbf{LLM Response (Not Text---A Tool Call)}]
\begin{verbatim}
{
  "tool": "execute_sql",
  "parameters": {
    "query": "SELECT customer_id, SUM(amount) as total
              FROM sales GROUP BY customer_id
              ORDER BY total DESC LIMIT 5"
  }
}
\end{verbatim}
\end{shadedbox}
\vspace{0.3cm}
\begin{baritemize}\small
  \item LLM doesn't return text to user yet
  \item Instead, requests a tool execution
  \item Agent code intercepts this and runs the SQL
  \item Database returns results
\end{baritemize}
\end{frame}

\begin{frame}[fragile]{Step 3: Tool Result Returns to LLM}
\begin{shadedbox}[title=\textbf{Messages Now Include Tool Result}]
\begin{verbatim}
[
  {"role": "system", "content": "You are a data analyst..."},
  {"role": "user", "content": "Analyze quarterly revenue..."},
  {"role": "assistant", "tool_call": "execute_sql(...)"},
  {"role": "tool", "content": "customer_id,total\n
                               ACME,450000\nGlobex,380000\n..."}
]
\end{verbatim}
\end{shadedbox}
\vspace{0.3cm}
\begin{center}
\alert{The LLM sees the full history including tool results}
\end{center}
\end{frame}

\begin{frame}{Step 4: LLM Continues Reasoning}
\begin{shadedbox}
With the top 5 customers identified, the LLM decides what to do next.
\end{shadedbox}
\vspace{0.3cm}
\begin{columns}[T]
\begin{column}{0.5\textwidth}
\begin{shadedbox}[title=\textbf{LLM Thinks}]
``I have the top 5 customers: ACME, Globex, Initech, Umbrella, Wayne. Now I need quarterly data for each.''
\end{shadedbox}
\end{column}
\begin{column}{0.5\textwidth}
\begin{shadedbox}[title=\textbf{Next Tool Call}]
\texttt{execute\_sql}: Get quarterly revenue for these 5 customers...
\end{shadedbox}
\end{column}
\end{columns}
\vspace{0.3cm}
\begin{center}
The loop continues: tool call $\rightarrow$ result $\rightarrow$ reasoning $\rightarrow$ next action
\end{center}
\end{frame}

\begin{frame}[fragile]{Step 5: Python for Analysis}
\begin{shadedbox}[title=\textbf{LLM Requests Python Execution}]
\begin{verbatim}
{
  "tool": "execute_python",
  "parameters": {
    "code": "import pandas as pd
             import matplotlib.pyplot as plt
             df = pd.DataFrame(quarterly_data)
             # Calculate growth rates
             # Create trend visualization
             plt.savefig('trends.png')"
  }
}
\end{verbatim}
\end{shadedbox}
\vspace{0.3cm}
\begin{baritemize}\small
  \item LLM writes Python code as a string
  \item Agent executes it in a sandbox
  \item Output (prints, files, errors) returned to LLM
\end{baritemize}
\end{frame}

\begin{frame}{Interim Communication with User}
\begin{shadedbox}
Sometimes the agent needs clarification or wants to provide progress updates.
\end{shadedbox}
\vspace{0.3cm}
\begin{columns}[T]
\begin{column}{0.5\textwidth}
\begin{shadedbox}[title=\textbf{Agent Might Ask}]
\begin{itemize}\small
  \item ``Should I include returns/refunds in revenue?''
  \item ``The data goes back 3 years. How many quarters?''
  \item ``Globex has two divisions. Combine or separate?''
\end{itemize}
\end{shadedbox}
\end{column}
\begin{column}{0.5\textwidth}
\begin{shadedbox}[title=\textbf{Progress Updates}]
\begin{itemize}\small
  \item ``Found top 5 customers...''
  \item ``Retrieved quarterly data...''
  \item ``Generating visualizations...''
\end{itemize}
\end{shadedbox}
\end{column}
\end{columns}
\vspace{0.3cm}
\begin{center}
\alert{The agent can pause the loop to interact with the user}
\end{center}
\end{frame}

\begin{frame}{Step 6: Final Report}
\begin{shadedbox}
After all tool calls complete, the LLM generates the final response.
\end{shadedbox}
\vspace{0.3cm}
\begin{columns}[T]
\begin{column}{0.55\textwidth}
\begin{shadedbox}[title=\textbf{LLM Has Seen}]
\begin{itemize}\small
  \item Original user request
  \item SQL query results (raw data)
  \item Python execution output
  \item Any user clarifications
  \item Generated charts/files
\end{itemize}
\end{shadedbox}
\end{column}
\begin{column}{0.45\textwidth}
\begin{shadedbox}[title=\textbf{LLM Generates}]
\begin{itemize}\small
  \item Written analysis
  \item Key findings
  \item Trends identified
  \item Recommendations
  \item References charts
\end{itemize}
\end{shadedbox}
\end{column}
\end{columns}
\end{frame}

\begin{frame}{The Complete Message History}
\begin{shadedbox}[title=\textbf{What the LLM Sees at Report Time}]
\begin{center}
\begin{tabular}{cl}
\# & \textbf{Message} \\
\midrule
1 & system: You are a data analyst... \\
2 & user: Analyze quarterly revenue... \\
3 & assistant: [tool call: SQL for top 5] \\
4 & tool: [results: ACME, Globex...] \\
5 & assistant: [tool call: SQL for quarterly data] \\
6 & tool: [results: Q1, Q2, Q3...] \\
7 & assistant: [tool call: Python analysis] \\
8 & tool: [output: stats, saved trends.png] \\
9 & assistant: Here is my analysis... (final report)
\end{tabular}
\end{center}
\end{shadedbox}
\end{frame}

\begin{frame}{Agent Architecture Summary}
\begin{columns}[T]
\begin{column}{0.5\textwidth}
\begin{shadedbox}[title=\textbf{Components}]
\begin{itemize}\small
  \item \textbf{LLM}: Reasoning engine
  \item \textbf{System Prompt}: Defines tools \& behavior
  \item \textbf{Tools}: SQL, Python, file I/O, etc.
  \item \textbf{Agent Loop}: Orchestrates everything
  \item \textbf{Message History}: Context for LLM
\end{itemize}
\end{shadedbox}
\end{column}
\begin{column}{0.5\textwidth}
\begin{shadedbox}[title=\textbf{Flow}]
\begin{enumerate}\small
  \item User prompt arrives
  \item LLM reasons, picks tool
  \item Agent executes tool
  \item Result added to history
  \item LLM reasons again
  \item Repeat until done
  \item Final response to user
\end{enumerate}
\end{shadedbox}
\end{column}
\end{columns}
\end{frame}

\begin{frame}{The Orchestration Layer}
\begin{shadedbox}
The agent's control logic coordinates everything: which LLM to call, which system prompt to use, and what to do with each response.
\end{shadedbox}
\vspace{0.3cm}
\begin{columns}[T]
\begin{column}{0.5\textwidth}
\begin{shadedbox}[title=\textbf{Different Tasks, Different Prompts}]
\begin{itemize}\small
  \item SQL generation $\rightarrow$ database schema prompt
  \item Python analysis $\rightarrow$ data science prompt
  \item Report writing $\rightarrow$ communication prompt
  \item Each task gets specialized instructions
\end{itemize}
\end{shadedbox}
\end{column}
\begin{column}{0.5\textwidth}
\begin{shadedbox}[title=\textbf{Different Tasks, Different LLMs}]
\begin{itemize}\small
  \item Simple classification $\rightarrow$ fast, cheap model
  \item Complex reasoning $\rightarrow$ powerful model
  \item Code generation $\rightarrow$ code-specialized model
  \item Cost and speed optimization
\end{itemize}
\end{shadedbox}
\end{column}
\end{columns}
\end{frame}

\begin{frame}{How Coordination Works}
\begin{shadedbox}
The agent combines \textbf{LLM responses} with \textbf{pre-programmed logic} to decide the next action.
\end{shadedbox}
\vspace{0.3cm}
\begin{columns}[T]
\begin{column}{0.5\textwidth}
\begin{shadedbox}[title=\textbf{LLM Decides}]
\begin{itemize}\small
  \item Which tool to call next
  \item What parameters to pass
  \item When the task is complete
  \item What to ask the user
\end{itemize}
\end{shadedbox}
\end{column}
\begin{column}{0.5\textwidth}
\begin{shadedbox}[title=\textbf{Code Decides}]
\begin{itemize}\small
  \item If tool\_call $\rightarrow$ execute tool
  \item If error $\rightarrow$ retry or report
  \item If question $\rightarrow$ prompt user
  \item If done $\rightarrow$ return response
\end{itemize}
\end{shadedbox}
\end{column}
\end{columns}
\vspace{0.3cm}
\begin{center}
\alert{The agent is part LLM intelligence, part traditional programming}
\end{center}
\end{frame}

\begin{frame}{Example: Multi-Model Routing}
\begin{shadedbox}[title=\textbf{Our Database Agent with Model Selection}]
\begin{center}
\begin{tabular}{lll}
\textbf{Task} & \textbf{Model} & \textbf{System Prompt} \\
\midrule
Parse user request & GPT-4 & Extract intent and entities \\
Generate SQL & Claude & Database schema + SQL examples \\
Validate SQL & GPT-3.5 (fast) & Check syntax only \\
Analyze results & Claude & Data analysis instructions \\
Write report & GPT-4 & Business writing style \\
\end{tabular}
\end{center}
\end{shadedbox}
\vspace{0.3cm}
\begin{baritemize}\small
  \item Each step uses the best model for that task
  \item Faster models for simple tasks save time and money
  \item Specialized prompts improve quality at each step
\end{baritemize}
\end{frame}

\begin{frame}[fragile]{Orchestration Logic (Pseudocode)}
\begin{shadedbox}[title=\textbf{The Agent's Decision Loop}]
\begin{verbatim}
while not done:
    response = call_llm(messages, current_system_prompt)

    if response.has_tool_call:
        result = execute_tool(response.tool_call)
        messages.append(tool_result)
    elif response.needs_user_input:
        answer = ask_user(response.question)
        messages.append(user_answer)
    elif response.is_final:
        done = True
        return response.content
\end{verbatim}
\end{shadedbox}
\end{frame}

\begin{frame}{Why This Matters}
\begin{shadedbox}
Agents can accomplish complex, multi-step tasks that chatbots cannot.
\end{shadedbox}
\vspace{0.3cm}
\begin{columns}[T]
\begin{column}{0.5\textwidth}
\begin{shadedbox}[title=\textbf{Chatbot Limitations}]
\begin{itemize}\small
  \item Can only suggest SQL to run
  \item Cannot see actual data
  \item Cannot execute analysis
  \item User must do all the work
\end{itemize}
\end{shadedbox}
\end{column}
\begin{column}{0.5\textwidth}
\begin{shadedbox}[title=\textbf{Agent Capabilities}]
\begin{itemize}\small
  \item Writes and executes SQL
  \item Sees and reasons about data
  \item Performs analysis end-to-end
  \item Delivers finished product
\end{itemize}
\end{shadedbox}
\end{column}
\end{columns}
\vspace{0.3cm}
\begin{center}
\alert{Claude Code is an agent---it has tools for files, code, web, and more}
\end{center}
\end{frame}

\begin{frame}{Building Your Own Agent}
\begin{shadedbox}
The key pieces you need to build an agent:
\end{shadedbox}
\vspace{0.3cm}
\begin{barenumerate}
  \item \textbf{Define tools}: What actions can the agent take?
  \item \textbf{Write system prompt}: Describe tools, schema, rules
  \item \textbf{Implement agent loop}: Send messages, parse tool calls, execute, repeat
  \item \textbf{Handle tool execution}: Safely run SQL, Python, etc.
  \item \textbf{Manage context}: Keep message history, handle token limits
\end{barenumerate}
\vspace{0.3cm}
\begin{center}
\alert{Or use an agent framework: LangChain, CrewAI, Claude Code SDK}
\end{center}
\end{frame}

\begin{frame}{Summary}
\begin{columns}[T]
\begin{column}{0.5\textwidth}
\begin{shadedbox}[title=\textbf{Key Concepts}]
\begin{itemize}\small
  \item Agent = Chatbot + Tools
  \item LLM decides which tools to use
  \item Tool results feed back to LLM
  \item Agent loop orchestrates flow
  \item Full history provides context
\end{itemize}
\end{shadedbox}
\end{column}
\begin{column}{0.5\textwidth}
\begin{shadedbox}[title=\textbf{Our Example}]
\begin{itemize}\small
  \item SQL tool for database queries
  \item Python tool for analysis
  \item Multiple tool calls in sequence
  \item Interim user communication
  \item Final report from LLM
\end{itemize}
\end{shadedbox}
\end{column}
\end{columns}
\vspace{0.5cm}
\begin{center}
\begin{shadedbox}[width=0.85\textwidth]
\centering
\alert{Agents extend LLMs from conversation to action}
\end{shadedbox}
\end{center}
\end{frame}

\end{document}
