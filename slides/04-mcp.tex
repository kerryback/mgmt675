\documentclass[aspectratio=169]{beamer}
\usetheme{metropolis}
\usepackage{appendixnumberbeamer}
\usepackage{booktabs, hyperref, graphicx}

\input{mgmt675-style}

\usepackage{tikz}
\usetikzlibrary{shapes.geometric, arrows.meta, positioning, calc}

% Title info
\subtitle{MGMT 675: Generative AI for Finance}
\title{Connecting Tools to AI}
\author{Kerry Back}
\date{}

\begin{document}

\maketitle

% ============================================================
% WHAT IS MCP?
% ============================================================

\begin{frame}{The Problem: Isolated AI}
Claude Desktop is powerful---but out of the box, it can only read and write text. It cannot control your browser, run terminal commands, or interact with other software.
\vspace{0.3cm}
\begin{itemize}
  \item Want Claude to fill out a web form? It can't reach your browser.
  \item Want Claude to run a shell command? It has no terminal access.
  \item \alert{MCP servers solve this problem.}
\end{itemize}
\end{frame}

\begin{frame}{What is MCP?}
The \textbf{Model Context Protocol (MCP)} is an open standard that lets AI applications connect to external tools and data sources through a uniform interface.
\vspace{0.3cm}
\begin{columns}[T]
\begin{column}{0.5\textwidth}
\begin{shadedbox}[title=\textbf{The Idea}]
\begin{itemize}\small
  \item A standard ``plug-in'' protocol
  \item AI app connects to MCP servers
  \item Each server provides specific tools
  \item Tools appear automatically in the AI
\end{itemize}
\end{shadedbox}
\end{column}
\begin{column}{0.5\textwidth}
\begin{shadedbox}[title=\textbf{Analogy}]
\begin{itemize}\small
  \item USB is a standard for peripherals
  \item Plug in a keyboard, it just works
  \item MCP is a standard for AI tools
  \item Connect a server, tools just appear
\end{itemize}
\end{shadedbox}
\end{column}
\end{columns}
\end{frame}

\begin{frame}{How MCP Works}
\begin{center}
\begin{tikzpicture}[
  node distance=1.2cm and 2cm,
  every node/.style={font=\small},
  sbox/.style={draw=titlegray, rounded corners, minimum width=3.2cm, minimum height=0.9cm, fill=excelinput, text=titlegray, font=\small\bfseries, align=center},
  sarrow/.style={-{Stealth[length=3mm]}, thick, draw=accentblue}
]
\node[sbox, minimum width=3.5cm] (app) {Claude Desktop\\(MCP Client)};
\node[sbox, right=2.5cm of app, yshift=1.5cm] (s1) {Browser\\Server};
\node[sbox, right=2.5cm of app] (s2) {Terminal\\Server};
\node[sbox, right=2.5cm of app, yshift=-1.5cm] (s3) {Database\\Server};

\draw[sarrow] (app) -- (s1) node[midway, above, font=\scriptsize\itshape, text=titlegray, sloped] {MCP protocol};
\draw[sarrow] (app) -- (s2);
\draw[sarrow] (app) -- (s3);

\node[right=0.5cm of s1, font=\scriptsize, text=titlegray] {Chrome, Firefox, ...};
\node[right=0.5cm of s2, font=\scriptsize, text=titlegray] {PowerShell, Bash, ...};
\node[right=0.5cm of s3, font=\scriptsize, text=titlegray] {PostgreSQL, SQLite, ...};
\end{tikzpicture}
\end{center}
\vspace{0.3cm}
\begin{itemize}
  \item Claude Desktop is the \textbf{client}; each external service runs as a \textbf{server}
  \item Servers expose \textbf{tools} (functions the AI can call) via a standard protocol
  \item You can connect multiple servers at once
\end{itemize}
\end{frame}

% ============================================================
% CONNECTING A SERVER IN CLAUDE DESKTOP
% ============================================================
\section{Connecting an MCP Server in Claude Desktop}

\begin{frame}{Two Ways to Install MCP Servers}
\begin{columns}[T]
\begin{column}{0.5\textwidth}
\begin{shadedbox}[title=\textbf{Desktop Extensions (Easy)}]
\begin{enumerate}\small
  \item Open Claude Desktop
  \item Go to \textbf{Settings $\rightarrow$ Extensions}
  \item Click \textbf{Browse extensions}
  \item Find the server you want
  \item Click \textbf{Install}
\end{enumerate}
\vspace{0.2cm}
One-click install. Anthropic-reviewed.
\end{shadedbox}
\end{column}
\begin{column}{0.5\textwidth}
\begin{shadedbox}[title=\textbf{Manual Configuration}]
\begin{enumerate}\small
  \item Install prerequisites (next slide)
  \item Edit a JSON config file
  \item Specify the server command
  \item Restart Claude Desktop
  \item Tools appear automatically
\end{enumerate}
\vspace{0.2cm}
More control. Works with any server.
\end{shadedbox}
\end{column}
\end{columns}
\end{frame}

\begin{frame}[fragile]{Prerequisites: Node.js and uv}
MCP servers are distributed as Node.js or Python packages. You need one or both installed depending on which servers you use.
\vspace{0.2cm}
\begin{columns}[T]
\begin{column}{0.5\textwidth}
\begin{shadedbox}[title=\textbf{Node.js} (for \texttt{npx} servers)]
\begin{itemize}\small
  \item Download the \textbf{LTS} installer from \url{https://nodejs.org}
  \item Run the installer (includes \texttt{npm} and \texttt{npx})
  \item Verify: open a terminal and type \texttt{node --version}
\end{itemize}
\vspace{0.1cm}
{\scriptsize Used by: DesktopCommander, FMP}
\end{shadedbox}
\end{column}
\begin{column}{0.5\textwidth}
\begin{shadedbox}[title=\textbf{uv} (for \texttt{uvx} servers)]
\begin{itemize}\small
  \item \textbf{Mac}: run in Terminal:\\{\scriptsize\texttt{brew install uv}}
  \item \textbf{Windows}: run in PowerShell:\\{\scriptsize\texttt{winget install astral-sh.uv}}
  \item Verify: \texttt{uvx --version}
\end{itemize}
\vspace{0.1cm}
{\scriptsize Used by: Browser-Use, Alpha Vantage}
\end{shadedbox}
\end{column}
\end{columns}
\vspace{0.2cm}
\begin{center}\small
\alert{Install these before configuring any MCP server.} \texttt{uv} also installs Python if you don't have it.
\end{center}
\end{frame}

\begin{frame}{Config File Location}
For manual installation, edit the Claude Desktop configuration file:
\vspace{0.3cm}
\begin{columns}[T]
\begin{column}{0.5\textwidth}
\begin{shadedbox}[title=\textbf{macOS}]
{\small\texttt{\textasciitilde/Library/Application Support/\\Claude/claude\_desktop\_config.json}}
\end{shadedbox}
\end{column}
\begin{column}{0.5\textwidth}
\begin{shadedbox}[title=\textbf{Windows}]
{\small\texttt{\%APPDATA\%\textbackslash Claude\textbackslash\\claude\_desktop\_config.json}}
\end{shadedbox}
\end{column}
\end{columns}
\vspace{0.3cm}
\begin{itemize}
  \item Create the file if it doesn't exist
  \item The file defines which MCP servers to launch when Claude Desktop starts
  \item After editing, \alert{fully quit and restart} Claude Desktop
\end{itemize}
\end{frame}

\begin{frame}[fragile]{Config File Structure}
\begin{shadedbox}[title=\textbf{Example: Adding Two MCP Servers}]
\begin{verbatim}
{
  "mcpServers": {
    "my-server-1": {
      "command": "npx",
      "args": ["-y", "some-mcp-package"]
    },
    "my-server-2": {
      "command": "uvx",
      "args": ["another-mcp-package"]
    }
  }
}
\end{verbatim}
\end{shadedbox}
\vspace{0.2cm}
\begin{itemize}
  \item Each server has a name, a command, and arguments
  \item \texttt{npx} runs Node.js packages; \texttt{uvx} runs Python packages
  \item Claude Desktop starts each server automatically on launch
\end{itemize}
\end{frame}

\begin{frame}{Verifying the Connection}
\begin{enumerate}
  \item Restart Claude Desktop after editing the config file
  \item Look for the \textbf{hammer icon} in the chat input area---click it to see available tools
  \item If a server fails to start, check the logs:
  \begin{itemize}\small
    \item macOS: \texttt{\textasciitilde/Library/Logs/Claude/mcp*.log}
    \item Windows: \texttt{\%APPDATA\%\textbackslash Claude\textbackslash logs\textbackslash mcp*.log}
  \end{itemize}
\end{enumerate}
\vspace{0.3cm}
When Claude wants to use a tool, it will ask for your \textbf{permission} before executing. You stay in control.
\end{frame}

% ============================================================
% BROWSER-USE MCP SERVER
% ============================================================
\section{Browser-Use: AI-Driven Browser Automation}

\begin{frame}{What is Browser-Use?}
\textbf{Browser-Use} is an MCP server that lets Claude autonomously control a web browser. You describe a task in plain English, and it handles the navigation, clicking, and typing.
\vspace{0.3cm}
\begin{columns}[T]
\begin{column}{0.5\textwidth}
\textbf{Example Tasks}
\begin{itemize}\small
  \item ``Fill out the application form on this page with my information''
  \item ``Download all linked CSV files from this page''
  \item ``Search for AAPL on Yahoo Finance and get the P/E ratio''
\end{itemize}
\end{column}
\begin{column}{0.5\textwidth}
\textbf{Key Features}
\begin{itemize}\small
  \item Plain-language task descriptions
  \item Navigates pages and fills forms autonomously
  \item Extracts data from web pages
\end{itemize}
\end{column}
\end{columns}
\end{frame}

\begin{frame}{How Browser-Use Works}
\begin{center}
\begin{tikzpicture}[
  node distance=1.5cm,
  every node/.style={font=\small},
  sbox/.style={draw=titlegray, rounded corners, minimum width=3cm, minimum height=0.9cm, fill=excelinput, text=titlegray, font=\small\bfseries, align=center},
  sarrow/.style={-{Stealth[length=3mm]}, thick, draw=accentblue}
]
\node[sbox] (claude) {Claude\\Desktop};
\node[sbox, right=1.5cm of claude] (mcp) {Browser-Use\\MCP Server};
\node[sbox, right=1.5cm of mcp] (browser) {Web\\Browser};
\node[sbox, right=1.5cm of browser] (web) {Website};

\draw[sarrow] (claude) -- (mcp) node[midway, above, font=\scriptsize\itshape, text=titlegray] {task};
\draw[sarrow] (mcp) -- (browser) node[midway, above, font=\scriptsize\itshape, text=titlegray] {commands};
\draw[sarrow] (browser) -- (web);
\draw[sarrow, dashed] (web) -- ++(0,-1.2) -| (claude) node[near start, below, font=\scriptsize\itshape, text=titlegray] {results returned to Claude};
\end{tikzpicture}
\end{center}
\vspace{0.3cm}
\begin{itemize}
  \item Claude sends a high-level task (e.g., ``fill out this form'')
  \item Browser-Use translates the task into browser actions (click, type, scroll)
  \item Results (extracted text, confirmation, screenshots) return to Claude
\end{itemize}
\end{frame}

\begin{frame}{Browser-Use: Two Deployment Options}
\begin{columns}[T]
\begin{column}{0.5\textwidth}
\begin{shadedbox}[title=\textbf{Cloud (Hosted)}]
\begin{itemize}\small
  \item No local setup required---browser runs on their servers
  \item \alert{10-step limit per task}; requires a Browser-Use API key
  \item Persistent login profiles available
\end{itemize}
\end{shadedbox}
\end{column}
\begin{column}{0.5\textwidth}
\begin{shadedbox}[title=\textbf{Local (Self-Hosted)}]
\begin{itemize}\small
  \item Browser runs on your machine---no step limits
  \item Requires your own LLM API key (adds cost per action)
  \item \alert{Recommended for most users}; full privacy
\end{itemize}
\end{shadedbox}
\end{column}
\end{columns}
\vspace{0.3cm}
Docs: \url{https://docs.browser-use.com/customize/integrations/mcp-server}
\end{frame}

\begin{frame}[fragile]{Installing Browser-Use}
\begin{shadedbox}[title=\textbf{Add to Claude Desktop Config}]
\begin{verbatim}
"browser-use": {
  "command": "uvx",
  "args": ["--from", "browser-use[cli]",
           "browser-use", "--mcp"],
  "env": {
    "ANTHROPIC_API_KEY": "sk-ant-..."
  }
}
\end{verbatim}
\end{shadedbox}
\vspace{0.2cm}
\begin{itemize}
  \item \textbf{Prerequisite}: \texttt{uv} installed and Chrome/Chromium on your machine
  \item The \texttt{[cli]} extra is required---without it, the MCP server won't start
  \item \texttt{ANTHROPIC\_API\_KEY}: Browser-Use calls an LLM to navigate pages, so it needs its own API key
\end{itemize}
\end{frame}

\begin{frame}{Browser-Use: Strengths and Limitations}
\begin{columns}[T]
\begin{column}{0.45\textwidth}
\begin{shadedbox}[title=\textbf{Strengths}]
\begin{itemize}\small
  \item Natural language task descriptions---no selectors or scripts
  \item Handles multi-step workflows and form filling
  \item Cross-platform (Mac, Windows, Linux)
\end{itemize}
\end{shadedbox}
\end{column}
\begin{column}{0.45\textwidth}
\begin{shadedbox}[title=\textbf{Limitations}]
\begin{itemize}\small
  \item AI-driven: can be unpredictable on complex pages
  \item Local version needs an LLM API key (adds cost per action)
  \item May struggle with heavily dynamic sites
\end{itemize}
\end{shadedbox}
\end{column}
\end{columns}
\end{frame}

% ============================================================
% DESKTOP COMMANDER MCP SERVER
% ============================================================
\section{DesktopCommanderMCP: Terminal Control}

\begin{frame}{What is DesktopCommanderMCP?}
\textbf{DesktopCommanderMCP} is an MCP server that gives Claude Desktop the ability to execute terminal commands, manage files, and control processes on your computer.
\vspace{0.3cm}
\begin{columns}[T]
\begin{column}{0.5\textwidth}
\textbf{Example Tasks}
\begin{itemize}\small
  \item ``Run \texttt{pip install pandas}''
  \item ``Show me all Python files in this folder''
  \item ``Start a Jupyter notebook server''
\end{itemize}
\end{column}
\begin{column}{0.5\textwidth}
\textbf{Key Features}
\begin{itemize}\small
  \item Execute terminal commands
  \item Read, write, and search files
  \item Manage running processes
\end{itemize}
\end{column}
\end{columns}
\end{frame}

\begin{frame}{Why Give Claude a Terminal?}
\begin{itemize}
  \item \textbf{Software installation}: ``Install the Python libraries I need for this project''
  \item \textbf{Data processing}: ``Convert all the .xlsx files in this folder to .csv''
  \item \textbf{Git operations}: ``Initialize a git repo, commit my changes, and push''
\end{itemize}
\vspace{0.3cm}
With DesktopCommander, Claude Desktop can do much more than chat---it becomes an active tool on your computer.
\end{frame}

\begin{frame}[fragile]{Installing DesktopCommanderMCP}
\begin{shadedbox}[title=\textbf{Add to Claude Desktop Config}]
\begin{verbatim}
"desktop-commander": {
  "command": "npx",
  "args": ["-y", "desktop-commander-mcp"]
}
\end{verbatim}
\end{shadedbox}
\vspace{0.2cm}
\begin{itemize}
  \item \textbf{Prerequisite}: Node.js (LTS) installed
  \item \texttt{npx -y} downloads and runs the package automatically---no API keys needed
  \item Restart Claude Desktop after editing the config
\end{itemize}
\vspace{0.2cm}
GitHub: \url{https://github.com/wonderwhy-er/DesktopCommanderMCP}
\end{frame}

\begin{frame}{DesktopCommanderMCP: Tools Available}
\begin{center}
\small
\begin{tabular}{@{} ll @{}}
\toprule
\textbf{Tool} & \textbf{Description} \\
\midrule
\texttt{execute\_command} & Run a terminal command \\
\texttt{read\_file} & Read file contents \\
\texttt{write\_file} & Create or overwrite a file \\
\texttt{edit\_block} & Make targeted edits using diff blocks \\
\texttt{search\_files} & Search file contents with regex \\
\texttt{list\_directory} & List files and folders \\
\texttt{get\_file\_info} & Get file metadata (size, dates) \\
\texttt{list\_processes} & Show running processes \\
\texttt{kill\_process} & Stop a running process \\
\bottomrule
\end{tabular}
\end{center}
\end{frame}

\begin{frame}{DesktopCommanderMCP: Strengths and Limitations}
\begin{columns}[T]
\begin{column}{0.45\textwidth}
\begin{shadedbox}[title=\textbf{Strengths}]
\begin{itemize}\small
  \item Full terminal access with file operations built in
  \item Well-established, large community
  \item Cross-platform; no API keys needed
\end{itemize}
\end{shadedbox}
\end{column}
\begin{column}{0.45\textwidth}
\begin{shadedbox}[title=\textbf{Limitations}]
\begin{itemize}\small
  \item Powerful = potentially dangerous
  \item A bad command could delete files or break your system
  \item No built-in sandboxing
\end{itemize}
\end{shadedbox}
\end{column}
\end{columns}
\vspace{0.3cm}
\alert{Always review terminal commands before clicking ``Allow''---Claude asks permission for a reason.}
\end{frame}

% ============================================================
% FINANCIAL DATA MCP SERVERS
% ============================================================
\section{Financial Data Servers}

\begin{frame}[fragile]{MCP Servers for Financial Data}
\begin{columns}[T]
\begin{column}{0.5\textwidth}
\begin{shadedbox}[title=\textbf{Alpha Vantage}]
\begin{itemize}\small
  \item 115+ tools: stock prices, options, forex, crypto, economic indicators
  \item Free API key at \url{https://www.alphavantage.co}
  \item \alert{Free tier}: 25 requests/day, 5/minute
\end{itemize}
\end{shadedbox}
\end{column}
\begin{column}{0.5\textwidth}
\begin{shadedbox}[title=\textbf{Financial Modeling Prep}]
\begin{itemize}\small
  \item 253+ tools: financial statements, fundamentals, SEC filings, earnings
  \item Free API key at \url{https://financialmodelingprep.com}
  \item \alert{Free tier}: 250 requests/day, 5 years of annual data
\end{itemize}
\end{shadedbox}
\end{column}
\end{columns}
\end{frame}

\begin{frame}[fragile]{Installing Financial Data Servers}
\begin{shadedbox}[title=\textbf{Alpha Vantage}]
\begin{verbatim}
"alphavantage": {
  "command": "uvx",
  "args": ["av-mcp", "YOUR_API_KEY"]
}
\end{verbatim}
{\small\textbf{Prerequisite}: \texttt{uv} installed. Get a free API key at \url{https://www.alphavantage.co/support/\#api-key}}
\end{shadedbox}
\vspace{0.2cm}
\begin{shadedbox}[title=\textbf{Financial Modeling Prep}]
\begin{verbatim}
"financial-modeling-prep": {
  "command": "npx",
  "args": ["-y", "@houtini/fmp-mcp"],
  "env": { "FMP_API_KEY": "YOUR_API_KEY" }
}
\end{verbatim}
{\small\textbf{Prerequisite}: Node.js 18+ installed. Get a free API key at \url{https://financialmodelingprep.com}}
\end{shadedbox}
\vspace{0.2cm}
\begin{center}\small
Replace \texttt{YOUR\_API\_KEY} with your actual key. Then restart Claude Desktop.
\end{center}
\end{frame}

% ============================================================
% GOOGLE CALENDAR MCP SERVER
% ============================================================
\section{Google Calendar: Scheduling from Chat}

\begin{frame}{What is Google Calendar MCP?}
The \textbf{Google Calendar MCP server} connects Claude to your Google Calendar, letting you manage events, check availability, and schedule meetings---all from a conversation.
\vspace{0.3cm}
\begin{columns}[T]
\begin{column}{0.5\textwidth}
\begin{shadedbox}[title=\textbf{Tools Provided}]
\begin{itemize}\small
  \item \texttt{list-events} / \texttt{search-events}
  \item \texttt{create-event} / \texttt{update-event} / \texttt{delete-event}
  \item \texttt{get-freebusy} (check availability)
\end{itemize}
\end{shadedbox}
\end{column}
\begin{column}{0.5\textwidth}
\begin{shadedbox}[title=\textbf{Example Tasks}]
\begin{itemize}\small
  \item ``What's on my calendar tomorrow?''
  \item ``Schedule a team lunch next Friday at noon''
  \item ``Am I free Tuesday afternoon?''
\end{itemize}
\end{shadedbox}
\end{column}
\end{columns}
\end{frame}

\begin{frame}[fragile]{Installing Google Calendar MCP}
\begin{shadedbox}[title=\textbf{Add to Claude Desktop Config}]
\begin{verbatim}
"google-calendar": {
  "command": "npx",
  "args": ["@cocal/google-calendar-mcp"],
  "env": {
    "GOOGLE_OAUTH_CREDENTIALS":
      "/path/to/gcp-oauth.keys.json"
  }
}
\end{verbatim}
\end{shadedbox}
\vspace{0.2cm}
\begin{itemize}
  \item \textbf{Prerequisite}: Node.js installed; a Google Cloud OAuth 2.0 credential file
  \item \textbf{Setup}: Create a project in Google Cloud Console, enable the Calendar API, create OAuth Desktop credentials, and download the JSON key file
  \item On first launch, the server opens your browser for OAuth login; supports multiple accounts
\end{itemize}
\end{frame}

\begin{frame}{The Bigger Picture: Chat as Interface}
\begin{columns}[T]
\begin{column}{0.55\textwidth}
\vspace{0.3cm}
\includegraphics[width=\textwidth]{images/ui-evolution-monitors}
\end{column}
\begin{column}{0.45\textwidth}
\begin{itemize}
  \item \textbf{Terminal}: type commands
  \item \textbf{GUI}: point and click
  \item \textbf{Chat}: describe what you want
\end{itemize}
\vspace{0.2cm}
MCP is the protocol that makes the third screen work. It connects the chat interface to \alert{everything else}---your calendar, browser, terminal, databases, and APIs.
\end{column}
\end{columns}
\end{frame}

% ============================================================
% PUTTING IT ALL TOGETHER
% ============================================================
\section{Putting It All Together}

\begin{frame}{Combining Multiple MCP Servers}
You can connect multiple MCP servers at once. Claude sees all their tools and can use them together in a single conversation.
\vspace{0.3cm}
\begin{columns}[T]
\begin{column}{0.5\textwidth}
\begin{shadedbox}[title=\textbf{Example Combination}]
\begin{itemize}\small
  \item \textbf{Browser-Use}: navigate to a website and download data
  \item \textbf{DesktopCommander}: run a Python script to process the data
  \item \textbf{Result}: end-to-end automation with no manual steps
\end{itemize}
\end{shadedbox}
\end{column}
\begin{column}{0.5\textwidth}
\begin{shadedbox}[title=\textbf{Other Popular Servers}]
\begin{itemize}\small
  \item \textbf{Filesystem / GitHub}: file access, pull requests, issues
  \item \textbf{PostgreSQL / SQLite}: database queries
  \item Browse all: \url{https://github.com/modelcontextprotocol/servers}
\end{itemize}
\end{shadedbox}
\end{column}
\end{columns}
\end{frame}

\begin{frame}[fragile]{Example: Multiple Servers in One Config}
\begin{shadedbox}[title=\textbf{All entries go inside \texttt{"mcpServers": \{ \ldots\ \}}}]
\begin{verbatim}
{
  "mcpServers": {
    "browser-use": {
      "command": "uvx",
      "args": ["--from", "browser-use[cli]",
               "browser-use", "--mcp"],
      "env": {"ANTHROPIC_API_KEY": "sk-ant-..."}
    },
    "desktop-commander": {
      "command": "npx",
      "args": ["-y", "desktop-commander-mcp"]
    },
    "alphavantage": {
      "command": "uvx",
      "args": ["av-mcp", "YOUR_AV_KEY"]
    }
  }
}
\end{verbatim}
\end{shadedbox}
\vspace{0.1cm}
\begin{center}\small
Add as many servers as you like. All start when Claude Desktop launches.
\end{center}
\end{frame}

\begin{frame}{Troubleshooting MCP Servers}
MCP servers can be finicky to set up. If tools don't appear after restarting Claude Desktop, work through this checklist:
\vspace{0.2cm}
\begin{enumerate}
  \item \textbf{Validate your JSON}: a missing comma or extra comma will silently break the config. Paste your config into \url{https://jsonlint.com} to check.
  \item \textbf{Fully quit Claude Desktop}: on Mac, \texttt{Cmd+Q}; on Windows, right-click the system tray icon and \textbf{Quit}. Simply closing the window is not enough.
  \item \textbf{Check the logs}: Mac: \texttt{\textasciitilde/Library/Logs/Claude/mcp*.log}; Windows: \texttt{\%APPDATA\%\textbackslash Claude\textbackslash logs\textbackslash mcp*.log}
\end{enumerate}
\end{frame}

\begin{frame}{Looking Ahead: Built-In Alternatives to MCP}
DesktopCommander and Browser-Use add terminal and browser control to Claude Desktop's \textbf{Chat} mode. But Claude Desktop also has other modes---available on paid plans---that have these capabilities built in. We will cover them later in the course.
\vspace{0.2cm}
\begin{center}
\small
\begin{tabular}{@{} l c c c @{}}
\toprule
& \textbf{Chat} & \textbf{Code tab} & \textbf{Cowork tab} \\
\midrule
Terminal / files & \alert{MCP needed} & Built in & Built in \\
Browser control & \alert{MCP needed} & Chrome ext & Chrome ext \\
Available on & All plans & Pro+ & Pro+ \\
\bottomrule
\end{tabular}
\end{center}
\vspace{0.2cm}
\begin{itemize}
  \item \textbf{Code tab} and \textbf{Cowork tab}: built-in terminal, file access, and sandboxed agents
  \item \textbf{Claude in Chrome}: a browser extension---built-in alternative to Browser-Use MCP
  \item MCP servers for \textbf{data sources} (Alpha Vantage, FMP, databases) remain valuable on any plan
\end{itemize}
\end{frame}

\begin{frame}{Summary}
\begin{itemize}
  \item \textbf{MCP} is a standard protocol that connects AI apps to external tools via plug-in servers
  \item Servers covered: \textbf{Browser-Use} (web automation), \textbf{DesktopCommander} (terminal), \textbf{Alpha Vantage / FMP} (financial data), \textbf{Google Calendar} (scheduling)
  \item Combine multiple servers for end-to-end workflows
\end{itemize}
\vspace{0.3cm}
\alert{MCP turns Claude from a conversation partner into a tool that takes action on your computer.}
\end{frame}

\end{document}
