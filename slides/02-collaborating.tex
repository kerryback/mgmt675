\documentclass[aspectratio=169]{beamer}
\usetheme{metropolis}
\usepackage{appendixnumberbeamer}
\usepackage{booktabs, hyperref}

\input{mgmt675-style}

\subtitle{MGMT 675: Generative AI for Finance}
\title{Collaborating with AI}
\author{Kerry Back}

\date{}

\begin{document}

\maketitle

\begin{frame}{The Mindset Shift}
\begin{baritemize}
  \item AI is not a search engine---it's a \alert{collaborator}
  \item Do not try to craft the ``perfect prompt''
  \item Instead: have a \alert{conversation}
  \item Think of AI as a capable colleague, not a vending machine
  \item The goal is \textit{iterative refinement}, not one-shot perfection
\end{baritemize}
\end{frame}

\begin{frame}{Conversation Over Perfect Prompts}
\begin{columns}[T]
  \begin{column}{0.45\textwidth}
\begin{shadedbox}[title=\textbf{Don't Do This}]
\begin{itemize}
  \item Spend 30 minutes crafting one prompt
  \item Try to anticipate every edge case
  \item Give up when the first response is wrong
  \item Treat AI as a one-shot tool
\end{itemize}
\end{shadedbox}
\end{column}
\begin{column}{0.45\textwidth}
\begin{shadedbox}[title=\textbf{Do This Instead}]
\begin{itemize}
  \item Start with a rough request
  \item Refine based on the response
  \item Ask follow-up questions
  \item Iterate until you get what you need
\end{itemize}
\end{shadedbox}
\end{column}
\end{columns}
\end{frame}

\begin{frame}{Ask the AI What It Can Do}
\begin{baritemize}
  \item AI systems know their own capabilities
  \item Ask: ``What information do you need from me to do this?''
  \item Ask: ``What are the different ways you could approach this?''
  \item Ask: ``What should I consider before we start?''
  \item Let the AI \alert{guide the conversation}---it often knows what questions to ask
\end{baritemize}

\vspace{0.5cm}
\begin{shadedbox}
\centerline{\textit{``I want to build a portfolio optimizer. What do you need to know to help me?''}}
\end{shadedbox}
\end{frame}

\begin{frame}{Explain What, Not How}
\begin{baritemize}
  \item Describe your \alert{goal}, not the implementation steps
  \item Let the AI suggest the approach
  \item You provide domain expertise; AI provides technical execution
  \item Focus on outcomes: what does success look like?
\end{baritemize}

\vspace{0.3cm}
\begin{columns}[T]
  \begin{column}{0.45\textwidth}
\begin{shadedbox}[title=\textbf{Too Prescriptive}]
\small
``Write a Python function that uses scipy.optimize.minimize with SLSQP to find portfolio weights...''
\end{shadedbox}
\end{column}
\begin{column}{0.45\textwidth}
\begin{shadedbox}[title=\textbf{Better}]
\small
``I want to find the portfolio that maximizes the Sharpe ratio. Here are my expected returns and covariance matrix...''
\end{shadedbox}
\end{column}
\end{columns}
\end{frame}

\begin{frame}{The Plan-Execute-Evaluate Cycle}
\begin{center}
\begin{tikzpicture}[scale=1.2]
  % Nodes
  \node[draw, rounded corners, fill=accentblue!20, minimum width=2.5cm, minimum height=1cm] (plan) at (0,2) {\textbf{Plan}};
  \node[draw, rounded corners, fill=accentblue!20, minimum width=2.5cm, minimum height=1cm] (execute) at (4,2) {\textbf{Execute}};
  \node[draw, rounded corners, fill=accentblue!20, minimum width=2.5cm, minimum height=1cm] (evaluate) at (2,0) {\textbf{Evaluate}};

  % Arrows
  \draw[->, thick, accentblue] (plan) -- (execute);
  \draw[->, thick, accentblue] (execute) -- (evaluate);
  \draw[->, thick, accentblue] (evaluate) -- (plan);
\end{tikzpicture}
\end{center}

\begin{baritemize}
  \item \textbf{Plan:} Define the task, gather requirements, outline approach
  \item \textbf{Execute:} Let AI do the work with your guidance
  \item \textbf{Evaluate:} Check results, identify gaps, refine
  \item Repeat until satisfied
\end{baritemize}
\end{frame}

\begin{frame}{Phase 1: Planning}
\begin{baritemize}
  \item Don't dive straight into execution
  \item First, \alert{brainstorm with the AI}
  \item Ask it to help you define requirements
  \item Request a step-by-step plan before starting work
  \item Identify edge cases and potential problems upfront
\end{baritemize}

\vspace{0.3cm}
\begin{shadedbox}
\textit{``Before we start coding, can you outline a plan? What are the key steps and what could go wrong?''}
\end{shadedbox}
\end{frame}

\begin{frame}{Phase 2: Execution}
\begin{baritemize}
  \item Work through the plan step by step
  \item Provide feedback after each step
  \item If something doesn't look right, say so immediately
  \item Share context: constraints, preferences, examples
  \item Let the AI iterate---errors are opportunities to improve
\end{baritemize}

\vspace{0.3cm}
\begin{shadedbox}
\textit{``That looks good, but we also need to handle the case where returns are negative. Can you update it?''}
\end{shadedbox}
\end{frame}

\begin{frame}{Phase 3: Evaluation}
\begin{baritemize}
  \item Don't assume the output is correct
  \item \alert{Test} the results against known cases
  \item Ask the AI to explain its reasoning
  \item Check for edge cases and assumptions
  \item Consider: does this actually solve my problem?
\end{baritemize}

\vspace{0.3cm}
\begin{shadedbox}
\textit{``Can you walk me through how you calculated these weights? What assumptions did you make?''}
\end{shadedbox}
\end{frame}

\begin{frame}{Cross-Evaluation with Other AI Conversations}
\begin{baritemize}
  \item Start a \alert{new conversation} to evaluate the output
  \item Even with the \textit{same model}, a fresh context can catch errors
  \item Ask the new conversation to review, critique, or verify
  \item Different phrasing may reveal blind spots
  \item Consider using a \textit{different model} for additional perspective
\end{baritemize}

\vspace{0.3cm}
\begin{shadedbox}
\textit{``I received this analysis from another session. Can you review it for errors or questionable assumptions?''}
\end{shadedbox}
\end{frame}

\begin{frame}{Why Cross-Evaluation Works}
\begin{columns}[T]
  \begin{column}{0.48\textwidth}
\begin{shadedbox}[title=\textbf{Same Model, Fresh Context}]
\begin{itemize}
  \item No accumulated assumptions
  \item No confirmation bias from earlier turns
  \item Approaches the problem fresh
  \item May spot logical gaps
\end{itemize}
\end{shadedbox}
\end{column}
\begin{column}{0.48\textwidth}
\begin{shadedbox}[title=\textbf{Different Model}]
\begin{itemize}
  \item Different training emphasis
  \item Different reasoning patterns
  \item May catch model-specific blind spots
  \item Provides true second opinion
\end{itemize}
\end{shadedbox}
\end{column}
\end{columns}
\end{frame}


\begin{frame}{Provide Context}
\begin{baritemize}
  \item AI works better with \alert{context about you and your situation}
  \item Share your role, goals, and constraints
  \item Explain why you're doing this, not just what
  \item Context guides the AI toward relevant knowledge
  \item More context = more tailored responses
\end{baritemize}

\vspace{0.3cm}
\begin{shadedbox}
\textit{``I'm a finance student working on a case competition. We need to present our portfolio recommendation to a panel of judges who are industry professionals...''}
\end{shadedbox}
\end{frame}

\begin{frame}{Learn Through Experimentation}
\begin{baritemize}
  \item There's no substitute for \alert{hands-on practice}
  \item Experiment with different approaches
  \item Push the boundaries---see what works and what doesn't
   \item Each domain (coding, writing, analysis) has different patterns
\end{baritemize}

\vspace{0.3cm}
\begin{shadedbox}
\centerline{The best prompters are those who have experimented the most}
\end{shadedbox}
\end{frame}

\begin{frame}{Resources}
\begin{baritemize}
  \item \href{https://www.oneusefulthing.org/p/working-with-ai-two-paths-to-prompting}{Ethan Mollick: Working with AI - Two Paths to Prompting}
  \item \href{https://mitsloanedtech.mit.edu/ai/basics/effective-prompts/}{MIT Sloan: Effective Prompts for AI}
  \item \href{https://www.addventures.com/en/stories/five-tips-for-collaborating-with-ai-preparing-for-2025}{Five Tips for Collaborating with AI}
  \item \href{https://www.controlaltachieve.com/2024/05/ai-prompting-collaborative-prompt.html}{Collaborative Prompt Technique}
  \item \href{https://www.news.aakashg.com/p/prompt-engineering}{Prompt Engineering in 2025: Latest Best Practices}
\end{baritemize}
\end{frame}

\end{document}
