\documentclass[aspectratio=169]{beamer}
\usetheme{metropolis}
\usepackage{appendixnumberbeamer}
\usepackage{booktabs}

\input{mgmt675-style}

\subtitle{MGMT 675: Generative AI for Finance}
\title{VS Code with Claude Code}
\author{Kerry Back}

\date{}
\titlegraphic{\includegraphics[width=4cm]{images/slide1_img1.png}}

\begin{document}

\maketitle

\begin{frame}{What is VS Code?}
\begin{baritemize}
  \item Visual Studio Code: a free code editor from Microsoft
  \item Works on Windows, Mac, and Linux
  \item Lightweight but powerful
  \item Huge ecosystem of extensions
  \item We'll use it as a UI for \alert{Claude Code}
\end{baritemize}
\end{frame}

\begin{frame}{Why VS Code + Claude Code?}
\begin{columns}[T]
  \begin{column}{0.5\textwidth}
\begin{shadedbox}[title=\textbf{Colab}]
\begin{itemize}
  \item Browser-based
  \item No installation
  \item Google Drive storage
  \item Google Gemini AI
\end{itemize}
\end{shadedbox}
\end{column}
\begin{column}{0.5\textwidth}
\begin{shadedbox}[title=\textbf{VS Code}]
\begin{itemize}
  \item Desktop application
  \item Local file access
  \item Claude Code AI
  \item More powerful tools
\end{itemize}
\end{shadedbox}
\end{column}
\end{columns}
\vspace{0.5cm}
\centering Both support Jupyter notebooks!
\end{frame}

\begin{frame}{What You Need}
\begin{baritemize}
  \item VS Code installed (free from \alert{code.visualstudio.com})
  \item Claude Code extension (free to install)
  \item Anthropic account for Claude API access
  \item Python extension (for notebooks)
  \item Jupyter extension (for notebooks)
\end{baritemize}
\end{frame}

\begin{frame}{Installing the Claude Code Extension}
\begin{barenumerate}
  \item Open VS Code
  \item Press Ctrl+Shift+X (Windows) or Cmd+Shift+X (Mac)
  \item Search for ``Claude Code''
  \item Click \textbf{Install} on the official Anthropic extension
  \item Sign in when prompted
\end{barenumerate}
\end{frame}

\begin{frame}{The Claude Code Interface}
\begin{center}
\includegraphics[width=0.85\textwidth]{images/vscode_claude_interface.png}
\end{center}
\end{frame}

\begin{frame}{Opening Claude Code}
\begin{baritemize}
  \item \textbf{Spark icon}: Click the spark icon in the top-right corner of any open file
  \item \textbf{Status bar}: Click ``Claude Code'' in the bottom-right corner
  \item \textbf{Command Palette}: Ctrl+Shift+P $\rightarrow$ type ``Claude Code''
  \item \textbf{Keyboard shortcut}: Cmd+Esc (Mac) / Ctrl+Esc (Windows)
\end{baritemize}
\end{frame}

\begin{frame}{How Claude Code Works}
\begin{center}
\includegraphics[width=0.85\textwidth]{images/vscode_claude_workflow.png}
\end{center}
\end{frame}

\begin{frame}{Chatting with Claude}
\begin{baritemize}
  \item Type your question or request in the prompt box
  \item Press Enter to send
  \item Claude can see your selected code automatically
  \item Use \texttt{@filename} to reference specific files
  \item Claude asks permission before making changes
\end{baritemize}
\end{frame}

\begin{frame}{What Claude Code Can Do}
\begin{baritemize}
  \item Explain code and answer questions
  \item Write new code from descriptions
  \item Fix errors and debug problems
  \item Edit files (with your approval)
  \item Run commands in the terminal
  \item Create and modify Jupyter notebooks
\end{baritemize}
\end{frame}

\begin{frame}{Reviewing Changes}
\begin{baritemize}
  \item Claude shows changes in a side-by-side diff view
  \item Green = additions, Red = deletions
  \item You can \alert{Accept} or \alert{Reject} each change
  \item Or tell Claude what to do differently
  \item Changes are not applied until you approve them
\end{baritemize}
\end{frame}

\begin{frame}{Jupyter Notebooks in VS Code}
\vspace{0.5cm}
\begin{center}
\begin{shadedbox}[width=0.7\textwidth]
\centering\large\textbf{Same concept as Colab, local execution}
\end{shadedbox}
\end{center}
\vspace{0.5cm}
\begin{baritemize}
  \item Code cells and text cells
  \item Run cells with Shift+Enter
  \item Output appears below each cell
  \item No browser or internet required
\end{baritemize}
\end{frame}

\begin{frame}{Creating a New Notebook}
\begin{barenumerate}
  \item Ctrl+Shift+P (or Cmd+Shift+P on Mac)
  \item Type ``Create: New Jupyter Notebook''
  \item Select a Python kernel from the top-right picker
  \item Start coding!
\end{barenumerate}
\vspace{0.3cm}
Or simply create a new file with \texttt{.ipynb} extension
\end{frame}

\begin{frame}{Opening Existing Notebooks}
\begin{baritemize}
  \item File $\rightarrow$ Open File $\rightarrow$ select \texttt{.ipynb} file
  \item Drag and drop notebook file into VS Code
  \item Double-click notebook in the Explorer sidebar
  \item Download from Colab and open locally
\end{baritemize}
\end{frame}

\begin{frame}{The VS Code Notebook Interface}
\begin{center}
\includegraphics[width=0.85\textwidth]{images/vscode_notebook.png}
\end{center}
\end{frame}

\begin{frame}{Running Cells}
\begin{baritemize}
  \item \textbf{Shift+Enter}: Run cell and move to next
  \item \textbf{Ctrl+Enter}: Run cell and stay
  \item \textbf{Alt+Enter}: Run cell and insert new cell below
  \item Click the play button ($\triangleright$) left of the cell
  \item ``Run All'' button in toolbar runs entire notebook
\end{baritemize}
\end{frame}

\begin{frame}{Notebook Keyboard Shortcuts}
\begin{columns}[T]
  \begin{column}{0.5\textwidth}
\begin{shadedbox}[title=\textbf{Running Code}]
\begin{itemize}
  \item Shift+Enter: Run cell
  \item Ctrl+Enter: Run, stay
  \item Alt+Enter: Run, insert below
\end{itemize}
\end{shadedbox}
\end{column}
\begin{column}{0.5\textwidth}
\begin{shadedbox}[title=\textbf{Managing Cells}]
\begin{itemize}
  \item A: Add cell above
  \item B: Add cell below
  \item DD: Delete cell
\end{itemize}
\end{shadedbox}
\end{column}
\end{columns}
\vspace{0.3cm}
\centering Press \textbf{Esc} to enter command mode, \textbf{Enter} to edit
\end{frame}

\begin{frame}{Selecting a Kernel}
\begin{baritemize}
  \item The kernel runs your code (like Colab's runtime)
  \item Click the kernel picker in the top-right corner
  \item Select ``Python Environments'' $\rightarrow$ choose your Python
  \item Kernel must be selected before running code
\end{baritemize}
\end{frame}

\begin{frame}{Using Claude with Notebooks}
\begin{baritemize}
  \item Ask Claude to create a notebook for you
  \item Claude can add, edit, or delete cells
  \item Select code and ask Claude to explain it
  \item Request data visualizations or analysis
  \item Claude can fix errors in your notebook code
\end{baritemize}
\end{frame}

\begin{frame}{Key Differences from Colab}
\begin{columns}[T]
  \begin{column}{0.5\textwidth}
\begin{shadedbox}[title=\textbf{Colab}]
\begin{itemize}
  \item Browser-based
  \item Gemini AI built-in
  \item Auto-saves to Drive
  \item Free GPU access
\end{itemize}
\end{shadedbox}
\end{column}
\begin{column}{0.5\textwidth}
\begin{shadedbox}[title=\textbf{VS Code}]
\begin{itemize}
  \item Desktop app
  \item Claude Code AI
  \item Local file storage
  \item Uses your computer
\end{itemize}
\end{shadedbox}
\end{column}
\end{columns}
\end{frame}

\begin{frame}{Quick Reference}
\begin{baritemize}
  \item \textbf{Open Claude}: Click spark icon or Ctrl+Esc
  \item \textbf{New notebook}: Ctrl+Shift+P $\rightarrow$ ``New Jupyter Notebook''
  \item \textbf{Run cell}: Shift+Enter
  \item \textbf{Select kernel}: Click picker in top-right
  \item \textbf{Reference file}: Type \texttt{@filename} in Claude prompt
\end{baritemize}
\end{frame}

\end{document}
