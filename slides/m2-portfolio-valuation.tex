\documentclass[aspectratio=169]{beamer}
\usetheme{metropolis}
\usepackage{appendixnumberbeamer}
\usepackage{booktabs, hyperref}

\input{mgmt675-style}

\usepackage{tikz}
\usetikzlibrary{shapes.geometric, arrows.meta, positioning, calc}

% Title info
\subtitle{MGMT 675: Generative AI for Finance}
\title{Module 2: Portfolio Optimization and Company Valuation}
\author{Kerry Back}
\date{}

\begin{document}

\maketitle

\begin{frame}{Learning Objectives}
\begin{enumerate}
  \item Construct optimal portfolios with real-world constraints using AI
  \item Build two-stage DCF models driven by a small set of assumptions
  \item Perform one-way and two-way sensitivity analysis
  \item Run Monte Carlo simulations to quantify valuation uncertainty
\end{enumerate}

\vspace{0.5cm}
These are core tools of investment management and corporate finance.  AI handles the mathematics and code.  \alert{You focus on the assumptions and the interpretation.}
\end{frame}

% ============================================================
% PART 1: MEAN-VARIANCE ANALYSIS
% ============================================================
\section{Portfolio Optimization}

\begin{frame}{The Problem}
Given expected returns, risks (standard deviations), and correlations for multiple assets, plus a risk-free rate:

\begin{itemize}
  \item What is the \textbf{best portfolio}?
  \item What does the set of all possible portfolios look like?
  \item How do real-world constraints change the answer?
\end{itemize}

\vspace{0.3cm}
Mean-variance analysis provides the framework.  AI handles the computation.
\end{frame}

\begin{frame}{Key Concepts}
\begin{columns}[T]
\begin{column}{0.48\textwidth}
\begin{shadedbox}[title=\textbf{Portfolios}]
\begin{itemize}\small
  \item \textbf{Tangency portfolio:} the risky portfolio with the highest Sharpe ratio (excess return per unit of risk)
  \item \textbf{Global minimum variance (GMV):} the risky portfolio with the lowest possible risk
  \item \textbf{Efficient frontier:} the set of risky portfolios with the highest return for each level of risk
\end{itemize}
\end{shadedbox}
\end{column}
\begin{column}{0.48\textwidth}
\begin{shadedbox}[title=\textbf{Capital Allocation Line}]
\begin{itemize}\small
  \item The straight line from the risk-free rate through the tangency portfolio
  \item Represents all combinations of the risk-free asset and the tangency portfolio
  \item The best attainable risk-return tradeoff
\end{itemize}
\end{shadedbox}
\end{column}
\end{columns}
\end{frame}

\begin{frame}{Visualizing the Opportunity Set}
\begin{center}
\begin{tikzpicture}[scale=0.7]
  % Axes
  \draw[->, thick] (0,0) -- (10,0) node[right] {Standard Deviation};
  \draw[->, thick] (0,0) -- (0,6.5) node[above] {Expected Return};

  % Portfolio cloud (scatter dots)
  \foreach \i in {1,...,60}{
    \pgfmathsetmacro{\x}{1.5 + 6*rnd}
    \pgfmathsetmacro{\y}{0.5 + 5*rnd}
    \pgfmathsetmacro{\valid}{\y < 0.4*\x + 2.5 ? 1 : 0}
    \ifnum\valid=1
      \fill[accentblue!20] (\x,\y) circle (1.5pt);
    \fi
  }

  % Efficient frontier (curve)
  \draw[very thick, alertorange, domain=1.8:8, samples=50]
    plot (\x, {0.7*sqrt(\x-1) + 1.5});

  % CAL (line from risk-free through tangency)
  \draw[very thick, accentblue, dashed] (0,1.2) -- (9, 5.7);

  % Risk-free rate
  \fill[titlegray] (0,1.2) circle (3pt) node[left, font=\small] {$r_f$};

  % Tangency portfolio
  \fill[alertorange] (4.5,4.0) circle (4pt) node[above right, font=\small\bfseries] {Tangency};

  % GMV
  \fill[titlegray] (1.8,2.3) circle (4pt) node[below right, font=\small\bfseries] {GMV};
\end{tikzpicture}
\end{center}

\small The \textcolor{accentblue!50}{cloud} shows random portfolios.  The \textcolor{alertorange}{orange curve} is the efficient frontier.  The \textcolor{accentblue}{dashed line} is the capital allocation line.
\end{frame}

\begin{frame}{Adding Real-World Constraints}
\begin{columns}[T]
\begin{column}{0.48\textwidth}
\textbf{Common constraints}
\begin{itemize}
  \item No short sales (all weights $\geq 0$)
  \item Maximum position (e.g., $\leq 40\%$ per asset)
  \item Minimum position (e.g., $\geq 2\%$ if held)
  \item Margin requirement (sum of absolute values $\leq 2$)
\end{itemize}
\end{column}
\begin{column}{0.48\textwidth}
\textbf{What happens}
\begin{itemize}
  \item The efficient frontier shifts \alert{inward} (lower returns and/or higher risk)
  \item The tangency portfolio changes
  \item The Sharpe ratio cannot improve with constraints---it can only stay the same or get worse
  \item The constrained frontier is still useful: it reflects what you can actually implement
\end{itemize}
\end{column}
\end{columns}
\end{frame}

\begin{frame}{Solution Methods}
\begin{itemize}
  \item \textbf{Solver (numerical optimization)}
    \begin{itemize}
      \item Maximize Sharpe ratio (tangency portfolio)
      \item Minimize risk for a target return (efficient frontier)
      \item Minimize risk (global minimum variance portfolio)
    \end{itemize}
  \item \textbf{Analytic/algebraic solution}
    \begin{itemize}
      \item Solve systems of linear equations for tangency, GMV, and frontier
      \item Available only when there are no constraints
    \end{itemize}
\end{itemize}

\vspace{0.3cm}
Python solver options: \texttt{scipy.minimize}, \texttt{cvxopt}, \texttt{cvxpy}---AI chooses and writes the code.

\vspace{0.3cm}
\begin{shadedbox}[title={Ask Claude}]
\small Discuss the advantages and disadvantages of these solver options for mean-variance analysis with inequality constraints.
\end{shadedbox}
\end{frame}

\begin{frame}[shrink=5]{Exercise: Portfolio Cloud}

Consider three assets with the following characteristics:

\vspace{0.2cm}
\begin{center}\small
\begin{tabular}{@{}l c c c@{}}
\toprule
& \textbf{Asset A} & \textbf{Asset B} & \textbf{Asset C} \\
\midrule
Expected return & 8\% & 12\% & 15\% \\
Standard deviation & 14\% & 20\% & 26\% \\
\bottomrule
\end{tabular}
\end{center}

\vspace{0.1cm}
\begin{center}\small
Correlations: $\rho_{AB} = 0.3$, \quad $\rho_{AC} = 0.1$, \quad $\rho_{BC} = 0.5$
\end{center}

\vspace{0.1cm}
Assume a risk-free rate of 4\%.  Ask Claude to:
\begin{enumerate}
  \item Generate 10{,}000 random portfolios (random weights summing to 1, no short sales) and plot each portfolio's expected return vs.\ standard deviation.
  \item Overlay the efficient frontier and the capital allocation line on the same plot.
  \item Mark the tangency portfolio and the global minimum variance portfolio.
\end{enumerate}
\end{frame}

% ============================================================
% PART 2: DCF VALUATION
% ============================================================
\section{Company Valuation: DCF}

\begin{frame}{The Big Picture}
Discounted cash flow (DCF) valuation estimates the value of a company's operations by forecasting future free cash flows and discounting them to the present.

\vspace{0.3cm}
\begin{itemize}
  \item A small set of \textbf{assumptions} drives the entire model
  \item AI builds the pro forma financial statements and computes free cash flow
  \item You focus on \alert{choosing the right assumptions} and \alert{interpreting the output}
\end{itemize}
\end{frame}

\begin{frame}{The Assumption Set}
A pro forma model is driven by a small number of assumptions.  Everything else is computed from these.
\vspace{0.2cm}
\begin{center}
\small
\begin{tabular}{@{} l l @{}}
\toprule
\textbf{Assumption} & \textbf{Driver} \\
\midrule
Sales growth rate & \% per year \\
COGS & \% of sales \\
SG\&A & Base amount + \% of sales \\
Net working capital (NWC) & \% of sales \\
PP\&E & \% of sales \\
Net other operating assets & \% of sales \\
Depreciation & \% of PP\&E \\
Tax rate & \% of pre-tax income \\
\bottomrule
\end{tabular}
\end{center}
\vspace{0.2cm}
Capital expenditures are not assumed directly.  \alert{Cap ex = target PP\&E $-$ prior PP\&E + depreciation}.
\end{frame}

\begin{frame}{Pro Forma Income Statement}
\begin{center}
\small
\begin{tabular}{@{} l l @{}}
\toprule
\textbf{Line Item} & \textbf{How It's Computed} \\
\midrule
Sales & Prior sales $\times$ (1 + growth rate) \\
COGS & COGS\% $\times$ sales \\
\textbf{Gross profit} & Sales $-$ COGS \\
SG\&A & Base + SG\&A\% $\times$ sales \\
Depreciation & Depr\% $\times$ PP\&E \\
\textbf{EBIT} & Gross profit $-$ SG\&A $-$ depreciation \\
Taxes & Tax rate $\times$ EBIT \\
\textbf{NOPAT} & EBIT $-$ taxes \\
\bottomrule
\end{tabular}
\end{center}
\vspace{0.3cm}
We use \textbf{NOPAT} (net operating profit after tax) rather than net income because DCF values the \textit{operations} of the firm, independent of how they are financed.
\end{frame}

\begin{frame}{Pro Forma Balance Sheet and Cap Ex}
\begin{columns}[T]
\begin{column}{0.48\textwidth}
\begin{shadedbox}[title=\textbf{Balance Sheet Items}]
\begin{itemize}\small
  \item \textbf{NWC} = NWC\% $\times$ sales
  \item \textbf{PP\&E} = PP\&E\% $\times$ sales
  \item \textbf{Net other OA} = Net other OA\% $\times$ sales
  \item Change in each = current $-$ prior year
\end{itemize}
\end{shadedbox}
\end{column}
\begin{column}{0.48\textwidth}
\begin{shadedbox}[title=\textbf{Capital Expenditures}]
\begin{itemize}\small
  \item Target PP\&E$_t$ = PP\&E\% $\times$ sales$_t$
  \item Depreciation$_t$ = depr\% $\times$ PP\&E$_{t-1}$
  \item \alert{Cap ex$_t$ = PP\&E$_t$ $-$ PP\&E$_{t-1}$ + depr$_t$}
\end{itemize}
\end{shadedbox}
\end{column}
\end{columns}
\vspace{0.3cm}
\textbf{Free Cash Flow to the Firm}
\begin{center}
\textbf{FCF = NOPAT + Depreciation $-$ Cap Ex $-$ $\Delta$NWC $-$ $\Delta$Net Other OA}
\end{center}
\vspace{0.2cm}
Equivalently: FCF = NOPAT $-$ net investment in PP\&E $-$ net investment in NWC $-$ net investment in other operating assets.
\end{frame}

% ============================================================
% TWO-STAGE DCF
% ============================================================
\section{Two-Stage DCF}

\begin{frame}{Two-Stage DCF: Overview}
\begin{center}
\begin{tikzpicture}[
  node distance=0.8cm and 2cm,
  every node/.style={font=\small},
  sbox/.style={draw=titlegray, rounded corners, minimum width=3.5cm, minimum height=0.9cm, fill=excelinput, text=titlegray, font=\small\bfseries, align=center},
  sarrow/.style={-{Stealth[length=3mm]}, thick, draw=accentblue}
]
\node[sbox] (s1) {Stage 1\\Explicit Forecast\\(e.g., 5--10 years)};
\node[sbox, right=2cm of s1] (s2) {Stage 2\\Terminal Value\\(perpetuity)};
\node[sbox, below=1.2cm of $(s1)!0.5!(s2)$] (ev) {Enterprise Value\\= PV(Stage 1) + PV(Stage 2)};

\draw[sarrow] (s1) -- (ev);
\draw[sarrow] (s2) -- (ev);
\end{tikzpicture}
\end{center}
\vspace{0.2cm}
\begin{itemize}
  \item \textbf{Stage 1}: Project FCF each year from pro forma assumptions
  \item \textbf{Stage 2}: Terminal value = FCF$_{T+1}$ / (WACC $-$ $g$), where $g$ is the long-run growth rate
  \item \textbf{Equity value} = Enterprise value $-$ net debt (discount at the WACC)
\end{itemize}
\end{frame}

\begin{frame}{Terminal Value}
The terminal value typically accounts for \alert{60--80\%} of total enterprise value.  Getting it right matters more than the explicit forecast.
\vspace{0.3cm}
\begin{columns}[T]
\begin{column}{0.48\textwidth}
\begin{shadedbox}[title=\textbf{Growing Perpetuity}]
\begin{itemize}\small
  \item TV = FCF$_{T+1}$ / (WACC $-$ $g$)
  \item FCF$_{T+1}$ = year $T$ FCF $\times$ (1 + $g$)
  \item $g$ = long-run nominal growth (typically 2--3\%)
  \item Requires $g <$ WACC
\end{itemize}
\end{shadedbox}
\end{column}
\begin{column}{0.48\textwidth}
\begin{shadedbox}[title=\textbf{Exit Multiple}]
\begin{itemize}\small
  \item TV = EBITDA$_T$ $\times$ exit multiple
  \item Multiple from comparable firms
  \item Common in practice (M\&A, PE)
  \item Cross-check against perpetuity method
\end{itemize}
\end{shadedbox}
\end{column}
\end{columns}
\end{frame}

% ============================================================
% SENSITIVITY ANALYSIS
% ============================================================
\section{Sensitivity Analysis}

\begin{frame}{Sensitivity Tables}
A sensitivity table shows how the output (e.g., equity value per share) changes as you vary one or two key inputs.
\vspace{0.3cm}
\begin{columns}[T]
\begin{column}{0.48\textwidth}
\begin{shadedbox}[title=\textbf{One-Way Table}]
\begin{itemize}\small
  \item Vary a single input (e.g., WACC from 8\% to 12\%)
  \item Hold everything else constant
  \item Shows which inputs matter most
\end{itemize}
\end{shadedbox}
\end{column}
\begin{column}{0.48\textwidth}
\begin{shadedbox}[title=\textbf{Two-Way Table}]
\begin{itemize}\small
  \item Vary two inputs simultaneously
  \item Classic: WACC vs.\ terminal growth rate
  \item Also useful: sales growth vs.\ COGS margin
  \item Reveals interaction effects
\end{itemize}
\end{shadedbox}
\end{column}
\end{columns}
\vspace{0.3cm}
Ask Claude to produce sensitivity tables in Excel (with formulas) or as formatted output.  In Excel, the Data Table feature automates two-way tables.
\end{frame}

% ============================================================
% MONTE CARLO
% ============================================================
\section{Monte Carlo Simulation}

\begin{frame}{Why Simulate?}
\begin{itemize}
  \item Sensitivity tables vary one or two inputs; in reality, \alert{all assumptions are uncertain simultaneously}
  \item Monte Carlo simulation draws random values for each assumption, computes FCF and enterprise value, and repeats thousands of times
  \item The result: a \textbf{distribution} of enterprise values, not a single point estimate
\end{itemize}
\end{frame}

\begin{frame}{Simulation Setup}
Assign a probability distribution to each uncertain assumption, then sample and compute.
\vspace{0.2cm}
\begin{center}
\small
\begin{tabular}{@{} l l l @{}}
\toprule
\textbf{Assumption} & \textbf{Distribution} & \textbf{Example} \\
\midrule
Sales growth & Normal & $\mu = 8\%$, $\sigma = 3\%$ \\
COGS \% & Normal & $\mu = 60\%$, $\sigma = 2\%$ \\
Terminal growth & Uniform & 1.5\% to 3.5\% \\
WACC & Normal & $\mu = 10\%$, $\sigma = 1\%$ \\
\bottomrule
\end{tabular}
\end{center}
\vspace{0.2cm}
\begin{itemize}
  \item Run 10{,}000 simulations $\rightarrow$ 10{,}000 enterprise values
  \item Report mean, median, 10th/90th percentiles; plot a histogram
  \item Identify which assumptions drive the most variance (tornado chart)
\end{itemize}
\end{frame}

% ============================================================
% BUILDING WITH AI
% ============================================================
\section{Building These Analyses with AI}

\begin{frame}{Multiple Ways to Build with AI}
\begin{center}
\small
\begin{tabular}{@{} l p{9.5cm} @{}}
\toprule
\textbf{Platform} & \textbf{What to Ask} \\
\midrule
Chat & ``Build a two-stage DCF model for a company with these assumptions \ldots\ Generate an Excel file with pro formas, FCF, and a sensitivity table.'' \\
Cowork & Point Claude at a folder with data files. ``Read the financial data, build pro formas, run a DCF, and save the results to Excel.'' \\
Code & ``Build a DCF model in Python.  Fetch Apple's financials from FMP, estimate assumptions from historical data, and run a Monte Carlo simulation.'' \\
Excel add-in & Open a blank workbook.  ``Build a two-stage DCF with pro forma statements, a sensitivity table for WACC vs.\ terminal growth, and a tornado chart.'' \\
\bottomrule
\end{tabular}
\end{center}
\vspace{0.2cm}
\alert{Key difference:} Chat and Cowork generate spreadsheets via Python (no internet).  Code mode can \alert{fetch live data} from APIs before building the model.
\end{frame}

\begin{frame}{}
\vspace{2cm}
\begin{center}
{\Large\bfseries The same analysis that takes hours with traditional tools takes minutes with AI.}

\vspace{1cm}
Describing the problem clearly---choosing the right assumptions, interpreting the output---is the hard part.  The computation is delegated entirely.
\end{center}
\end{frame}

% ============================================================
% EXERCISES
% ============================================================
\section{Exercises}

\begin{frame}{Exercise 1: Tangency Portfolio}
\begin{itemize}
  \item Download data for 5 ETFs + T-bill rate
  \item Ask AI to compute the tangency portfolio: portfolio weights, expected return, standard deviation, and Sharpe ratio
  \item Plot the efficient frontier and capital allocation line
  \item Add a no-short-sales constraint and compare: how do the weights, Sharpe ratio, and frontier change?
  \item Submit the data, plots, and a short paragraph on how constraints changed the result
\end{itemize}
\end{frame}

\begin{frame}{Exercise 2: Portfolio Cloud Artifact}
Using the three-asset data from earlier:
\begin{center}\small
\begin{tabular}{@{}l c c c@{}}
\toprule
& \textbf{Asset A} & \textbf{Asset B} & \textbf{Asset C} \\
\midrule
Expected return & 8\% & 12\% & 15\% \\
Standard deviation & 14\% & 20\% & 26\% \\
\bottomrule
\end{tabular}
\end{center}
\begin{center}\small
Correlations: $\rho_{AB} = 0.3$, \quad $\rho_{AC} = 0.1$, \quad $\rho_{BC} = 0.5$; \quad $r_f = 4\%$
\end{center}
\vspace{0.2cm}
Ask Claude to build an \textbf{interactive artifact} that generates 10{,}000 random portfolios, overlays the efficient frontier, and marks the tangency and GMV portfolios.  Publish and submit the link.
\end{frame}

\begin{frame}{Exercise 3: DCF Model}
Build a two-stage DCF for a hypothetical company:
\begin{itemize}\small
  \item Current sales: \$500M; sales growth: 10\% (years 1--5), 3\% terminal
  \item COGS: 58\% of sales; SG\&A: \$20M + 12\% of sales
  \item NWC: 15\% of sales; PP\&E: 40\% of sales; net other OA: 5\% of sales; depreciation: 10\% of PP\&E
  \item Tax rate: 25\%; WACC: 9\%; net debt: \$200M; shares: 50M
\end{itemize}
\vspace{0.2cm}
Produce an Excel workbook with pro forma income statement, balance sheet items, FCF calculation, enterprise value, per-share value, and a two-way sensitivity table (WACC vs.\ terminal growth rate).
\end{frame}

\begin{frame}{Exercise 4: DCF with Live Data}
\begin{itemize}
  \item In Code mode, ask Claude to fetch a real company's financials from Financial Modeling Prep (or Alpha Vantage)
  \item Estimate each assumption from historical patterns (trailing averages)
  \item Build a 5-year pro forma and run a two-stage DCF
  \item Compare your AI-generated valuation to the current market cap
  \item Submit the code output + a short memo explaining the comparison
\end{itemize}

\vspace{0.3cm}
\textbf{Note:} This exercise requires Code mode (internet access for the API call).
\end{frame}

\begin{frame}{Exercise 5: Monte Carlo Simulation}
\begin{itemize}
  \item Take your DCF model (from exercise 3 or 4)
  \item Assign distributions to 4 key assumptions:
  \begin{itemize}\small
    \item Sales growth, COGS \%, terminal growth, WACC
    \item Choose normal or uniform distributions with reasonable parameters
  \end{itemize}
  \item Run 10{,}000 simulations
  \item Report: mean, median, 10th and 90th percentiles
  \item Plot: histogram of enterprise values + tornado chart
  \item Discuss: which assumption drives the most variation?
\end{itemize}
\end{frame}

\end{document}
