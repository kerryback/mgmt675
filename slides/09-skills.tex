\documentclass[aspectratio=169]{beamer}
\usetheme{metropolis}
\usepackage{appendixnumberbeamer}
\usepackage{booktabs, hyperref}

\input{mgmt675-style}

\subtitle{MGMT 675: Generative AI for Finance}
\title{Customizing AI with Prompts}
\author{Kerry Back}

\date{}

\begin{document}

\maketitle

\begin{frame}{The Big Idea: It's All Just Text}
Every AI customization---skills, slash commands, custom instructions---works the same way: \alert{additional text is added to the prompt} that the model sees.
\vspace{0.3cm}
\begin{itemize}
  \item \textbf{Skills} = instructions for \textit{how} (a system prompt extension loaded when relevant)
  \item \textbf{Slash commands} = triggers that say \textit{do} (typing \texttt{/name} loads that skill's instructions into context)
  \item Same pattern everywhere: Custom GPTs, Copilot instructions, \texttt{.cursorrules}, Gems
\end{itemize}
\vspace{0.3cm}
This deck uses Claude's implementation as the example, but the concepts are universal.  Any platform that lets you add instructions to the prompt is doing the same thing.
\end{frame}

\begin{frame}{What is a Skill?}
A \textbf{skill} is a set of instructions (and optional code) that specializes a general-purpose AI for a specific domain or task.  In Claude, it's a folder with a markdown file; other platforms use similar mechanisms.
\vspace{0.5cm}
\begin{columns}[T]
\begin{column}{0.5\textwidth}
\begin{shadedbox}[title=\textbf{A Skill Provides}]
\begin{itemize}\small
  \item System prompt (domain knowledge)
  \item Workflow instructions
  \item Best practices and constraints
\end{itemize}
\end{shadedbox}
\end{column}
\begin{column}{0.5\textwidth}
\begin{shadedbox}[title=\textbf{The AI Platform Provides}]
\begin{itemize}\small
  \item The LLM (Opus or Sonnet)
  \item Agent control loop and tools
  \item Code execution sandbox
\end{itemize}
\end{shadedbox}
\end{column}
\end{columns}
\end{frame}

\begin{frame}[fragile]{Anatomy of a Skill}
\begin{columns}[T]
\begin{column}{0.45\textwidth}
\textbf{Skill Folder Structure}\small
\begin{verbatim}
skills/
  xlsx/
    SKILL.md     <- Main file
    scripts/
      recalc.py
    references/
      schema.md
\end{verbatim}
\end{column}
\begin{column}{0.55\textwidth}
\textbf{SKILL.md Structure}\small
\begin{verbatim}
---
name: xlsx
description: "Excel file..."
---

# Requirements for Outputs
- Zero formula errors
- Use formulas, not hardcodes

# Workflows
1. Choose pandas or openpyxl
2. Create/modify file
3. Recalculate formulas
\end{verbatim}
\end{column}
\end{columns}
\end{frame}

\begin{frame}{SKILL.md: The System Prompt}
\begin{itemize}
  \item \textbf{Frontmatter}: Name and description (YAML header)
  \item \textbf{Requirements}: Quality standards and constraints
  \item \textbf{Workflows}: Step-by-step procedures
\end{itemize}
\vspace{0.3cm}
The SKILL.md file is automatically injected into Claude's context when working in a project that contains the skill.
\end{frame}

\begin{frame}{Scripts: The Custom Tools}
\begin{columns}
\begin{column}{0.5\textwidth}
\begin{itemize}\small
  \item Python or JavaScript files in \texttt{scripts/} folder
  \item Claude Code can execute them
  \item Handle tasks LLM can't do directly
\end{itemize}
\end{column}
\begin{column}{0.5\textwidth}
\textbf{Example Scripts}
\begin{itemize}\small
  \item \texttt{recalc.py} -- Recalculate Excel formulas via LibreOffice
  \item \texttt{validate.py} -- Check file structure
  \item \texttt{html2pptx.js} -- Convert HTML to PowerPoint
\end{itemize}
\end{column}
\end{columns}
\end{frame}

\begin{frame}{Example: Rice Database Skill}
\begin{columns}
\begin{column}{0.5\textwidth}
\begin{shadedbox}[title=\textbf{Stand-Alone App}]
\begin{itemize}\small
  \item Custom web application with fixed agent logic
  \item Hard-coded SQL generation prompt
  \item Requires developer to update
\end{itemize}
\end{shadedbox}
\end{column}
\begin{column}{0.5\textwidth}
\begin{shadedbox}[title=\textbf{Claude Code + Skill}]
\begin{itemize}\small
  \item SKILL.md with database schema and table descriptions
  \item \alert{Plus}: Can also make charts, Excel files, Word reports
  \item User can extend easily
\end{itemize}
\end{shadedbox}
\end{column}
\end{columns}
\vspace{0.3cm}
\begin{center}
\alert{Same database access, but infinitely more flexible}
\end{center}
\end{frame}

\begin{frame}{Skill vs Stand-Alone Agent}
\begin{center}
\begin{tabular}{lcc}
\toprule
\textbf{Feature} & \textbf{Stand-Alone} & \textbf{Skill} \\
\midrule
Agent logic & Custom code & Claude Code \\
System prompt & Hard-coded & SKILL.md file \\
LLM & Your choice & Claude Opus/Sonnet \\
Tools & Custom built & Scripts + built-in \\
Maintenance & Developer & Edit markdown \\
Combine tasks & No & Yes \\
\bottomrule
\end{tabular}
\end{center}
\vspace{0.5cm}
Skills let you create \alert{specialized agents} without writing agent logic. Claude Code handles the control flow; you just provide the domain knowledge.
\end{frame}

\begin{frame}{Where Skills Live}
\begin{itemize}
  \item \textbf{Project skills}: \texttt{.claude/skills/} in your project folder
  \item \textbf{User skills}: \texttt{\textasciitilde/.claude/skills/} (shared across projects)
  \item Skills also work in \textbf{Claude.ai}, \textbf{Chat}, and \textbf{Cowork} (details later)
\end{itemize}
\vspace{0.3cm}
\textbf{Community Skills Repository:} \href{https://github.com/VoltAgent/awesome-agent-skills}{github.com/VoltAgent/awesome-agent-skills} --- 200+ skills from official teams and the community
\end{frame}

\begin{frame}{Creating Your Own Skill}
\begin{enumerate}
  \item Create folder: \texttt{.claude/skills/my-skill/}
  \item Create \texttt{SKILL.md} with:
  \begin{itemize}
    \item YAML frontmatter (name, description)
    \item Domain knowledge and instructions
    \item Code examples and workflows
  \end{itemize}
  \item Optionally add \texttt{scripts/} folder with helper code
  \item Start using Claude Code -- skill is automatically loaded
\end{enumerate}
\end{frame}

\begin{frame}[fragile,shrink=10]{Example: Earnings Call Skill}
\begin{columns}[T]
\begin{column}{0.45\textwidth}
\textbf{SKILL.md}\scriptsize
\begin{verbatim}
---
name: earnings-call
description: "Analyze earnings
  call transcripts"
---

# Output Format
- Key metrics vs. consensus
- Management guidance summary
- Analyst Q&A highlights
- Sentiment assessment
- Three actionable takeaways

# Workflows
1. Parse transcript
2. Extract financial figures
3. Summarize guidance changes
4. Write report in Word format
\end{verbatim}
\end{column}
\begin{column}{0.55\textwidth}
\textbf{What This Gives You}
\begin{itemize}\scriptsize
  \item Type: ``Analyze the Tesla Q4 earnings call''
  \item Claude follows the structure automatically
  \item Produces Word report with key highlights
\end{itemize}
\vspace{0.2cm}
\textbf{Other Finance Skill Ideas}
\begin{itemize}\scriptsize
  \item \textbf{Investment memo}: DCF, comps, risk factors, recommendation
  \item \textbf{Portfolio report}: Monthly performance, attribution, charts
  \item \textbf{Loan analysis}: Parse credit agreements, extract covenants
\end{itemize}
\end{column}
\end{columns}
\end{frame}

\begin{frame}[fragile,shrink=5]{In-Class Exercise: Personal Task List Skill}
Create a skill that manages a personal task list organized by categories.  Claude will add, complete, and review tasks---following \textit{your} format automatically.
\vspace{0.1cm}
\begin{columns}[T]
\begin{column}{0.45\textwidth}
\textbf{SKILL.md (Sketch)}\scriptsize
\begin{verbatim}
---
name: tasks
description: "Manage my personal
  task list"
---

# File Layout
- research/tasks.md
- teaching/tasks.md
- admin/tasks.md
- personal/tasks.md

# Task Format
- [ ] [#tag] Description
      — added YYYY-MM-DD
- [x] [#tag] Done task
      — done YYYY-MM-DD

# Behaviors
- "add task" -> append to file
- "what's pending" -> scan all
- "weekly review" -> summarize
\end{verbatim}
\end{column}
\begin{column}{0.55\textwidth}
\textbf{Try It}
\begin{enumerate}\scriptsize
  \item Create \texttt{.claude/skills/tasks/SKILL.md}
  \item Define your categories (work, personal, school, \ldots)
  \item Define your format (checkboxes, tags, dates)
  \item Start Claude Code and say: ``Add `buy groceries' to my personal tasks''
  \item Then: ``What's pending?''
\end{enumerate}
\vspace{0.2cm}
\textbf{Why This Works}
\begin{itemize}\scriptsize
  \item Claude creates the folder and files for you
  \item Follows your format every time---no drift
  \item Natural language in, structured markdown out
  \item \alert{You defined the system by writing a prompt}
\end{itemize}
\end{column}
\end{columns}
\end{frame}

\begin{frame}{Skills Across Claude Formats}
Skills are not limited to Claude Code. The same skill can be deployed and used across Claude.ai, Claude Desktop, and Claude Code.
\vspace{0.1cm}
\begin{center}
\small
\begin{tabular}{@{} l l l @{}}
\toprule
\textbf{Format} & \textbf{How to Install} & \textbf{How to Invoke} \\
\midrule
Claude.ai (web) & Upload ZIP in Settings $\rightarrow$ Capabilities & Automatic or \texttt{/name} \\
Chat (Desktop) & Same as Claude.ai (shared account) & Automatic or \texttt{/name} \\
Cowork (Desktop) & Install as a plugin via sidebar & Automatic or \texttt{/name} \\
Code tab (Desktop) & Place in \texttt{.claude/skills/} & Automatic or \texttt{/name} \\
Claude Code CLI & Place in \texttt{.claude/skills/} & Automatic or \texttt{/name} \\
VS Code extension & Place in \texttt{.claude/skills/} & Automatic or \texttt{/name} \\
\bottomrule
\end{tabular}
\end{center}
\vspace{0.1cm}
\begin{itemize}
  \item \textbf{Automatic}: Claude loads the skill when it detects a relevant task
  \item \textbf{\texttt{/name}}: You invoke it explicitly (e.g., \texttt{/investment-memo})
  \item Skills in \texttt{\textasciitilde/.claude/skills/} are shared across all your projects
\end{itemize}
\end{frame}

\begin{frame}{Deploying Skills: Three Methods}
\begin{columns}[T]
\begin{column}{0.33\textwidth}
\begin{shadedbox}[title=\textbf{Claude.ai / Chat}]
\begin{enumerate}\scriptsize
  \item ZIP your skill folder
  \item Go to \textbf{Settings} $\rightarrow$ \textbf{Capabilities} $\rightarrow$ \textbf{Upload skill}
  \item Toggle it \textbf{ON}
\end{enumerate}
\vspace{0.1cm}
{\scriptsize Pro, Max, Team, Enterprise}
\end{shadedbox}
\end{column}
\begin{column}{0.33\textwidth}
\begin{shadedbox}[title=\textbf{Cowork}]
\begin{enumerate}\scriptsize
  \item Click \textbf{Plugins} in sidebar
  \item Browse official plugins or click \textbf{Upload plugin}
  \item Plugins bundle skills + connectors + slash commands
\end{enumerate}
\vspace{0.1cm}
{\scriptsize Pro, Max, Team, Enterprise}
\end{shadedbox}
\end{column}
\begin{column}{0.33\textwidth}
\begin{shadedbox}[title=\textbf{Code / CLI / VS Code}]
\begin{enumerate}\scriptsize
  \item Create folder:\newline\texttt{.claude/skills/name/}
  \item Add \texttt{SKILL.md}
  \item Done---auto-detected
\end{enumerate}
\vspace{0.35cm}
{\scriptsize Pro, Max (via Claude Code)}
\end{shadedbox}
\end{column}
\end{columns}
\end{frame}

\begin{frame}[shrink=2]{Using Skills: Examples Across Formats}
\begin{columns}[T]
\begin{column}{0.5\textwidth}
\textbf{Claude.ai or Chat}
\begin{itemize}\scriptsize
  \item ``Write an investment memo for Tesla''
  \item Claude loads the skill automatically
  \item Produces downloadable Word doc
\end{itemize}
\vspace{0.2cm}
\textbf{Claude Code / VS Code}
\begin{itemize}\scriptsize
  \item Type \texttt{/investment-memo} to invoke explicitly
  \item Creates files directly in your project folder
  \item Can combine with other skills (e.g., xlsx)
\end{itemize}
\end{column}
\begin{column}{0.5\textwidth}
\textbf{Cowork}
\begin{itemize}\scriptsize
  \item Drag a folder of 10-K filings into Cowork
  \item ``Analyze these and produce an investment memo for each company''
  \item Claude works through them autonomously
\end{itemize}
\vspace{0.2cm}
\textbf{Key Difference}
\begin{itemize}\scriptsize
  \item \textbf{Claude.ai / Chat}: one-off tasks; download results
  \item \textbf{Cowork}: batch tasks on local files
  \item \textbf{Code / VS Code}: see and edit the code
\end{itemize}
\end{column}
\end{columns}
\end{frame}

\begin{frame}{Skills and Slash Commands}
Typing \texttt{/name} simply loads that skill's instructions into the AI's context---it's just a prompt.  Every skill has a \textbf{name} that becomes a \textbf{slash command}.
\vspace{0.3cm}
\begin{columns}[T]
\begin{column}{0.5\textwidth}
\begin{shadedbox}[title=\textbf{How It Works}]
\begin{itemize}\small
  \item A skill's name is set in the YAML frontmatter of \texttt{SKILL.md}
  \item Type \texttt{/name} to invoke it explicitly
  \item Or just describe your task---Claude loads the skill automatically when it detects a match
\end{itemize}
\end{shadedbox}
\end{column}
\begin{column}{0.5\textwidth}
\begin{shadedbox}[title=\textbf{Examples}]
\begin{itemize}\small
  \item \texttt{/xlsx} --- invoke the Excel skill
  \item \texttt{/earnings-call} --- invoke an earnings call analysis skill
  \item \texttt{/investment-memo} --- invoke a memo-writing skill
\end{itemize}
\vspace{0.2cm}
{\scriptsize These appear alongside built-in commands like \texttt{/clear} and \texttt{/compact} when you type \texttt{/}.}
\end{shadedbox}
\end{column}
\end{columns}
\end{frame}

\begin{frame}{Skills vs.\ Subagents}
A \textbf{skill} is a reference manual Claude follows in your conversation. A \textbf{subagent} is a separate worker you dispatch independently.
\vspace{0.1cm}
\begin{center}
\scriptsize
\begin{tabular}{@{} l l l @{}}
\toprule
& \textbf{Skill} & \textbf{Subagent} \\
\midrule
What it is & Instructions + scripts & A separate agent \\
Analogy & Giving Claude a playbook & Handing a task to a colleague \\
Runs where & Same conversation & Its own context window \\
Works in & Claude.ai, Chat, Cowork, Code & Claude Code only \\
Defined in & \texttt{SKILL.md} & \texttt{AGENT.md} \\
Created with & Manual or \texttt{/agents} & \texttt{/agents} \\
Best for & Standards, templates, workflows & Delegating well-defined tasks \\
\bottomrule
\end{tabular}
\end{center}
\vspace{0.1cm}
\begin{itemize}\scriptsize
  \item \textbf{Use a skill} when you want Claude to follow specific standards \textit{while you work with it interactively}
  \item \textbf{Use a subagent} when you want to \textit{hand off} a self-contained task (e.g., ``analyze these 10 files'')
  \item They combine naturally: a subagent can have skills loaded, giving it both independence and domain expertise
\end{itemize}
\end{frame}

\begin{frame}{What is a Plugin?}
A \textbf{plugin} is a \textit{packaged collection} of skills, agents, hooks, and MCP servers that can be shared across teams and projects.  Think of it as the distribution format for skills.
\vspace{0.3cm}
\begin{columns}[T]
\begin{column}{0.5\textwidth}
\textbf{A Plugin Can Include}
\begin{itemize}\small
  \item Skills (\texttt{SKILL.md} files)
  \item Agents (\texttt{AGENT.md} files)
  \item MCP servers (external tools)
\end{itemize}
\end{column}
\begin{column}{0.5\textwidth}
\textbf{Why Use Plugins?}
\begin{itemize}\small
  \item Share skills with your team
  \item Namespace prevents conflicts
  \item Install from marketplaces
\end{itemize}
\end{column}
\end{columns}
\end{frame}

\begin{frame}[fragile]{Anatomy of a Plugin}
\begin{columns}[T]
\begin{column}{0.45\textwidth}
\textbf{Directory Structure}\small
\begin{verbatim}
my-plugin/
  .claude-plugin/
    plugin.json    <- Manifest
  skills/
    code-review/
      SKILL.md
    documentation/
      SKILL.md
  agents/
    reviewer.md
  hooks/
    hooks.json
  .mcp.json
\end{verbatim}
\end{column}
\begin{column}{0.55\textwidth}
\textbf{plugin.json (Required)}\small
\begin{verbatim}
{
  "name": "my-plugin",
  "description": "What it does",
  "version": "1.0.0",
  "author": {
    "name": "Your Name"
  },
  "repository":
    "github.com/user/plugin",
  "license": "MIT"
}
\end{verbatim}
\end{column}
\end{columns}
\end{frame}

\begin{frame}{Skills vs.\ Plugins}
\begin{center}
\small
\begin{tabular}{@{} l l l @{}}
\toprule
\textbf{Aspect} & \textbf{Standalone Skill} & \textbf{Plugin} \\
\midrule
Best for & Personal workflows & Team / community sharing \\
Invocation & \texttt{/skill-name} & \texttt{/plugin:skill-name} \\
Location & \texttt{.claude/skills/} & Own directory with manifest \\
Components & Skills only & Skills + agents + hooks + MCP \\
Versioning & None built-in & Semantic versioning \\
Distribution & Copy the folder & Marketplace or \texttt{/plugin install} \\
\bottomrule
\end{tabular}
\end{center}
\vspace{0.3cm}
\textbf{Practical workflow:} Start with a standalone skill for quick prototyping.  When you're ready to share, wrap it in a plugin by adding a \texttt{.claude-plugin/plugin.json} manifest.
\end{frame}

\begin{frame}[fragile,shrink=5]{Installing and Managing Plugins}
\begin{columns}[T]
\begin{column}{0.5\textwidth}
\begin{shadedbox}[title=\textbf{CLI Commands}]\small
\begin{verbatim}
# Install from marketplace
/plugin install my-plugin

# Install from local dir
claude --plugin-dir ./my-plugin

# List installed plugins
/plugin list

# Update a plugin
/plugin update my-plugin

# Remove a plugin
/plugin uninstall my-plugin
\end{verbatim}
\end{shadedbox}
\end{column}
\begin{column}{0.5\textwidth}
\begin{shadedbox}[title=\textbf{Namespacing}]
\begin{itemize}\small
  \item Plugin name becomes a namespace prefix
  \item Prevents skill name collisions
  \item Example: plugin \texttt{finance} with skill \texttt{memo} becomes \texttt{/finance:memo}
\end{itemize}
\end{shadedbox}
\vspace{0.1cm}
\begin{shadedbox}[title=\textbf{Sources}]
\begin{itemize}\small
  \item Anthropic official marketplace
  \item GitHub repositories
  \item Local directories (for testing)
\end{itemize}
\end{shadedbox}
\end{column}
\end{columns}
\end{frame}

\begin{frame}{Plugin Ecosystem}
The plugin ecosystem is growing rapidly.  Plugins are available for code quality, external integrations, development workflows, and domain-specific tasks.
\vspace{0.3cm}
\begin{columns}[T]
\begin{column}{0.5\textwidth}
\textbf{Example Plugin Categories}
\begin{itemize}\small
  \item Code review and linting
  \item Database connectors (MCP)
  \item API integrations (Slack, Jira, \ldots)
\end{itemize}
\end{column}
\begin{column}{0.5\textwidth}
\textbf{Resources}
\begin{itemize}\small
  \item \href{https://code.claude.com/docs/en/plugins.md}{Plugins documentation}
  \item \href{https://code.claude.com/docs/en/plugin-marketplaces.md}{Plugin marketplaces}
  \item \href{https://agentskills.io}{Agent Skills Standard}
\end{itemize}
\end{column}
\end{columns}
\end{frame}

\begin{frame}[fragile,shrink=5]{Example: The Finance Plugin}
Anthropic's official \textbf{finance plugin} bundles skills, MCP connectors, and slash commands for accounting and FP\&A workflows---a concrete example of a plugin in the wild.
\vspace{0.1cm}
\begin{columns}[T]
\begin{column}{0.45\textwidth}
\textbf{Plugin Structure}\scriptsize
\begin{verbatim}
finance/
  .claude-plugin/
    plugin.json
  .mcp.json        <- Snowflake,
                      BigQuery, ...
  commands/
    journal-entry/
    reconciliation/
    income-statement/
    variance-analysis/
    sox-testing/
  skills/
    accounting/
      SKILL.md
\end{verbatim}
\end{column}
\begin{column}{0.55\textwidth}
\textbf{Slash Commands}
\begin{itemize}\scriptsize
  \item \texttt{/finance:journal-entry} --- accruals, fixed assets, payroll with debits/credits
  \item \texttt{/finance:reconciliation} --- compare GL to bank or subledger
  \item \texttt{/finance:variance-analysis} --- decompose budget vs.\ actual gaps
\end{itemize}
\vspace{0.2cm}
\textbf{Data Connectors (MCP)}
\begin{itemize}\scriptsize
  \item Snowflake, Databricks, BigQuery
  \item Microsoft 365, Slack
  \item Or just upload a spreadsheet
\end{itemize}
\end{column}
\end{columns}
\end{frame}

\begin{frame}{What is Variance Analysis?}
\textbf{Variance analysis} compares \alert{budgeted} (planned) figures to \alert{actual} results and decomposes the differences into actionable drivers.  It is the core analytical task in FP\&A (Financial Planning \& Analysis).
\vspace{0.3cm}
\begin{columns}[T]
\begin{column}{0.5\textwidth}
\textbf{Why It Matters}
\begin{itemize}\small
  \item Every public company does it quarterly
  \item Boards and investors ask ``why did we miss?''
  \item Drives re-forecasting and capital allocation
\end{itemize}
\end{column}
\begin{column}{0.5\textwidth}
\textbf{Key Decompositions}
\begin{itemize}\small
  \item \textbf{Revenue}: volume vs.\ price vs.\ mix
  \item \textbf{COGS}: volume vs.\ unit cost
  \item \textbf{SG\&A}: headcount vs.\ rate vs.\ discretionary
\end{itemize}
\end{column}
\end{columns}
\vspace{0.3cm}
Revenue variance = \alert{Volume effect} + \alert{Price effect} + \alert{Mix effect}
\end{frame}

\begin{frame}[shrink=5]{Variance Analysis: Example Data}
\textbf{Scenario:} You are an FP\&A analyst at a consumer products company.  Q1 actuals just closed. The CEO wants to know why operating income missed budget by \$1.3M.
\vspace{0.2cm}
\begin{center}
\scriptsize
\begin{tabular}{@{} l r r r r @{}}
\toprule
& \textbf{Budget} & \textbf{Actual} & \textbf{Variance} & \textbf{\$ Var} \\
\midrule
\textbf{Revenue} &&&&\\
\quad Units sold & 100,000 & 95,000 & $-$5,000 & \\
\quad Avg price & \$50.00 & \$51.00 & +\$1.00 & \\
\quad \textit{Total Revenue} & \textit{\$5,000,000} & \textit{\$4,845,000} & & \textit{(\$155,000)} \\
\midrule
\textbf{Cost of Goods Sold} &&&&\\
\quad Unit cost & \$30.00 & \$32.00 & +\$2.00 & \\
\quad \textit{Total COGS} & \textit{\$3,000,000} & \textit{\$3,040,000} & & \textit{(\$40,000)} \\
\midrule
\textit{Gross Profit} & \textit{\$2,000,000} & \textit{\$1,805,000} & & \textit{(\$195,000)} \\
\midrule
\textbf{SG\&A} &&&&\\
\quad Salaries (50 FTEs @ \$15K/mo) & \$750,000 & \$780,000 & & (\$30,000) \\
\quad Marketing & \$500,000 & \$575,000 & & (\$75,000) \\
\quad Rent \& facilities & \$200,000 & \$200,000 & & --- \\
\quad \textit{Total SG\&A} & \textit{\$1,450,000} & \textit{\$1,555,000} & & \textit{(\$105,000)} \\
\midrule
\textbf{Operating Income} & \textbf{\$550,000} & \textbf{\$250,000} & & \textbf{(\$300,000)} \\
\bottomrule
\end{tabular}
\end{center}
\vspace{0.1cm}
{\scriptsize Unfavorable variances shown in parentheses.  Data is simplified for illustration.}
\end{frame}

\begin{frame}[shrink=10]{Decomposing the Variances}
\begin{columns}[T]
\begin{column}{0.5\textwidth}
\textbf{Revenue Variance: $-$\$155K}
\begin{itemize}\scriptsize
  \item \textbf{Volume effect}: $-$5,000 units $\times$ \$50 budget price = \alert{$-$\$250,000}
  \item \textbf{Price effect}: +\$1.00 $\times$ 95,000 actual units = \alert{+\$95,000}
  \item Total: $-$\$250K + \$95K = $-$\$155K \checkmark
\end{itemize}
\vspace{0.1cm}
{\scriptsize \textit{We sold fewer units but at a higher price.  Volume shortfall dominated.}}
\vspace{0.2cm}

\textbf{COGS Variance: $-$\$40K}
\begin{itemize}\scriptsize
  \item \textbf{Volume effect}: $-$5,000 units $\times$ \$30 = \alert{+\$150,000} (favorable---fewer units)
  \item \textbf{Rate effect}: +\$2.00 $\times$ 95,000 units = \alert{$-$\$190,000} (unfavorable)
  \item Total: +\$150K $-$ \$190K = $-$\$40K \checkmark
\end{itemize}
\vspace{0.1cm}
{\scriptsize \textit{Lower volume saved money, but unit costs rose \$2---likely supply chain issues.}}
\end{column}
\begin{column}{0.5\textwidth}
\textbf{SG\&A Variance: $-$\$105K}
\begin{itemize}\scriptsize
  \item \textbf{Salaries}: $-$\$30K (raises? overtime?)
  \item \textbf{Marketing}: $-$\$75K (unplanned campaign?)
  \item \textbf{Rent}: \$0 (fixed cost, on budget)
\end{itemize}
\vspace{0.1cm}
{\scriptsize \textit{Marketing overspend is the largest driver---was there an approved campaign change?}}
\vspace{0.2cm}

\textbf{Operating Income Bridge}
\begin{itemize}\scriptsize
  \item Budget OI: \$550,000
  \item Revenue volume: $-$\$250,000
  \item Revenue price: +\$95,000
  \item COGS volume: +\$150,000
  \item COGS rate: $-$\$190,000
  \item SG\&A: $-$\$105,000
  \item \textbf{Actual OI: \$250,000}
\end{itemize}
\end{column}
\end{columns}
\end{frame}

\begin{frame}[shrink=10]{Using the Plugin: \texttt{/finance:variance-analysis}}
With the finance plugin installed, you type one command and provide the data:
\vspace{0.2cm}

\textbf{Prompt}\\
\texttt{/finance:variance-analysis}\\[0.3em]
\small ``Here is our Q1 budget vs.\ actuals spreadsheet. Decompose the \$300K operating income miss into volume, price, cost, and discretionary spending drivers.  Produce a waterfall chart and a summary memo for the CFO.''
\vspace{0.2cm}

\textbf{What the Plugin Does}
\begin{enumerate}\small
  \item \textbf{Skill activates}: domain knowledge about variance decomposition formulas
  \item \textbf{Decomposes} revenue into volume + price; COGS into volume + rate; generates waterfall chart
  \item \textbf{Writes} CFO-ready memo in Word format with tables and chart
\end{enumerate}
\vspace{0.2cm}
{\small This is one slash command replacing what typically takes an FP\&A analyst 2--4 hours.}
\end{frame}

\begin{frame}[shrink=10]{The ``SaaSpocalypse'': Market Reaction to Plugins}
On January 30, 2026, Anthropic released 11 open-source plugins for Claude.  Within days, \alert{\$285 billion} in market cap evaporated from software, legal tech, and financial services stocks.
\vspace{0.2cm}
\begin{columns}[T]
\begin{column}{0.5\textwidth}
\begin{shadedbox}[title=\textbf{Hardest-Hit Stocks}]
\scriptsize
\begin{tabular}{@{} l r @{}}
\toprule
\textbf{Company} & \textbf{Drop} \\
\midrule
LegalZoom & $-$20\% \\
Thomson Reuters & $-$16\% \\
RELX (LexisNexis) & $-$14\% \\
Wolters Kluwer & $-$13\% \\
Xero (accounting) & worst day since 2013 \\
Intuit & $>-$10\% \\
Salesforce & $-$7\% \\
ServiceNow & $-$7\% \\
Adobe & $-$7\% \\
GS Software Basket & $-$6\% \\
Indian IT (Infosys, TCS, \ldots) & $-$6 to 8\% \\
\bottomrule
\end{tabular}
\end{shadedbox}
\end{column}
\begin{column}{0.5\textwidth}
\begin{shadedbox}[title=\textbf{Why It Happened}]
\begin{itemize}\scriptsize
  \item Plugins showed AI could replicate \textit{core workflows} of specialized enterprise software
  \item \textbf{No proprietary code}---just markdown instructions + database connectors
  \item Investors asked: ``Why pay \$50K/yr for software when a plugin does it?''
\end{itemize}
\end{shadedbox}
\vspace{0.1cm}
\textbf{The Counterargument}
\begin{itemize}\scriptsize
  \item Plugins are ``research preview''---not production-ready
  \item Regulated industries require audit trails, SOC 2, etc.
  \item But the \textit{direction of travel} spooked the market
\end{itemize}
\end{column}
\end{columns}
\end{frame}

\begin{frame}{Summary}
\begin{columns}[T]
\begin{column}{0.33\textwidth}
\begin{shadedbox}[title=\textbf{Claude Code}]
\begin{itemize}\small
  \item General-purpose agent
  \item LLM + Agent Logic + Tools
  \item Works out of the box
\end{itemize}
\end{shadedbox}
\end{column}
\begin{column}{0.33\textwidth}
\begin{shadedbox}[title=\textbf{Skills}]
\begin{itemize}\small
  \item Specialize Claude Code
  \item System prompt + scripts
  \item Easy to create and modify
\end{itemize}
\end{shadedbox}
\end{column}
\begin{column}{0.33\textwidth}
\begin{shadedbox}[title=\textbf{Plugins}]
\begin{itemize}\small
  \item Package and share skills
  \item Bundle multiple components
  \item Install from marketplaces
\end{itemize}
\end{shadedbox}
\end{column}
\end{columns}
\vspace{0.5cm}
\begin{center}
\textbf{All AI customization is prompt engineering}\\[0.3em]
Skills, plugins, Custom GPTs, Copilot instructions, \texttt{.cursorrules}---same pattern, different platforms
\end{center}
\end{frame}

\end{document}
