\documentclass[aspectratio=169]{beamer}
\usetheme{metropolis}
\usepackage{appendixnumberbeamer}
\usepackage{booktabs, hyperref, graphicx}

\input{mgmt675-style}

\subtitle{MGMT 675: Generative AI for Finance}
\title{Claude Code in Obsidian}
\author{Kerry Back}
\date{}

\begin{document}

\maketitle

\begin{frame}{Why Obsidian?}
\begin{shadedbox}
\textbf{Obsidian} is a free, local-first note-taking app built on plain markdown files.  Your notes are just files in a folder---no cloud lock-in, no proprietary format.
\end{shadedbox}
\vspace{0.3cm}
\begin{columns}[T]
\begin{column}{0.5\textwidth}
\begin{shadedbox}[title=\textbf{Key Features}]
\begin{itemize}\small
  \item Plain markdown files on your computer
  \item Wiki-style \texttt{[[backlinks]]} between notes
  \item Rich plugin ecosystem (2,000+)
  \item Graph view of note connections
  \item Works offline, syncs via any method
\end{itemize}
\end{shadedbox}
\end{column}
\begin{column}{0.5\textwidth}
\begin{shadedbox}[title=\textbf{Why It Pairs Well with AI}]
\begin{itemize}\small
  \item Files are plain text---AI can read and edit them directly
  \item Folder structure = project organization
  \item No API needed---just read/write files
  \item \texttt{CLAUDE.md} at vault root gives Claude context every session
\end{itemize}
\end{shadedbox}
\end{column}
\end{columns}
\end{frame}

\begin{frame}{Obsidian + Claude Code: The Idea}
\begin{shadedbox}
Combine Obsidian's note-taking strengths with Claude Code's agentic capabilities.  Claude can read, write, search, and organize your entire vault---from inside Obsidian.
\end{shadedbox}
\vspace{0.3cm}
\begin{baritemize}
  \item ``Summarize my meeting notes from this week''
  \item ``Add backlinks to all people mentioned in my journal entries''
  \item ``Create a research brief from these three papers''
  \item ``Reorganize my project notes into a cleaner folder structure''
  \item ``Generate a weekly review from my daily notes''
  \item \alert{Claude works directly on your files---results stay in your vault}
\end{baritemize}
\end{frame}

\begin{frame}{Three Ways to Connect Claude Code to Obsidian}
\begin{columns}[T]
\begin{column}{0.33\textwidth}
\begin{shadedbox}[title=\textbf{1.\ Claudian Plugin}]
\begin{itemize}\scriptsize
  \item Full Claude Code agent inside Obsidian
  \item Reads/writes vault files
  \item Runs bash commands
  \item Vision support (images)
  \item \alert{Recommended}
\end{itemize}
\vspace{0.1cm}
{\scriptsize Install from Community Plugins}
\end{shadedbox}
\end{column}
\begin{column}{0.33\textwidth}
\begin{shadedbox}[title=\textbf{2.\ Terminal Plugin}]
\begin{itemize}\scriptsize
  \item Adds a terminal pane to Obsidian
  \item Run \texttt{claude} CLI directly
  \item Full Claude Code experience
  \item Also run Python, git, etc.
  \item Lightweight approach
\end{itemize}
\vspace{0.1cm}
{\scriptsize Search ``Terminal'' in plugins}
\end{shadedbox}
\end{column}
\begin{column}{0.33\textwidth}
\begin{shadedbox}[title=\textbf{3.\ MCP Server}]
\begin{itemize}\scriptsize
  \item Claude Desktop reads your vault via MCP
  \item \texttt{mcp-obsidian} server
  \item Tools: search, read, write, append, delete
  \item Works from Claude Desktop
\end{itemize}
\vspace{0.1cm}
{\scriptsize Best for Desktop app users}
\end{shadedbox}
\end{column}
\end{columns}
\end{frame}

\begin{frame}{The Claudian Plugin}
\begin{columns}[T]
\begin{column}{0.55\textwidth}
\includegraphics[width=\textwidth]{images/claudian-preview}
\end{column}
\begin{column}{0.45\textwidth}
\begin{shadedbox}[title=\textbf{Features}]
\begin{itemize}\small
  \item Claude Code agent in sidebar
  \item Reads and edits your notes
  \item Runs bash commands
  \item Drag-and-drop image analysis
  \item Model selector (Opus, Sonnet)
  \item Shows diffs for file edits
\end{itemize}
\end{shadedbox}
\vspace{0.1cm}
\begin{shadedbox}
\textbf{Install}: Settings $\rightarrow$ Community Plugins $\rightarrow$ search ``Claudian'' $\rightarrow$ Install $\rightarrow$ Enable\\[0.2em]
\textbf{Requires}: Anthropic API key
\end{shadedbox}
\end{column}
\end{columns}
\end{frame}

\begin{frame}[fragile]{Setting Up Your Vault for Claude}
\begin{columns}[T]
\begin{column}{0.45\textwidth}
\begin{shadedbox}[title=\textbf{Vault Structure}]\small
\begin{verbatim}
my-vault/
  CLAUDE.md       <- Key file
  .claude/
    commands/
      day.md
      research.md
      brainstorm.md
  journal/
  research/
  projects/
  entities/
\end{verbatim}
\end{shadedbox}
\end{column}
\begin{column}{0.55\textwidth}
\begin{shadedbox}[title=\textbf{CLAUDE.md: Your AI Instructions}]
\begin{itemize}\small
  \item Placed at vault root
  \item Claude reads it every session
  \item Contains: vault conventions, note format, tagging rules, workflows
  \item Think of it as a \textbf{skill for your vault}
\end{itemize}
\end{shadedbox}
\vspace{0.1cm}
\begin{shadedbox}[title=\textbf{Commands Folder}]
\begin{itemize}\small
  \item \texttt{/day} --- create today's daily note
  \item \texttt{/research} --- start a research note
  \item \texttt{/brainstorm} --- guided brainstorm
  \item Reusable slash commands in markdown
\end{itemize}
\end{shadedbox}
\end{column}
\end{columns}
\end{frame}

\begin{frame}[fragile,shrink=5]{Example: CLAUDE.md for a Finance Vault}
\begin{shadedbox}[title=\textbf{CLAUDE.md (Sketch)}]\small
\begin{verbatim}
# My Finance Knowledge Base

## Vault Conventions
- Daily notes go in journal/ with format YYYY-MM-DD.md
- Research notes go in research/ with YAML frontmatter
- Use [[backlinks]] to connect people, companies, and concepts
- Tags: #earnings, #valuation, #macro, #reading

## Task Format
- [ ] [#tag] Task description — added YYYY-MM-DD

## Workflows
- "weekly review": Scan journal/ for the past 7 days, summarize
- "add task": Append to tasks.md in the appropriate category
- "research brief": Read files I specify, produce a structured summary
\end{verbatim}
\end{shadedbox}
\vspace{0.1cm}
\begin{center}\small
This file is loaded into Claude's context automatically---it's the same concept as a \textbf{skill}, applied to your personal vault.
\end{center}
\end{frame}

\begin{frame}{Workflows: What Can You Do?}
\begin{columns}[T]
\begin{column}{0.5\textwidth}
\begin{shadedbox}[title=\textbf{Knowledge Management}]
\begin{itemize}\small
  \item Auto-add backlinks to entities
  \item Generate summaries from daily notes
  \item Tag and categorize notes
  \item Build reading lists from highlights
  \item Create indexes and MOCs (Maps of Content)
\end{itemize}
\end{shadedbox}
\vspace{0.1cm}
\begin{shadedbox}[title=\textbf{Research}]
\begin{itemize}\small
  \item Summarize papers into structured notes
  \item Extract key findings across sources
  \item Generate literature review outlines
  \item Compare arguments across notes
\end{itemize}
\end{shadedbox}
\end{column}
\begin{column}{0.5\textwidth}
\begin{shadedbox}[title=\textbf{Task \& Project Management}]
\begin{itemize}\small
  \item Manage task lists by category
  \item Weekly reviews from journal entries
  \item Track action items across meetings
  \item Generate project status updates
\end{itemize}
\end{shadedbox}
\vspace{0.1cm}
\begin{shadedbox}[title=\textbf{Content Creation}]
\begin{itemize}\small
  \item Draft documents from notes
  \item Convert rough notes into polished prose
  \item Generate slide outlines from research
  \item Create structured templates
\end{itemize}
\end{shadedbox}
\end{column}
\end{columns}
\end{frame}

\begin{frame}{MCP Server: Claude Desktop + Obsidian}
\begin{shadedbox}
If you prefer using \textbf{Claude Desktop} (rather than a plugin inside Obsidian), you can connect to your vault via an MCP server.
\end{shadedbox}
\vspace{0.3cm}
\begin{columns}[T]
\begin{column}{0.5\textwidth}
\begin{shadedbox}[title=\textbf{mcp-obsidian}]
\begin{itemize}\small
  \item Connects via Obsidian's REST API
  \item Tools: \texttt{list\_files}, \texttt{get\_file\_contents}, \texttt{search}, \texttt{patch\_content}, \texttt{append\_content}
  \item GitHub: \href{https://github.com/MarkusPfundstein/mcp-obsidian}{MarkusPfundstein/mcp-obsidian}
\end{itemize}
\end{shadedbox}
\end{column}
\begin{column}{0.5\textwidth}
\begin{shadedbox}[title=\textbf{Alternative: Just Use the Filesystem}]
\begin{itemize}\small
  \item Obsidian vaults are just folders
  \item Point Claude Code at your vault directory
  \item No MCP server needed---Claude reads/writes files directly
  \item Add \texttt{CLAUDE.md} and go
\end{itemize}
\end{shadedbox}
\end{column}
\end{columns}
\vspace{0.2cm}
\begin{center}\small
\alert{Simplest approach}: Open Claude Code in your vault folder. No plugins or servers required.
\end{center}
\end{frame}

\begin{frame}{Getting Started}
\begin{barenumerate}
  \item \textbf{Install Obsidian}: Free at \url{https://obsidian.md}
  \item \textbf{Create a vault}: Pick a folder for your notes
  \item \textbf{Add CLAUDE.md}: Write your vault conventions and workflows
  \item \textbf{Choose your integration}:
  \begin{itemize}
    \item Claudian plugin (richest experience inside Obsidian)
    \item Terminal plugin (lightweight, run \texttt{claude} CLI)
    \item Claude Code CLI pointed at your vault folder (no plugin needed)
  \end{itemize}
  \item \textbf{Start working}: ``Summarize my notes from today'' or \texttt{/day}
\end{barenumerate}
\vspace{0.2cm}
\begin{shadedbox}
\textbf{Further Reading}:\\[0.2em]
\href{https://github.com/YishenTu/claudian}{Claudian plugin (GitHub)} $\cdot$
\href{https://github.com/MarkusPfundstein/mcp-obsidian}{mcp-obsidian (GitHub)} $\cdot$
\href{https://www.axtonliu.ai/newsletters/ai-2/posts/obsidian-claude-code-workflows}{Obsidian $\times$ Claude Code workflows}
\end{shadedbox}
\end{frame}

\end{document}
