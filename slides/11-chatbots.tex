\documentclass[aspectratio=169]{beamer}
\usetheme{metropolis}
\usepackage{appendixnumberbeamer}
\usepackage{booktabs}

\input{mgmt675-style}

\subtitle{MGMT 675: Generative AI for Finance}
\title{Creating Custom Chatbots}
\author{Kerry Back}

\date{}

\begin{document}

\maketitle

\begin{frame}{What is a Chatbot?}
\begin{columns}[T]
\begin{column}{0.5\textwidth}
\begin{shadedbox}[title=\textbf{Simple Definition}]
A chatbot is a program that:
\begin{enumerate}\small
  \item Accepts user input (prompt)
  \item Sends it to an LLM via API
  \item Displays the response
  \item Repeats in a loop
\end{enumerate}
\end{shadedbox}
\end{column}
\begin{column}{0.5\textwidth}
\begin{baritemize}\small
  \item ChatGPT, Claude, Gemini are chatbots
  \item The LLM is the ``brain''
  \item The chatbot is the interface
  \item You can build your own!
\end{baritemize}
\end{column}
\end{columns}
\end{frame}

\begin{frame}{The API: Talking to an LLM}
\begin{shadedbox}
An \textbf{API} (Application Programming Interface) lets your code communicate with an LLM service over the internet.
\end{shadedbox}
\vspace{0.3cm}
\begin{columns}[T]
\begin{column}{0.5\textwidth}
\begin{shadedbox}[title=\textbf{How It Works}]
\begin{enumerate}\small
  \item Your code sends a request
  \item Request includes your prompt
  \item LLM processes the prompt
  \item API returns the response
  \item Your code displays it
\end{enumerate}
\end{shadedbox}
\end{column}
\begin{column}{0.5\textwidth}
\begin{shadedbox}[title=\textbf{What You Need}]
\begin{itemize}\small
  \item API endpoint (URL)
  \item API key (authentication)
  \item Model name to use
  \item Your prompt/messages
\end{itemize}
\end{shadedbox}
\end{column}
\end{columns}
\end{frame}

\begin{frame}{OpenRouter: One API, Many Models}
\begin{columns}[T]
\begin{column}{0.5\textwidth}
\begin{shadedbox}[title=\textbf{What is OpenRouter?}]
\begin{itemize}\small
  \item Unified API for 100+ models
  \item OpenAI, Anthropic, Google, Meta, etc.
  \item Single API key for all models
  \item Some models are free!
\end{itemize}
\end{shadedbox}
\vspace{0.3cm}
\begin{center}
\texttt{openrouter.ai}
\end{center}
\end{column}
\begin{column}{0.5\textwidth}
\begin{shadedbox}[title=\textbf{Free Models on Hugging Face}]
\begin{itemize}\small
  \item Hugging Face hosts open models
  \item OpenRouter provides free access
  \item Examples:
  \begin{itemize}\footnotesize
    \item \texttt{mistralai/mistral-7b-instruct:free}
    \item \texttt{meta-llama/llama-3-8b-instruct:free}
    \item \texttt{google/gemma-7b-it:free}
  \end{itemize}
\end{itemize}
\end{shadedbox}
\end{column}
\end{columns}
\end{frame}

\begin{frame}[fragile]{A Single API Call}
\begin{shadedbox}[title=\textbf{Basic Structure}]
\begin{verbatim}
import requests

response = requests.post(
    "https://openrouter.ai/api/v1/chat/completions",
    headers={"Authorization": f"Bearer {API_KEY}"},
    json={
        "model": "mistralai/mistral-7b-instruct:free",
        "messages": [{"role": "user", "content": prompt}]
    }
)
answer = response.json()["choices"][0]["message"]["content"]
\end{verbatim}
\end{shadedbox}
\end{frame}

\begin{frame}{The Problem: LLMs Have No Memory}
\begin{shadedbox}
Each API call is independent. The LLM doesn't remember previous messages unless you send them again.
\end{shadedbox}
\vspace{0.5cm}
\begin{columns}[T]
\begin{column}{0.5\textwidth}
\begin{shadedbox}[title=\textbf{Without History}]
\begin{itemize}\small
  \item User: ``My name is Alice''
  \item LLM: ``Nice to meet you, Alice!''
  \item User: ``What's my name?''
  \item LLM: ``I don't know your name.''
\end{itemize}
\end{shadedbox}
\end{column}
\begin{column}{0.5\textwidth}
\begin{shadedbox}[title=\textbf{With History}]
\begin{itemize}\small
  \item Send all previous messages
  \item LLM sees full conversation
  \item Can reference earlier context
  \item Feels like a real conversation
\end{itemize}
\end{shadedbox}
\end{column}
\end{columns}
\end{frame}

\begin{frame}{The Conversation Loop}
\begin{shadedbox}
To create a chatbot, you need a loop that maintains the conversation history.
\end{shadedbox}
\vspace{0.3cm}
\begin{columns}[T]
\begin{column}{0.55\textwidth}
\begin{shadedbox}[title=\textbf{The Pattern}]
\begin{enumerate}\small
  \item Initialize empty message list
  \item Get user input
  \item Append user message to list
  \item Send \alert{entire list} to API
  \item Get response from API
  \item Append assistant response to list
  \item Display response
  \item Go to step 2
\end{enumerate}
\end{shadedbox}
\end{column}
\begin{column}{0.45\textwidth}
\begin{baritemize}\small
  \item History grows each turn
  \item LLM sees everything
  \item Context is preserved
  \item This is how ChatGPT works!
\end{baritemize}
\end{column}
\end{columns}
\end{frame}

\begin{frame}[fragile]{Message Format}
\begin{shadedbox}[title=\textbf{Messages Are a List of Dictionaries}]
\begin{verbatim}
messages = [
    {"role": "user", "content": "My name is Alice"},
    {"role": "assistant", "content": "Nice to meet you!"},
    {"role": "user", "content": "What's my name?"}
]
\end{verbatim}
\end{shadedbox}
\vspace{0.3cm}
\begin{columns}[T]
\begin{column}{0.5\textwidth}
\begin{shadedbox}[title=\textbf{Roles}]
\begin{itemize}\small
  \item \texttt{user}: Human messages
  \item \texttt{assistant}: LLM responses
  \item \texttt{system}: Instructions (special)
\end{itemize}
\end{shadedbox}
\end{column}
\begin{column}{0.5\textwidth}
\begin{baritemize}\small
  \item Each message has role + content
  \item Order matters (chronological)
  \item Send full list each API call
\end{baritemize}
\end{column}
\end{columns}
\end{frame}

\begin{frame}{The System Prompt}
\begin{shadedbox}
The \textbf{system prompt} is a special message that defines the chatbot's personality, knowledge, and behavior. It's the key to customization.
\end{shadedbox}
\vspace{0.3cm}
\begin{columns}[T]
\begin{column}{0.5\textwidth}
\begin{shadedbox}[title=\textbf{What It Does}]
\begin{itemize}\small
  \item Sets the chatbot's persona
  \item Defines its expertise
  \item Specifies response style
  \item Adds domain knowledge
  \item Sets guardrails/rules
\end{itemize}
\end{shadedbox}
\end{column}
\begin{column}{0.5\textwidth}
\begin{shadedbox}[title=\textbf{Placement}]
\begin{itemize}\small
  \item First message in the list
  \item Role is \texttt{system}
  \item Sent with every API call
  \item User never sees it directly
\end{itemize}
\end{shadedbox}
\end{column}
\end{columns}
\end{frame}

\begin{frame}[fragile]{System Prompt Examples}
\begin{columns}[T]
\begin{column}{0.5\textwidth}
\begin{shadedbox}[title=\textbf{Finance Tutor}]\small
\begin{verbatim}
You are a patient finance
tutor. Explain concepts
simply. Use examples with
real companies. Always
check understanding before
moving on.
\end{verbatim}
\end{shadedbox}
\end{column}
\begin{column}{0.5\textwidth}
\begin{shadedbox}[title=\textbf{Investment Analyst}]\small
\begin{verbatim}
You are a senior equity
analyst. Be concise and
data-driven. Always cite
sources. Flag risks and
uncertainties clearly.
\end{verbatim}
\end{shadedbox}
\end{column}
\end{columns}
\vspace{0.3cm}
\begin{center}
\alert{The system prompt transforms a generic LLM into a specialized assistant}
\end{center}
\end{frame}

\begin{frame}[fragile]{Complete Message Structure}
\begin{shadedbox}[title=\textbf{Messages with System Prompt}]
\begin{verbatim}
messages = [
    {"role": "system", "content": "You are a helpful
        finance tutor..."},
    {"role": "user", "content": "What is a P/E ratio?"},
    {"role": "assistant", "content": "A P/E ratio is..."},
    {"role": "user", "content": "How do I interpret it?"}
]
\end{verbatim}
\end{shadedbox}
\vspace{0.3cm}
\begin{baritemize}\small
  \item System prompt stays at the beginning
  \item User and assistant messages alternate
  \item Entire list sent with each API call
\end{baritemize}
\end{frame}

\begin{frame}{Putting It All Together}
\begin{shadedbox}[title=\textbf{Chatbot Architecture}]
\begin{center}
\begin{tabular}{cl}
\textbf{Component} & \textbf{Purpose} \\
\midrule
System Prompt & Defines chatbot personality/expertise \\
Message List & Stores conversation history \\
Loop & Continuously gets input, calls API, shows output \\
API Call & Sends messages, receives response \\
\end{tabular}
\end{center}
\end{shadedbox}
\vspace{0.3cm}
\begin{center}
\alert{These four components create any custom chatbot}
\end{center}
\end{frame}

\begin{frame}{Exercise: Build a Finance Chatbot}
\begin{shadedbox}
Build and deploy a Streamlit chatbot using OpenRouter and a free model.
\end{shadedbox}
\vspace{0.3cm}
\begin{columns}[T]
\begin{column}{0.5\textwidth}
\begin{shadedbox}[title=\textbf{Requirements}]
\begin{itemize}\small
  \item Custom system prompt
  \item Conversation history
  \item Streamlit interface
  \item Deploy to Koyeb
\end{itemize}
\end{shadedbox}
\end{column}
\begin{column}{0.5\textwidth}
\begin{shadedbox}[title=\textbf{Ask Claude}]
``Create a Streamlit chatbot app using OpenRouter with [model]. Make it a [describe persona]. Deploy to Koyeb.''
\end{shadedbox}
\end{column}
\end{columns}
\end{frame}

\begin{frame}{Summary}
\begin{columns}[T]
\begin{column}{0.5\textwidth}
\begin{shadedbox}[title=\textbf{Key Concepts}]
\begin{itemize}\small
  \item API calls send prompts to LLMs
  \item LLMs have no memory
  \item You must send full history
  \item System prompt customizes behavior
\end{itemize}
\end{shadedbox}
\end{column}
\begin{column}{0.5\textwidth}
\begin{shadedbox}[title=\textbf{Building Blocks}]
\begin{itemize}\small
  \item OpenRouter API
  \item Message list (history)
  \item Conversation loop
  \item System prompt
\end{itemize}
\end{shadedbox}
\end{column}
\end{columns}
\vspace{0.5cm}
\begin{center}
\begin{shadedbox}[width=0.8\textwidth]
\centering
\alert{A chatbot = Loop + History + System Prompt + API}
\end{shadedbox}
\end{center}
\end{frame}

\end{document}
