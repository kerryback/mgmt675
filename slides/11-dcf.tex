\documentclass[aspectratio=169]{beamer}
\usetheme{metropolis}
\usepackage{appendixnumberbeamer}
\usepackage{booktabs, hyperref}

\input{mgmt675-style}

\usepackage{tikz}
\usetikzlibrary{shapes.geometric, arrows.meta, positioning, calc}

\subtitle{MGMT 675: Generative AI for Finance}
\title{Valuing Companies with AI}
\author{Kerry Back}
\date{}

\begin{document}

\maketitle

% ============================================================
% OVERVIEW
% ============================================================

\begin{frame}{Goals}
\begin{itemize}
  \item Build \textbf{pro forma financial statements} and compute \textbf{free cash flow}
  \item Perform a \textbf{two-stage DCF valuation} with \textbf{sensitivity analysis}
  \item Use \textbf{Monte Carlo simulation} to produce a distribution of values
\end{itemize}
\end{frame}

% ============================================================
% PRO FORMA ASSUMPTIONS
% ============================================================
\section{Pro Forma Statements}

\begin{frame}{The Assumption Set}
A pro forma model is driven by a small number of assumptions.  Everything else is computed from these.
\vspace{0.2cm}
\begin{center}
\small
\begin{tabular}{@{} l l @{}}
\toprule
\textbf{Assumption} & \textbf{Driver} \\
\midrule
Sales growth rate & \% per year \\
COGS & \% of sales \\
SG\&A & Base amount + \% of sales \\
Net working capital (NWC) & \% of sales \\
PP\&E & \% of sales \\
Net other operating assets & \% of sales \\
Depreciation & \% of PP\&E \\
Tax rate & \% of pre-tax income \\
\bottomrule
\end{tabular}
\end{center}
\vspace{0.2cm}
Capital expenditures are not assumed directly.  \alert{Cap ex = target PP\&E $-$ prior PP\&E + depreciation}, i.e., whatever is needed to hit the PP\&E target after accounting for depreciation.
\end{frame}

\begin{frame}{Pro Forma Income Statement}
\begin{center}
\small
\begin{tabular}{@{} l l @{}}
\toprule
\textbf{Line Item} & \textbf{How It's Computed} \\
\midrule
Sales & Prior sales $\times$ (1 + growth rate) \\
COGS & COGS\% $\times$ sales \\
\textbf{Gross profit} & Sales $-$ COGS \\
SG\&A & Base + SG\&A\% $\times$ sales \\
Depreciation & Depr\% $\times$ PP\&E \\
\textbf{EBIT} & Gross profit $-$ SG\&A $-$ depreciation \\
Taxes & Tax rate $\times$ EBIT \\
\textbf{NOPAT} & EBIT $-$ taxes \\
\bottomrule
\end{tabular}
\end{center}
\vspace{0.3cm}
We use \textbf{NOPAT} (net operating profit after tax) rather than net income because DCF values the \textit{operations} of the firm, independent of how they are financed.
\end{frame}

\begin{frame}{Pro Forma Balance Sheet and Cap Ex}
\begin{columns}[T]
\begin{column}{0.48\textwidth}
\begin{shadedbox}[title=\textbf{Balance Sheet Items}]
\begin{itemize}\small
  \item \textbf{NWC} = NWC\% $\times$ sales
  \item \textbf{PP\&E} = PP\&E\% $\times$ sales
  \item \textbf{Net other OA} = Net other OA\% $\times$ sales
  \item Change in each = current $-$ prior year
\end{itemize}
\end{shadedbox}
\end{column}
\begin{column}{0.48\textwidth}
\begin{shadedbox}[title=\textbf{Capital Expenditures}]
\begin{itemize}\small
  \item Target PP\&E$_t$ = PP\&E\% $\times$ sales$_t$
  \item Depreciation$_t$ = depr\% $\times$ PP\&E$_{t-1}$
  \item \alert{Cap ex$_t$ = PP\&E$_t$ $-$ PP\&E$_{t-1}$ + depr$_t$}
\end{itemize}
\end{shadedbox}
\end{column}
\end{columns}
\vspace{0.3cm}
\textbf{Free Cash Flow to the Firm}
\begin{center}
\textbf{FCF = NOPAT + Depreciation $-$ Cap Ex $-$ $\Delta$NWC $-$ $\Delta$Net Other OA}
\end{center}
\vspace{0.2cm}
Equivalently: FCF = NOPAT $-$ net investment in PP\&E $-$ net investment in NWC $-$ net investment in other operating assets.
\end{frame}

% ============================================================
% TWO-STAGE DCF
% ============================================================
\section{Two-Stage DCF}

\begin{frame}{Two-Stage DCF: Overview}
\begin{center}
\begin{tikzpicture}[
  node distance=0.8cm and 2cm,
  every node/.style={font=\small},
  sbox/.style={draw=titlegray, rounded corners, minimum width=3.5cm, minimum height=0.9cm, fill=excelinput, text=titlegray, font=\small\bfseries, align=center},
  sarrow/.style={-{Stealth[length=3mm]}, thick, draw=accentblue}
]
\node[sbox] (s1) {Stage 1\\Explicit Forecast\\(e.g., 5--10 years)};
\node[sbox, right=2cm of s1] (s2) {Stage 2\\Terminal Value\\(perpetuity)};
\node[sbox, below=1.2cm of $(s1)!0.5!(s2)$] (ev) {Enterprise Value\\= PV(Stage 1) + PV(Stage 2)};

\draw[sarrow] (s1) -- (ev);
\draw[sarrow] (s2) -- (ev);
\end{tikzpicture}
\end{center}
\vspace{0.2cm}
\begin{itemize}
  \item \textbf{Stage 1}: Project FCF each year from pro forma assumptions
  \item \textbf{Stage 2}: Terminal value = FCF$_{T+1}$ / (WACC $-$ $g$), where $g$ is the long-run growth rate
  \item \textbf{Equity value} = Enterprise value $-$ net debt (discount at the WACC)
\end{itemize}
\end{frame}

\begin{frame}{Terminal Value}
The terminal value typically accounts for \alert{60--80\%} of total enterprise value.  Getting it right matters more than the explicit forecast.
\vspace{0.3cm}
\begin{columns}[T]
\begin{column}{0.48\textwidth}
\begin{shadedbox}[title=\textbf{Growing Perpetuity}]
\begin{itemize}\small
  \item TV = FCF$_{T+1}$ / (WACC $-$ $g$)
  \item FCF$_{T+1}$ = year $T$ FCF $\times$ (1 + $g$)
  \item $g$ = long-run nominal growth (typically 2--3\%)
  \item Requires $g <$ WACC
\end{itemize}
\end{shadedbox}
\end{column}
\begin{column}{0.48\textwidth}
\begin{shadedbox}[title=\textbf{Exit Multiple}]
\begin{itemize}\small
  \item TV = EBITDA$_T$ $\times$ exit multiple
  \item Multiple from comparable firms
  \item Common in practice (M\&A, PE)
  \item Cross-check against perpetuity method
\end{itemize}
\end{shadedbox}
\end{column}
\end{columns}
\end{frame}

% ============================================================
% SENSITIVITY ANALYSIS
% ============================================================
\section{Sensitivity Analysis}

\begin{frame}{Sensitivity Tables}
A sensitivity table shows how the output (e.g., equity value per share) changes as you vary one or two key inputs.
\vspace{0.3cm}
\begin{columns}[T]
\begin{column}{0.48\textwidth}
\begin{shadedbox}[title=\textbf{One-Way Table}]
\begin{itemize}\small
  \item Vary a single input (e.g., WACC from 8\% to 12\%)
  \item Hold everything else constant
  \item Shows which inputs matter most
\end{itemize}
\end{shadedbox}
\end{column}
\begin{column}{0.48\textwidth}
\begin{shadedbox}[title=\textbf{Two-Way Table}]
\begin{itemize}\small
  \item Vary two inputs simultaneously
  \item Classic: WACC vs.\ terminal growth rate
  \item Also useful: sales growth vs.\ COGS margin
  \item Reveals interaction effects
\end{itemize}
\end{shadedbox}
\end{column}
\end{columns}
\vspace{0.3cm}
Ask Claude to produce sensitivity tables in Excel (with formulas) or as formatted output.  In Excel, the Data Table feature automates two-way tables.
\end{frame}

% ============================================================
% MONTE CARLO SIMULATION
% ============================================================
\section{Monte Carlo Simulation}

\begin{frame}{Why Simulate?}
\begin{itemize}
  \item Sensitivity tables vary one or two inputs; in reality, \alert{all assumptions are uncertain simultaneously}
  \item Monte Carlo simulation draws random values for each assumption, computes FCF and enterprise value, and repeats thousands of times
  \item The result: a \textbf{distribution} of enterprise values, not a single point estimate
\end{itemize}
\end{frame}

\begin{frame}{Simulation Setup}
Assign a probability distribution to each uncertain assumption, then sample and compute.
\vspace{0.2cm}
\begin{center}
\small
\begin{tabular}{@{} l l l @{}}
\toprule
\textbf{Assumption} & \textbf{Distribution} & \textbf{Example} \\
\midrule
Sales growth & Normal & $\mu = 8\%$, $\sigma = 3\%$ \\
COGS \% & Normal & $\mu = 60\%$, $\sigma = 2\%$ \\
Terminal growth & Uniform & 1.5\% to 3.5\% \\
WACC & Normal & $\mu = 10\%$, $\sigma = 1\%$ \\
\bottomrule
\end{tabular}
\end{center}
\vspace{0.2cm}
\begin{itemize}
  \item Run 10{,}000 simulations $\rightarrow$ 10{,}000 enterprise values
  \item Report mean, median, 10th/90th percentiles
  \item Plot a histogram and identify which assumptions drive the most variance
\end{itemize}
\end{frame}

% ============================================================
% DOING IT WITH AI
% ============================================================
\section{Doing It with AI}

\begin{frame}{Multiple Ways to Build a DCF with AI}
\begin{center}
\small
\begin{tabular}{@{} l p{9.5cm} @{}}
\toprule
\textbf{Platform} & \textbf{What to Ask} \\
\midrule
Chat & ``Build a two-stage DCF model for a company with these assumptions \ldots\ Generate an Excel file with pro formas, FCF, and a sensitivity table.'' \\
Cowork & Point Claude at a folder with data files. ``Read the financial data, build pro formas, run a DCF, and save the results to Excel.'' \\
Code & ``Build a DCF model in Python.  Fetch Apple's financials from FMP, estimate assumptions from historical data, and run a Monte Carlo simulation.'' \\
Excel add-in & Open a blank workbook.  ``Build a two-stage DCF with pro forma statements, a sensitivity table for WACC vs.\ terminal growth, and a tornado chart.'' \\
\bottomrule
\end{tabular}
\end{center}
\vspace{0.2cm}
\alert{Key difference:} Chat and Cowork generate spreadsheets via Python (no internet).  Code mode can \alert{fetch live data} from APIs before building the model.
\vspace{0.1cm}

These exercises work with any capable AI---ChatGPT, Gemini, or Claude.
\end{frame}

% ============================================================
% EXERCISES
% ============================================================
\section{Exercises}

\begin{frame}{Exercise: DCF in Chat or Cowork}
\begin{itemize}
  \item In Claude.ai or Claude Desktop (Chat or Cowork), ask Claude to build a two-stage DCF model for a hypothetical company with the following assumptions:
  \begin{itemize}\small
    \item Current sales: \$500M; sales growth: 10\% (years 1--5), 3\% terminal
    \item COGS: 58\% of sales; SG\&A: \$20M + 12\% of sales
    \item NWC: 15\% of sales; PP\&E: 40\% of sales; net other OA: 5\% of sales; depreciation: 10\% of PP\&E
    \item Tax rate: 25\%; WACC: 9\%; net debt: \$200M; shares: 50M
  \end{itemize}
  \item Ask Claude to generate an Excel file with:
  \begin{itemize}\small
    \item Pro forma income statements and balance sheet items
    \item Free cash flow calculation
    \item Enterprise value and equity value per share
    \item A two-way sensitivity table: WACC (7--11\%) vs.\ terminal growth (1--4\%)
  \end{itemize}
\end{itemize}
\end{frame}

\begin{frame}{Exercise: DCF with Live Data in Code Mode}
\begin{itemize}
  \item In Code mode, ask Claude to fetch Apple's income statement and balance sheet from Financial Modeling Prep (or Alpha Vantage).
  \item Ask it to estimate historical averages for each assumption (COGS\%, SG\&A\%, NWC\%, PP\&E\%, net other OA\%, depreciation\%).
  \item Ask it to build a 5-year pro forma model using those historical averages, perform a two-stage DCF, and produce sensitivity tables.
  \item Compare the AI-estimated equity value to Apple's current market cap.
\end{itemize}
\end{frame}

\begin{frame}{Exercise: Monte Carlo Simulation}
\begin{itemize}
  \item Using the DCF model from either exercise above, ask Claude to run a Monte Carlo simulation with 10{,}000 trials.
  \item Ask it to vary sales growth, COGS\%, WACC, and terminal growth rate using normal or uniform distributions of your choice.
  \item Ask for:
  \begin{itemize}\small
    \item A histogram of simulated equity values per share
    \item The mean, median, 10th percentile, and 90th percentile
    \item A tornado chart showing which assumption has the largest impact on value
  \end{itemize}
  \item Discuss: How does the range of simulated values compare to the single point estimate?
\end{itemize}
\end{frame}

\begin{frame}{Exercise: DCF in the Excel Add-in}
\begin{itemize}
  \item Open a blank Excel workbook and launch the Claude add-in.
  \item Ask Claude to build a two-stage DCF model with:
  \begin{itemize}\small
    \item An assumptions panel at the top (editable cells)
    \item Pro forma income statements with \alert{Excel formulas} (not hardcoded values)
    \item FCF and enterprise value calculations
    \item A two-way sensitivity table using Excel's Data Table feature
  \end{itemize}
  \item Change a few assumptions manually and verify the model recalculates correctly.
  \item Ask Claude to add conditional formatting to highlight the cells where equity value per share is negative.
\end{itemize}
\end{frame}

\end{document}
