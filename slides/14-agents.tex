\documentclass[aspectratio=169]{beamer}
\usetheme{metropolis}
\usepackage{appendixnumberbeamer}
\usepackage{booktabs, hyperref}

\input{mgmt675-style}

\subtitle{MGMT 675: Generative AI for Finance}
\title{Building AI Agents}
\author{Kerry Back}

\date{}

\begin{document}

\maketitle

% ========================================
% PART 1: WHAT IS AN AGENT?
% ========================================

\begin{frame}{Chatbot vs Agent}
\begin{columns}[T]
\begin{column}{0.5\textwidth}
\begin{shadedbox}[title=\textbf{Chatbot}]
\begin{itemize}\small
  \item Receives prompts
  \item Generates text responses
  \item Cannot take actions
\end{itemize}
\end{shadedbox}
\end{column}
\begin{column}{0.5\textwidth}
\begin{shadedbox}[title=\textbf{Agent}]
\begin{itemize}\small
  \item Can generate text \alert{or} request actions
  \item Has access to \textbf{tools}
  \item Takes actions to accomplish goals
\end{itemize}
\end{shadedbox}
\end{column}
\end{columns}
\vspace{0.3cm}
\begin{center}
\alert{An agent is a chatbot with tools}
\end{center}
\end{frame}

\begin{frame}{What Are Tools?}
\textbf{Tools} are functions the agent can call to interact with the outside world.
\vspace{0.3cm}
\begin{columns}[T]
\begin{column}{0.5\textwidth}
\begin{shadedbox}[title=\textbf{Example Tools}]
\begin{itemize}\small
  \item Execute SQL queries
  \item Run Python code
  \item Search the web
\end{itemize}
\end{shadedbox}
\end{column}
\begin{column}{0.5\textwidth}
\begin{shadedbox}[title=\textbf{How Tools Work}]
\begin{enumerate}\small
  \item LLM decides to use a tool
  \item Returns tool name + parameters
  \item Result sent back to LLM
\end{enumerate}
\end{shadedbox}
\end{column}
\end{columns}
\end{frame}

% ========================================
% PART 2: FOUNDATION --- THE API
% ========================================
\section{Foundation: Calling an LLM from Code}

\begin{frame}{The API: Talking to an LLM}
An \textbf{API} (Application Programming Interface) lets your code communicate with an LLM service over the internet.
\vspace{0.3cm}
\begin{columns}[T]
\begin{column}{0.5\textwidth}
\begin{shadedbox}[title=\textbf{How It Works}]
\begin{enumerate}\small
  \item Your code sends a request
  \item LLM processes the prompt
  \item API returns the response
\end{enumerate}
\end{shadedbox}
\end{column}
\begin{column}{0.5\textwidth}
\begin{shadedbox}[title=\textbf{What You Need}]
\begin{itemize}\small
  \item API endpoint (URL)
  \item API key (authentication)
  \item Model name to use
\end{itemize}
\end{shadedbox}
\end{column}
\end{columns}
\end{frame}

\begin{frame}{OpenRouter: One API, Many Models}
\begin{columns}[T]
\begin{column}{0.5\textwidth}
\begin{shadedbox}[title=\textbf{What is OpenRouter?}]
\begin{itemize}\small
  \item Unified API for 100+ models
  \item Single API key for all models
  \item Some models are free!
\end{itemize}
\end{shadedbox}
\vspace{0.3cm}
\begin{center}
\texttt{openrouter.ai}
\end{center}
\end{column}
\begin{column}{0.5\textwidth}
\begin{shadedbox}[title=\textbf{Free Models on Hugging Face}]
\begin{itemize}\small
  \item Hugging Face hosts open models
  \item OpenRouter provides free access
  \item Examples:
  \begin{itemize}\footnotesize
    \item \texttt{mistralai/mistral-7b-instruct:free}
    \item \texttt{meta-llama/llama-3-8b-instruct:free}
    \item \texttt{google/gemma-7b-it:free}
  \end{itemize}
\end{itemize}
\end{shadedbox}
\end{column}
\end{columns}
\end{frame}

\begin{frame}[fragile]{A Single API Call}
\begin{shadedbox}[title=\textbf{Basic Structure}]
\begin{verbatim}
import requests

response = requests.post(
    "https://openrouter.ai/api/v1/chat/completions",
    headers={"Authorization": f"Bearer {API_KEY}"},
    json={
        "model": "mistralai/mistral-7b-instruct:free",
        "messages": [{"role": "user", "content": prompt}]
    }
)
answer = response.json()["choices"][0]["message"]["content"]
\end{verbatim}
\end{shadedbox}
\end{frame}

\begin{frame}[fragile]{Message Format and Conversation History}
\begin{shadedbox}[title=\textbf{Messages Are a List of Dictionaries}]
\begin{verbatim}
messages = [
    {"role": "system", "content": "You are a helpful
        finance tutor..."},
    {"role": "user", "content": "What is a P/E ratio?"},
    {"role": "assistant", "content": "A P/E ratio is..."},
    {"role": "user", "content": "How do I interpret it?"}
]
\end{verbatim}
\end{shadedbox}
\vspace{0.3cm}
\begin{itemize}\small
  \item Each API call is independent---LLM has no memory
  \item You must send the \alert{entire conversation history} each time
  \item The \textbf{system prompt} (role \texttt{system}) defines the agent's behavior
\end{itemize}
\end{frame}

% ========================================
% PART 3: THE AGENT LOOP
% ========================================
\section{The Agent Loop}

\begin{frame}{The Agent Loop}
The agent runs in a loop: LLM thinks $\rightarrow$ tool executes $\rightarrow$ result returns $\rightarrow$ LLM thinks again.
\vspace{0.3cm}
\begin{center}
\begin{tabular}{ccc}
& \textbf{Tool Call} & \\
\fbox{\textbf{LLM}} & $\longrightarrow$ & \fbox{\textbf{Tool}} \\
& $\longleftarrow$ & \\
& \textbf{Result} & \\
\end{tabular}
\end{center}
\vspace{0.3cm}
\begin{columns}[T]
\begin{column}{0.5\textwidth}
\begin{itemize}\small
  \item LLM decides next action
  \item Returns tool name + parameters
  \item Agent executes the tool
\end{itemize}
\end{column}
\begin{column}{0.5\textwidth}
\begin{itemize}\small
  \item Result added to message history
  \item LLM sees result, reasons again
  \item Loop continues until task complete
\end{itemize}
\end{column}
\end{columns}
\end{frame}

\begin{frame}[fragile]{Agent Loop Pseudocode}
\begin{shadedbox}[title=\textbf{The Agent's Decision Loop}]
\begin{verbatim}
while not done:
    response = call_llm(messages, system_prompt, tools)

    if response.has_tool_call:
        result = execute_tool(response.tool_call)
        messages.append(tool_result)
    elif response.needs_user_input:
        answer = ask_user(response.question)
        messages.append(user_answer)
    elif response.is_final:
        done = True
        return response.content
\end{verbatim}
\end{shadedbox}
\vspace{0.2cm}
\begin{center}
\alert{The agent is part LLM intelligence, part traditional programming}
\end{center}
\end{frame}

% ========================================
% PART 4: BUILDING AN AGENT STEP BY STEP
% ========================================
\section{Building an Agent Step by Step}

\begin{frame}{The Key Pieces}
To build an agent, you need five components:
\vspace{0.3cm}
\begin{enumerate}
  \item \textbf{Define tools}: What actions can the agent take?
  \item \textbf{Write a system prompt}: Describe available tools, schema, and rules
  \item \textbf{Implement the agent loop}: Send messages, parse tool calls, execute, repeat
\end{enumerate}
\vspace{0.3cm}
\begin{center}
\alert{Or use an agent framework: LangChain, CrewAI, Claude Code SDK}
\end{center}
\end{frame}

\begin{frame}{Example: Database Analytics Agent}
User request: ``Analyze quarterly revenue trends for our top 5 customers and create a summary report.''
\vspace{0.3cm}
\textbf{What the Agent Needs to Do}
\begin{enumerate}\small
  \item Write SQL to identify top 5 customers by revenue
  \item Execute the SQL query against the database
  \item Write Python to analyze trends and create visualizations
\end{enumerate}
\end{frame}

\begin{frame}{Step 1: User Prompt Arrives}
\begin{shadedbox}[title=\textbf{Messages Sent to LLM}]
\begin{tabular}{ll}
\textbf{Role} & \textbf{Content} \\
\midrule
system & You are a data analyst with SQL and Python tools... \\
user & Analyze quarterly revenue trends for top 5 customers... \\
\end{tabular}
\end{shadedbox}
\vspace{0.3cm}
\begin{columns}[T]
\begin{column}{0.5\textwidth}
\begin{itemize}\small
  \item System prompt defines capabilities
  \item Lists available tools
  \item Specifies database schema
\end{itemize}
\end{column}
\begin{column}{0.5\textwidth}
\textbf{LLM Decides}\\[0.3em]
``I need to first find the top 5 customers. I'll write a SQL query.''
\end{column}
\end{columns}
\end{frame}

\begin{frame}[fragile]{Step 2: LLM Requests SQL Tool}
\begin{shadedbox}[title=\textbf{LLM Response (Not Text---A Tool Call)}]
\begin{verbatim}
{
  "tool": "execute_sql",
  "parameters": {
    "query": "SELECT customer_id, SUM(amount) as total
              FROM sales GROUP BY customer_id
              ORDER BY total DESC LIMIT 5"
  }
}
\end{verbatim}
\end{shadedbox}
\vspace{0.3cm}
\begin{itemize}\small
  \item LLM doesn't return text to user yet
  \item Instead, requests a tool execution
  \item Agent code intercepts this and runs the SQL
\end{itemize}
\end{frame}

\begin{frame}[fragile]{Step 3: Tool Result Returns to LLM}
\begin{shadedbox}[title=\textbf{Messages Now Include Tool Result}]
\begin{verbatim}
[
  {"role": "system", "content": "You are a data analyst..."},
  {"role": "user", "content": "Analyze quarterly revenue..."},
  {"role": "assistant", "tool_call": "execute_sql(...)"},
  {"role": "tool", "content": "customer_id,total\n
                               ACME,450000\nGlobex,380000\n..."}
]
\end{verbatim}
\end{shadedbox}
\vspace{0.3cm}
\begin{center}
\alert{The LLM sees the full history including tool results}
\end{center}
\end{frame}

\begin{frame}{Step 4: LLM Continues Reasoning}
With the top 5 customers identified, the LLM decides what to do next.
\vspace{0.3cm}
\begin{columns}[T]
\begin{column}{0.5\textwidth}
\textbf{LLM Thinks}\\[0.3em]
``I have the top 5 customers: ACME, Globex, Initech, Umbrella, Wayne. Now I need quarterly data for each.''
\end{column}
\begin{column}{0.5\textwidth}
\textbf{Next Tool Call}\\[0.3em]
\texttt{execute\_sql}: Get quarterly revenue for these 5 customers...
\end{column}
\end{columns}
\vspace{0.3cm}
\begin{center}
The loop continues: tool call $\rightarrow$ result $\rightarrow$ reasoning $\rightarrow$ next action
\end{center}
\end{frame}

\begin{frame}{The Complete Message History}
\begin{shadedbox}[title=\textbf{What the LLM Sees at Report Time}]
\begin{center}
\begin{tabular}{cl}
\# & \textbf{Message} \\
\midrule
1 & system: You are a data analyst... \\
2 & user: Analyze quarterly revenue... \\
3 & assistant: [tool call: SQL for top 5] \\
4 & tool: [results: ACME, Globex...] \\
5 & assistant: [tool call: SQL for quarterly data] \\
6 & tool: [results: Q1, Q2, Q3...] \\
7 & assistant: [tool call: Python analysis] \\
8 & tool: [output: stats, saved trends.png] \\
9 & assistant: Here is my analysis... (final report)
\end{tabular}
\end{center}
\end{shadedbox}
\end{frame}

% ========================================
% PART 5: REAL EXAMPLE --- RICE DATA PORTAL
% ========================================
\section{Example: Rice Data Portal}

\begin{frame}{Rice Data Portal: A Database Agent}
\begin{columns}
\begin{column}{0.4\textwidth}
  \vskip\baselineskip
\begin{itemize}\small
  \item Prompt + System Prompt $\rightarrow$ LLM
  \item LLM $\rightarrow$ SQL code or a question for the user
  \item Eventually, SQL code $\rightarrow$ database
\end{itemize}
\end{column}
\begin{column}{0.6\textwidth}
\vspace{1cm}
\begin{center}
  Visit \href{https://data-portal.rice-business.org}{data-portal.rice-business.org}
  \newline Get access token, Log in, Ask for data
\end{center}
\vspace*{-2\baselineskip}
\includegraphics[width=\textwidth]{images/agent_database.png}
\end{column}
\end{columns}
\end{frame}

\begin{frame}{How the Data Portal is Built}
\begin{itemize}
  \item System prompt provides the database schema and instructions to the LLM
  \item Agent logic is hard-coded as ``if \ldots\ then \ldots'' (\alert{not AI})
  \item The LLM writes SQL; the agent executes it
\end{itemize}
\vspace{0.3cm}
\textbf{Architecture}
\begin{center}
\begin{tabular}{ll}
\textbf{Component} & \textbf{Implementation} \\
\midrule
LLM & OpenRouter API (various models) \\
Tool & SQL query execution against PostgreSQL \\
Agent Loop & Python if/else logic \\
UI & Streamlit web interface \\
\end{tabular}
\end{center}
\end{frame}

% ========================================
% PART 6: UI AND ORCHESTRATION
% ========================================
\section{User Interface and Deployment}

\begin{frame}{Adding a User Interface}
An agent doesn't need a web UI---it can run in a terminal or as a script. But a UI makes it accessible to non-technical users.
\vspace{0.3cm}
\begin{columns}[T]
\begin{column}{0.5\textwidth}
\begin{shadedbox}[title=\textbf{Streamlit}]
\begin{itemize}\small
  \item Python library for web apps
  \item No HTML/CSS/JavaScript needed
  \item Built-in chat UI components
\end{itemize}
\end{shadedbox}
\end{column}
\begin{column}{0.5\textwidth}
\begin{shadedbox}[title=\textbf{Gradio}]
\begin{itemize}\small
  \item Python library for ML demos
  \item Even simpler chat interface
  \item Built-in sharing via public links
\end{itemize}
\end{shadedbox}
\end{column}
\end{columns}
\vspace{0.3cm}
\begin{center}
\alert{Both let you build a chat UI in a few lines of Python---no web development skills required}
\end{center}
\end{frame}

\begin{frame}{With or Without a UI}
\begin{columns}[T]
\begin{column}{0.5\textwidth}
\begin{shadedbox}[title=\textbf{Without UI (Terminal)}]
\begin{itemize}\small
  \item Agent runs in a Python script
  \item Results printed to the terminal
  \item Simple, fast to develop
\end{itemize}
\end{shadedbox}
\end{column}
\begin{column}{0.5\textwidth}
\begin{shadedbox}[title=\textbf{With Streamlit/Gradio UI}]
\begin{itemize}\small
  \item Agent wrapped in a web interface
  \item Shareable via URL
  \item Accessible to non-technical users
\end{itemize}
\end{shadedbox}
\end{column}
\end{columns}
\vspace{0.3cm}
The \textbf{agent logic is the same} either way. Streamlit and Gradio only handle how the user interacts with the agent---they don't change how the agent works.
\end{frame}

\begin{frame}{From Idea to Deployed Agent}
The complete workflow to build and deploy an agent---all handled by Claude Code with natural language requests.
\vspace{0.3cm}
\begin{enumerate}
  \item \textbf{Build the agent}: Define tools, system prompt, and agent loop
  \item \textbf{Add a UI} (optional): Wrap with Streamlit or Gradio
  \item \textbf{Deploy}: Push to a cloud platform (e.g., Koyeb) for a public URL
\end{enumerate}
\vspace{0.3cm}
\begin{center}
\alert{Ask Claude Code to do each step---no commands to memorize}
\end{center}
\end{frame}

% ========================================
% PART 7: ORCHESTRATION
% ========================================
\section{Advanced: Orchestration}

\begin{frame}{The Orchestration Layer}
The agent's control logic coordinates everything: which LLM to call, which system prompt to use, and what to do with each response.
\vspace{0.3cm}
\begin{columns}[T]
\begin{column}{0.5\textwidth}
\begin{shadedbox}[title=\textbf{Different Tasks, Different Prompts}]
\begin{itemize}\small
  \item SQL generation $\rightarrow$ database schema prompt
  \item Python analysis $\rightarrow$ data science prompt
  \item Each task gets specialized instructions
\end{itemize}
\end{shadedbox}
\end{column}
\begin{column}{0.5\textwidth}
\begin{shadedbox}[title=\textbf{Different Tasks, Different LLMs}]
\begin{itemize}\small
  \item Simple classification $\rightarrow$ fast, cheap model
  \item Complex reasoning $\rightarrow$ powerful model
  \item Cost and speed optimization
\end{itemize}
\end{shadedbox}
\end{column}
\end{columns}
\end{frame}

\begin{frame}{Claude Code: A General-Purpose Agent}
\begin{columns}
\begin{column}{0.45\textwidth}
\begin{itemize}\small
  \item Claude Code is itself an agent
  \item LLM: Claude Opus or Sonnet
  \item Built-in tools: file I/O, bash, code execution, web search
\end{itemize}
\end{column}
\begin{column}{0.55\textwidth}
\textbf{Claude Code Architecture}
\begin{center}
\begin{tabular}{ll}
\textbf{Component} & \textbf{Provider} \\
\midrule
LLM & Claude Opus/Sonnet \\
Agent Logic & Claude Code app \\
Base Tools & Claude Code app \\
System Prompt & \alert{Skill} \\
Custom Tools & \alert{Skill scripts} \\
\end{tabular}
\end{center}
\end{column}
\end{columns}
\end{frame}

% ========================================
% SUMMARY
% ========================================

\begin{frame}{Summary}
\begin{columns}[T]
\begin{column}{0.5\textwidth}
\begin{shadedbox}[title=\textbf{Key Concepts}]
\begin{itemize}\small
  \item Agent = Chatbot + Tools
  \item LLM decides which tools to use
  \item Agent loop orchestrates the flow
\end{itemize}
\end{shadedbox}
\end{column}
\begin{column}{0.5\textwidth}
\begin{shadedbox}[title=\textbf{Building Blocks}]
\begin{itemize}\small
  \item API call to an LLM (OpenRouter)
  \item System prompt with tool definitions
  \item Agent loop (while not done)
\end{itemize}
\end{shadedbox}
\end{column}
\end{columns}
\vspace{0.5cm}
\begin{center}
\alert{Agents extend LLMs from conversation to action}
\end{center}
\end{frame}

\end{document}
