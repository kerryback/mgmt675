\documentclass[aspectratio=169]{beamer}
\usetheme{metropolis}
\usepackage{appendixnumberbeamer}
\usepackage{booktabs}
\usepackage{textcomp}

\input{mgmt675-style}

\subtitle{MGMT 675: Generative AI for Finance}
\title{AI-Written Code in Jupyter Notebooks}
\author{Kerry Back}

\date{}
\titlegraphic{\includegraphics[width=4cm]{images/colab_slide1_img4.png}}

\begin{document}

\maketitle

\begin{frame}{A Different Approach}
Code Environment + Chatbot

\begin{itemize}
  \item ChatGPT and Claude are chatbots with code execution; Colab is a code environment with a chatbot
  \item Colab started as Jupyter notebooks in the cloud (2017); Gemini was integrated in 2024
  \item Philosophy: Write and run code first, use AI to assist
\end{itemize}
\end{frame}

\begin{frame}{What is Google Colab?}
\begin{itemize}
  \item A free tool from Google for running code in your browser
  \item No installation required --- works on any computer with internet access
  \item All your work saves automatically to Google Drive
\end{itemize}
\end{frame}

\begin{frame}{What You Need}
\begin{itemize}
  \item A Google account (Gmail works)
  \item A web browser (Chrome recommended)
\end{itemize}
\end{frame}

\begin{frame}{Accessing Colab: From Google Drive}
Click New $\rightarrow$ More $\rightarrow$ Google Colaboratory
\begin{center}
\includegraphics[width=0.8\textwidth]{images/colab_slide4_img2.png}
\end{center}
\end{frame}

\begin{frame}{Accessing Colab: Direct}
\begin{center}
{\Large\href{https://colab.research.google.com}{colab.research.google.com}}
\end{center}
\end{frame}

\begin{frame}{The Open Notebook Dialog}
\begin{center}
\includegraphics[width=0.9\textwidth]{images/colab_slide6_img2.png}
\end{center}
\end{frame}

\begin{frame}{Opening Notebooks}
\begin{itemize}
  \item Open from Examples, Recent, Google Drive, GitHub, or Upload
  \item Click + New notebook to start a fresh notebook
\end{itemize}
\end{frame}

\begin{frame}{The Colab Interface: Notebook + Gemini}
\begin{center}
\includegraphics[width=0.9\textwidth]{images/colab_slide8_img2.png}
\end{center}
\end{frame}

\begin{frame}{How Notebooks Work}
\begin{itemize}
\item Three elements: notebook (.ipynb file), interface (Colab), and Python runtime (kernel)
\item The interface renders your notebook and communicates with the runtime
\item When you run a cell, code goes to the kernel, executes, and results flow back to the interface
\end{itemize}
\end{frame}

\begin{frame}{Navigating a Notebook}
\begin{itemize}
  \item + Code / + Text: Add a new code or text cell
  \item Connect: Connect to Google's servers
  \item Files (folder icon): View and upload files
\end{itemize}
\end{frame}

\begin{frame}{What is a Cell?}
\begin{itemize}
  \item A cell is a box where you write code or text
  \item Code cells run Python code; text cells hold notes and explanations
  \item You can have as many cells as you need
\end{itemize}
\end{frame}

\begin{frame}{Your First Code: Simple Math}
Type 5*3 and press Shift + Enter $\rightarrow$ Result: 15
\begin{center}
\includegraphics[width=0.7\textwidth]{images/colab_slide12_img2.png}
\end{center}
\end{frame}

\begin{frame}{Your First Code: Hello World}
Type print('hello world') and press Shift + Enter
\begin{center}
\includegraphics[width=0.7\textwidth]{images/colab_slide13_img2.png}
\end{center}
\end{frame}

\begin{frame}{Running Code: Three Ways}
\begin{itemize}
  \item Click the play button ($>$) on the left of the cell
  \item Shift + Enter: run and move to next cell
  \item Ctrl + Enter: run and stay in cell
\end{itemize}
\end{frame}

\begin{frame}{Understanding the Play Button}
\begin{itemize}
  \item Play icon ($>$): Cell is ready to run
  \item Spinning circle: Code is executing
  \item Checkmark: Execution complete, output appears below
\end{itemize}
\end{frame}

\begin{frame}{Cell Numbers}
\begin{itemize}
  \item Numbers like [1], [2] show the order cells were run; [ ] means not yet run; [*] means running
  \item You can run cells in any order, but top to bottom avoids confusion
\end{itemize}
\end{frame}

\begin{frame}{Adding New Cells}
\begin{itemize}
  \item Toolbar: Click + Code or + Text
  \item Ctrl + M, B $\rightarrow$ Add cell below
  \item Ctrl + M, A $\rightarrow$ Add cell above
\end{itemize}
\end{frame}

\begin{frame}{Deleting and Moving Cells}
\begin{itemize}
  \item Delete: Trash icon or Ctrl + M, D
  \item Move: Up/down arrows or drag and drop
\end{itemize}
\end{frame}

\begin{frame}{Meet Gemini: Your AI Assistant}
Gemini is built into Colab to help you write code
\begin{center}
\includegraphics[width=0.8\textwidth]{images/colab_slide19_img2.png}
\end{center}
\end{frame}

\begin{frame}{What Gemini Can Do}
\begin{itemize}
  \item Generate code from plain English descriptions
  \item Explain existing code and fix errors
  \item Suggest improvements and answer Python questions
\end{itemize}
\end{frame}

\begin{frame}{Runtime: What Powers Your Code}
\begin{itemize}
  \item When you click Connect, Colab gives you a virtual computer with CPU, RAM, and disk space
  \item Optionally: GPU or TPU for machine learning
\end{itemize}
\end{frame}

\begin{frame}{Restarting the Runtime}
If your code isn't working as expected:

Runtime $\rightarrow$ Restart runtime

This clears all variables and starts fresh. You'll need to re-run your cells after restarting.
\end{frame}

\begin{frame}{Session Limits}
\begin{itemize}
  \item Sessions disconnect after $\sim$90 minutes idle
  \item Maximum $\sim$12 hours continuous use
  \item Limited GPU/TPU hours per week
\end{itemize}
\end{frame}


\begin{frame}{Get Started}
  \begin{enumerate}
  \item Ask Gemini to get monthly GDP data from FRED using the \texttt{pandas-datareader} library.
  \item Ask Gemini to plot GDP over time as a line chart.
  \item Ask Gemini how you can save the chart.
  \item Ask Gemini how you can save the GDP data as a CSV file.
  \item Ask Gemini how you can save the notebook.
  \end{enumerate}
\end{frame}

\begin{frame}{Reading Data Files in Colab}
\begin{itemize}
  \item Download the file to your \textbf{Google Drive}
  \item Ask Gemini to \textbf{mount your Google Drive} and authorize access when prompted
  \item Ask Gemini to load the file with pandas --- it will use the correct path and reader
\end{itemize}

\vspace{0.2cm}
When unsure about anything in Colab, ask Gemini.
\end{frame}

\begin{frame}{Exercise: Computing Returns}
\begin{itemize}
  \item Download \href{https://kerryback.com/mgmt675/files/prices-dividends.xlsx}{prices-dividends.xlsx} to your Google Drive
  \item Ask Gemini to mount your Google Drive
  \item Ask Gemini to compute daily returns including dividends
  \item Ask Gemini to calculate annualized mean return and volatility
\end{itemize}
\end{frame}

\begin{frame}{Exercise: Estimating Betas}
\begin{itemize}
  \item Download \href{https://kerryback.com/mgmt675/files/exercise10-betas.xlsx}{exercise10-betas.xlsx} to your Google Drive
  \item Ask Gemini to mount your Google Drive
  \item Ask Gemini to estimate betas for each stock using regression
  \item Ask Gemini to interpret the results
\end{itemize}
\end{frame}

\begin{frame}{Exercise: Pairs Trading Signals}
\begin{itemize}
  \item Download \href{https://kerryback.com/mgmt675/files/pairs-ko-pep.xlsx}{pairs-ko-pep.xlsx} (or \href{https://kerryback.com/mgmt675/files/pairs-f-gm.xlsx}{pairs-f-gm.xlsx}) to your Google Drive
  \item Ask Gemini to mount your Google Drive and load the price data
  \item Ask Gemini to compute the price ratio of the two stocks and plot it over time
  \item Ask Gemini to compute the rolling 30-day z-score of the ratio and plot it, with horizontal lines at $\pm 1$ and $\pm 2$
  \item Ask Gemini to highlight dates where the z-score crosses $\pm 2$ (potential entry signals) and where it returns to 0 (potential exit signals)
\end{itemize}
\end{frame}

\end{document}
